\chapter[THE SIXTH STAGE]{}

Now when they were got almost quite out of this wilderness, Faithful chanced to cast his eye back, and espied one coming after them, and he knew him. Oh! said Faithful to his brother, who comes yonder? Then Christian looked, and said, It is my good friend Evangelist. Aye, and my good friend too, said Faithful, for twas he that set me on the way to the gate. Now was Evangelist come up unto them, and thus saluted them.

Evangelist: Peace be with you, dearly beloved, and peace be to your helpers.

Christian: Welcome, welcome, my good Evangelist: the sight of thy countenance brings to my remembrance thy ancient kindness and unwearied labors for my eternal good.

Faithful: And a thousand times welcome, said good Faithful, thy company, O sweet Evangelist; how desirable is it to us poor pilgrims!

Evangelist: Then said Evangelist, How hath it fared with you, my friends, since the time of our last parting? What have you met with, and how have you behaved yourselves?

Then Christian and Faithful told him of all things that had happened to them in the way; and how, and with what difficulty, they had arrived to that place.

Right glad am I, said Evangelist, not that you have met with trials, but that you have been victors, and for that you have, notwithstanding many weaknesses, continued in the way to this very day.

I say, right glad am I of this thing, and that for mine own sake and yours: I have sowed, and you have reaped; and the day is coming, when ``both he that soweth, and they that reap, shall rejoice together," John 4:36; that is, if you hold out: ``for in due season ye shall reap, if ye faint not." Gal. 6:9. The crown is before you, and it is an incorruptible one; ``so run that ye may obtain it." 1 Cor. 9:24-27. Some there be that set out for this crown, and after they have gone far for it, another comes in and takes it from them: ``hold fast, therefore, that you have; let no man take your crown." Rev. 3:11. You are not yet out of the gunshot of the devil; ``you have not resisted unto blood, striving against sin." Let the kingdom be always before you, and believe steadfastly concerning the things that are invisible. Let nothing that is on this side the other world get within you. And, above all, look well to your own hearts and to the lusts thereof; for they are ``deceitful above all things, and desperately wicked." Set your faces like a flint; you have all power in heaven and earth on your side.

Christian: Then Christian thanked him for his exhortations; but told him withal, that they would have him speak farther to them for their help the rest of the way; and the rather, for that they well knew that he was a prophet, and could tell them of things that might happen unto them, and also how they might resist and overcome them. To which request Faithful also consented. So Evangelist began as followeth.

Evangelist: My sons, you have heard in the word of the truth of the Gospel, that you must ``through many tribulations enter into the kingdom of heaven;" and again, that ``in every city, bonds and afflictions abide you;" and therefore you cannot expect that you should go long on your pilgrimage without them, in some sort or other. You have found something of the truth of these testimonies upon you already, and more will immediately follow: for now, as you see, you are almost out of this wilderness, and therefore you will soon come into a town that you will by and by see before you; and in that town you will be hardly beset with enemies, who will strain hard but they will kill you; and be you sure that one or both of you must seal the testimony which you hold, with blood; but ``be you faithful unto death, and the King will give you a crown of life." He that shall die there, although his death will be unnatural, and his pain, perhaps, great, he will yet have the better of his fellow; not only because he will be arrived at the Celestial City soonest, but because he will escape many miseries that the other will meet with in the rest of his journey. But when you are come to the town, and shall find fulfilled what I have here related, then remember your friend, and quit yourselves like men, and ``commit the keeping of your souls to God in well doing, as unto a faithful Creator."

Then I saw in my dream, that when they were got out of the wilderness, they presently saw a town before them, and the name of that town is Vanity; and at the town there is a fair kept, called Vanity Fair. It is kept all the year long. It beareth the name of Vanity Fair, because the town where it is kept is lighter than vanity, Psa. 62:9; and also because all that is there sold, or that cometh thither, is vanity; as is the saying of the wise, ``All that cometh is vanity." Eccl. 11:8; see also 1:2-14; 2:11-17; Isa. 40:17.

This fair is no new-erected business but a thing of ancient standing. I will show you the original of it.

Almost five thousand years ago there were pilgrims walking to the Celestial City, as these two honest persons are: and Beelzebub, Apollyon, and Legion, with their companions, perceiving by the path that the pilgrims made, that their way to the city lay through this town of Vanity, they contrived here to set up a fair; a fair wherein should be sold all sorts of vanity, and that it should last all the year long. Therefore, at this fair are all such merchandise sold as houses, lands, trades, places, honors, preferments, titles, countries, kingdoms, lusts, pleasures; and delights of all sorts, as harlots, wives, husbands, children, masters, servants, lives, blood, bodies, souls, silver, gold, pearls, precious stones, and what not.

And moreover, at this fair there is at all times to be seen jugglings, cheats, games, plays, fools, apes, knaves, and rogues, and that of every kind.

Here are to be seen, too, and that for nothing, thefts, murders, adulteries, false-swearers, and that of a blood-red color.

And, as in other fairs of less moment, there are the several rows and streets under their proper names, where such and such wares are vended; so here, likewise, you have the proper places, rows, streets, (namely, countries and kingdoms,) where the wares of this fair are soonest to be found. Here is the Britain Row, the French Row, the Italian Row, the Spanish Row, the German Row, where several sorts of vanities are to be sold. But, as in other fairs, some one commodity is as the chief of all the fair; so the ware of Rome and her merchandise is greatly promoted in this fair; only our English nation, with some others, have taken a dislike thereat.

Now, as I said, the way to the Celestial City lies just through this town, where this lusty fair is kept; and he that will go to the city, and yet not go through this town, ``must needs go out of the world." 1 Cor. 4:10. The Prince of princes himself, when here, went through this town to his own country, and that upon a fair-day too; yea, and, as I think, it was Beelzebub, the chief lord of this fair, that invited him to buy of his vanities, yea, would have made him lord of the fair, would he but have done him reverence as he went through the town. Yea, because he was such a person of honor, Beelzebub had him from street to street, and showed him all the kingdoms of the world in a little time, that he might, if possible, allure that blessed One to cheapen and buy some of his vanities; but he had no mind to the merchandise, and therefore left the town, without laying out so much as one farthing upon these vanities. Matt. 4:8,9; Luke 4:5-7. This fair, therefore, is an ancient thing, of long standing, and a very great fair.

Now, these pilgrims, as I said, must needs go through this fair. Well, so they did; but behold, even as they entered into the fair, all the people in the fair were moved; and the town itself, as it were, in a hubbub about them, and that for several reasons: for,

First, The Pilgrims were clothed with such kind of raiment as was diverse from the raiment of any that traded in that fair. The people, therefore, of the fair made a great gazing upon them: some said they were fools; 1 Cor. 4:9,10; some, they were bedlams; and some, they were outlandish men.

Secondly, And as they wondered at their apparel, so they did likewise at their speech; for few could understand what they said. They naturally spoke the language of Canaan; but they that kept the fair were the men of this world: so that from one end of the fair to the other, they seemed barbarians each to the other. 1 Cor. 2:7,8.

Thirdly, But that which did not a little amuse the merchandisers was, that these pilgrims set very light by all their wares. They cared not so much as to look upon them; and if they called upon them to buy, they would put their fingers in their ears, and cry, ``Turn away mine eyes from beholding vanity," Psa. 119:37, and look upward, signifying that their trade and traffic was in heaven. Phil. 3: 20,21.

One chanced, mockingly, beholding the carriage of the men, to say unto them, ``What will ye buy?" But they, looking gravely upon him, said, ``We buy the truth." Prov. 23:23. At that there was an occasion taken to despise the men the more; some mocking, some taunting, some speaking reproachfully, and some calling upon others to smite them. At last, things came to an hubbub and great stir in the fair, insomuch that all order was confounded. Now was word presently brought to the great one of the fair, who quickly came down, and deputed some of his most trusty friends to take those men into examination about whom the fair was almost overturned. So the men were brought to examination; and they that sat upon them asked them whence they came, whither they went, and what they did there in such an unusual garb. The men told them they were pilgrims and strangers in the world, and that they were going to their own country, which was the heavenly Jerusalem, Heb. 11:13-16; and that they had given no occasion to the men of the town, nor yet to the merchandisrs, thus to abuse them, and to let them in their journey, except it was for that, when one asked them what they would buy, they said they would buy the truth. But they that were appointed to examine them did not believe them to be any other than bedlams and mad, or else such as came to put all things into a confusion in the fair. Therefore they took them and beat them, and besmeared them with dirt, and then put them into the cage, that they might be made a spectacle to all the men of the fair. There, therefore, they lay for some time, and were made the objects of any man's sport, or malice, or revenge; the great one of the fair laughing still at all that befell them. But the men being patient, and ``not rendering railing for railing, but contrariwise blessing," and giving good words for bad, and kindness for injuries done, some men in the fair, that were more observing and less prejudiced than the rest, began to check and blame the baser sort for their continual abuses done by them to the men. They, therefore, in an angry manner let fly at them again, counting them as bad as the men in the cage, and telling them that they seemed confederates, and should be made partakers of their misfortunes. The others replied that, for aught they could see, the men were quiet and sober, and intended nobody any harm; and that there were many that traded in their fair that were more worthy to be put into the cage, yea, and pillory too, than were the men that they had abused. Thus, after divers words had passed on both sides, (the men behaving themselves all the while very wisely and soberly before them,) they fell to some blows among themselves, and did harm one to another. Then were these two poor men brought before their examiners again, and were charged as being guilty of the late hubbub that had been in the fair. So they beat them pitifully, and hanged irons upon them, and led them in chains up and down the fair, for an example and terror to others, lest any should speak in their behalf, or join themselves unto them. But Christian and Faithful behaved themselves yet more wisely, and received the ignominy and shame that was cast upon them with so much meekness and patience, that it won to their side (though but few in comparison of the rest) several of the men in the fair. This put the other party yet into a greater rage, insomuch that they concluded the death of these two men. Wherefore they threatened that neither cage nor irons should serve their turn, but that they should die for the abuse they had done, and for deluding the men of the fair.

Then were they remanded to the cage again, until further order should be taken with them. So they put them in, and made their feet fast in the stocks.

Here, also, they called again to mind what they had heard from their faithful friend Evangelist, and were the more confirmed in their way and sufferings by what he told them would happen to them. They also now comforted each other, that whose lot it was to suffer, even he should have the best of it: therefore each man secretly wished that he might have that preferment. But committing themselves to the all-wise disposal of Him that ruleth all things, with much content they abode in the condition in which they were, until they should be otherwise disposed of.

Then a convenient time being appointed, they brought them forth to their trial, in order to their condemnation. When the time was come, they were brought before their enemies and arraigned. The judge's name was Lord Hate-good; their indictment was one and the same in substance, though somewhat varying in form; the contents whereof was this: ``That they were enemies to, and disturbers of, the trade; that they had made commotions and divisions in the town, and had won a party to their own most dangerous opinions, in contempt of the law of their prince."

Then Faithful began to answer, that he had only set himself against that which had set itself against Him that is higher than the highest. And, said he, as for disturbance, I make none, being myself a man of peace: the parties that were won to us, were won by beholding our truth and innocence, and they are only turned from the worse to the better. And as to the king you talk of, since he is Beelzebub, the enemy of our Lord, I defy him and all his angels.

Then proclamation was made, that they that had ought to say for their lord the king against the prisoner at the bar, should forthwith appear, and give in their evidence. So there came in three witnesses, to wit, Envy, Superstition, and Pickthank. They were then asked if they knew the prisoner at the bar; and what they had to say for their lord the king against him.

Then stood forth Envy, and said to this effect: My lord, I have known this man a long time, and will attest upon my oath before this honorable bench, that he is-

Judge: Hold; give him his oath.

So they sware him. Then he said, My lord, this man, notwithstanding his plausible name, is one of the vilest men in our country; he neither regardeth prince nor people, law nor custom, but doeth all that he can to possess all men with certain of his disloyal notions, which he in the general calls principles of faith and holiness. And in particular, I heard him once myself affirm, that Christianity and the customs of our town of Vanity were diametrically opposite, and could not be reconciled. By which saying, my lord, he doth at once not only condemn all our laudable doings, but us in the doing of them.

Then did the judge say to him, Hast thou any more to say?

Envy: My lord, I could say much more, only I would not be tedious to the court. Yet if need be, when the other gentlemen have given in their evidence, rather than any thing shall be wanting that will dispatch him, I will enlarge my testimony against him. So he was bid to stand by.

Then they called Superstition, and bid him look upon the prisoner. They also asked, what he could say for their lord the king against him. Then they sware him; so he began.

Superstition: My lord, I have no great acquaintance with this man, nor do I desire to have further knowledge of him. However, this I know, that he is a very pestilent fellow, from some discourse that I had with him the other day, in this town; for then, talking with him, I heard him say, that our religion was naught, and such by which a man could by no means please God. Which saying of his, my lord, your lordship very well knows what necessarily thence will follow, to wit, that we still do worship in vain, are yet in our sins, and finally shall be damned: and this is that which I have to say.

Then was Pickthank sworn, and bid say what he knew in the behalf of their lord the king against the prisoner at the bar.

Pickthank: My lord, and you gentlemen all, this fellow I have known of a long time, and have heard him speak things that ought not to be spoken; for he hath railed on our noble prince Beelzebub, and hath spoken contemptibly of his honorable friends, whose names are, the Lord Old Man, the Lord Carnal Delight, the Lord Luxurious, the Lord Desire of Vain Glory, my old Lord Lechery, Sir Having Greedy, with all the rest of our nobility: and he hath said, moreover, that if all men were of his mind, if possible, there is not one of these noblemen should have any longer a being in this town. Besides, he hath not been afraid to rail on you, my lord, who are now appointed to be his judge, calling you an ungodly villain, with many other such like vilifying terms, with which he hath bespattered most of the gentry of our town.

When this Pickthank had told his tale, the judge directed his speech to the prisoner at the bar, saying, Thou runagate, heretic, and traitor, hast thou heard what these honest gentlemen have witnessed against thee?

Faithful: May I speak a few words in my own defence?

Judge: Sirrah, sirrah, thou deservest to live no longer, but to be slain immediately upon the place; yet, that all men may see our gentleness towards thee, let us hear what thou, vile runagate, hast to say.

Faithful: 1. I say, then, in answer to what Mr. Envy hath spoken, I never said aught but this, that what rule, or laws, or custom, or people, were flat against the word of God, are diametrically opposite to Christianity. If I have said amiss in this, convince me of my error, and I am ready here before you to make my recantation.

2. As to the second, to wit, Mr. Superstition, and his charge against me, I said only this, that in the worship of God there is required a divine faith; but there can be no divine faith without a divine revelation of the will of God. Therefore, whatever is thrust into the worship of God that is not agreeable to divine revelation, cannot be done but by a human faith; which faith will not be profitable to eternal life.

3. As to what Mr. Pickthank hath said, I say, (avoiding terms, as that I am said to rail, and the like,) that the prince of this town, with all the rabblement, his attendants, by this gentleman named, are more fit for a being in hell than in this town and country. And so the Lord have mercy upon me.

Then the judge called to the jury, (who all this while stood by to hear and observe,) Gentlemen of the jury, you see this man about whom so great an uproar hath been made in this town; you have also heard what these worthy gentlemen have witnessed against him; also, you have heard his reply and confession: it lieth now in your breasts to hang him, or save his life; but yet I think meet to instruct you in our law.

There was an act made in the days of Pharaoh the Great, servant to our prince, that, lest those of a contrary religion should multiply and grow too strong for him, their males should be thrown into the river. Exod. 1:22. There was also an act made in the days of Nebuchadnezzar the Great, another of his servants, that whoever would not fall down and worship his golden image, should be thrown into a fiery furnace. Dan. 3:6. There was also an act made in the days of Darius, that whoso for some time called upon any god but him, should be cast into the lion's den. Dan. 6:7. Now, the substance of these laws this rebel has broken, not only in thought, (which is not to be borne,) but also in word and deed; which must, therefore, needs be intolerable.

For that of Pharaoh, his law was made upon a supposition to prevent mischief, no crime being yet apparent; but here is a crime apparent. For the second and third, you see he disputeth against our religion; and for the treason that he hath already confessed, he deserveth to die the death.

Then went the jury out, whose names were Mr. Blindman, Mr. No-good, Mr. Malice, Mr. Love-lust, Mr. Live-loose, Mr. Heady, Mr. High-mind, Mr. Enmity, Mr. Liar, Mr. Cruelty, Mr. Hate-light, and Mr. Implacable; who every one gave in his private verdict against him among themselves, and afterwards unanimously concluded to bring him in guilty before the judge. And first among themselves, Mr. Blindman, the foreman, said, I see clearly that this man is a heretic. Then said Mr. No-good, Away with such a fellow from the earth. Aye, said Mr. Malice, for I hate the very looks of him. Then said Mr. Love-lust, I could never endure him. Nor I, said Mr. Live-loose, for he would always be condemning my way. Hang him, hang him, said Mr. Heady. A sorry scrub, said Mr. High-mind. My heart riseth against him, said Mr. Enmity. He is a rogue, said Mr. Liar. Hanging is too good for him, said Mr. Cruelty. Let us dispatch him out of the way, said Mr. Hate-light. Then said Mr. Implacable, Might I have all the world given me, I could not be reconciled to him; therefore let us forthwith bring him in guilty of death.

And so they did; therefore he was presently condemned to be had from the place where he was, to the place from whence he came, and there to be put to the most cruel death that could be invented.

They therefore brought him out, to do with him according to their law; and first they scourged him, then they buffeted him, then they lanced his flesh with knives; after that, they stoned him with stones, then pricked him with their swords; and last of all, they burned him to ashes at the stake. Thus came Faithful to his end.

Now I saw, that there stood behind the multitude a chariot and a couple of horses waiting for Faithful, who (so soon as his adversaries had dispatched him) was taken up into it, and straightway was carried up through the clouds with sound of trumpet, the nearest way to the celestial gate. But as for Christian, he had some respite, and was remanded back to prison: so he there remained for a space. But he who overrules all things, having the power of their rage in his own hand, so wrought it about, that Christian for that time escaped them, and went his way.

And as he went, he sang, saying,
\begin{verse}
``Well, Faithful, thou hast faithfully profest\\
Unto thy Lord, with whom thou shalt be blest,\\
When faithless ones, with all their vain delights,\\
Are crying out under their hellish plights:\\
Sing, Faithful, sing, and let thy name survive;\\
For though they killed thee, thou art yet alive."\\
\end{verse}
