\chapter[THE FOURTH STAGE]{}

Then he began to go forward; but Discretion, Piety, Charity, and Prudence would accompany him down to the foot of the hill. So they went on together, reiterating their former discourses, till they came to go down the hill. Then said Christian, As it was difficult coming up, so, so far as I can see, it is dangerous going down. Yes, said Prudence, so it is; for it is a hard matter for a man to go down into the valley of Humiliation, as thou art now, and to catch no slip by the way; therefore, said they, we are come out to accompany thee down the hill. So he began to go down, but very warily; yet he caught a slip or two.

Then I saw in my dream, that these good companions, when Christian was got down to the bottom of the hill, gave him a loaf of bread, a bottle of wine, and a cluster of raisins; and then he went on his way, 
\begin{verse} 
``Whilst Christian is among his godly friends,\\
Their golden mouths make him sufficient mends\\
For all his griefs; and when they let him go,\\
He's clad with northern steel from top to toe."\\
\end{verse} 
But now, in this valley of Humiliation, poor Christian was hard put to it; for he had gone but a little way before he espied a foul fiend coming over the field to meet him: his name is Apollyon. Then did Christian begin to be afraid, and to cast in his mind whether to go back, or to stand his ground. But he considered again, that he had no armor for his back, and therefore thought that to turn the back to him might give him greater advantage with ease to pierce him with his darts; therefore he resolved to venture and stand his ground: for, thought he, had I no more in mine eye than the saving of my life, it would be the best way to stand.

So he went on, and Apollyon met him. Now the monster was hideous to behold: he was clothed with scales like a fish, and they are his pride; he had wings like a dragon, and feet like a bear, and out of his belly came fire and smoke; and his mouth was as the mouth of a lion. When he was come up to Christian, he beheld him with a disdainful countenance, and thus began to question him.

Apollyon: Whence came you, and whither are you bound?

Christian: I am come from the city of Destruction, which is the place of all evil, and I am going to the city of Zion.

Apollyon: By this I perceive thou art one of my subjects; for all that country is mine, and I am the prince and god of it. How is it, then, that thou hast run away from thy king? Were it not that I hope thou mayest do me more service, I would strike thee now at one blow to the ground.

Christian: I was, indeed, born in your dominions, but your service was hard, and your wages such as a man could not live on; for the wages of sin is death, Rom. 6:23; therefore, when I was come to years, I did, as other considerate persons do, look out if perhaps I might mend myself.

Apollyon: There is no prince that will thus lightly lose his subjects, neither will I as yet lose thee; but since thou complainest of thy service and wages, be content to go back, and what our country will afford I do here promise to give thee.

Christian: But I have let myself to another, even to the King of princes; and how can I with fairness go back with thee?

Apollyon: Thou hast done in this according to the proverb, ``changed a bad for a worse;" but it is ordinary for those that have professed themselves his servants, after a while to give him the slip, and return again to me. Do thou so to, and all shall be well.

Christian: I have given him my faith, and sworn my allegiance to him; how then can I go back from this, and not be hanged as a traitor.

Apollyon: Thou didst the same by me, and yet I am willing to pass by all, if now thou wilt yet turn again and go back.

Christian: What I promised thee was in my non-age: and besides, I count that the Prince, under whose banner I now stand, is able to absolve me, yea, and to pardon also what I did as to my compliance with thee. And besides, O thou destroying Apollyon, to speak truth, I like his service, his wages, his servants, his government, his company, and country, better than thine; therefore leave off to persuade me farther: I am his servant, and I will follow him.

Apollyon: Consider again, when thou art in cool blood, what thou art like to meet with in the way that thou goest. Thou knowest that for the most part his servants come to an ill end, because they are transgressors against me and my ways. How many of them have been put to shameful deaths! And besides, thou countest his service better than mine; whereas he never yet came from the place where he is, to deliver any that served him out of their enemies' hands: but as for me, how many times, as all the world very well knows, have I delivered, either by power or fraud, those that have faithfully served me, from him and his, though taken by them! And so will I deliver thee.

Christian: His forbearing at present to deliver them, is on purpose to try their love, whether they will cleave to him to the end: and as for the ill end thou sayest they come to, that is most glorious in their account. For, for present deliverance, they do not much expect it; for they stay for their glory; and then they shall have it, when their Prince comes in his and the glory of the angels.

Apollyon: Thou hast already been unfaithful in thy service to him; and how dost thou think to receive wages of him?

Christian: Wherein, O Apollyon, have I been unfaithful to him?

Apollyon: Thou didst faint at first setting out, when thou wast almost choked in the gulf of Despond. Thou didst attempt wrong ways to be rid of thy burden, whereas thou shouldst have stayed till thy Prince had taken it off. Thou didst sinfully sleep, and lose thy choice things. Thou wast almost persuaded also to go back at the sight of the lions. And when thou talkest of thy journey, and of what thou hast seen and heard, thou art inwardly desirous of vainglory in all that thou sayest or doest.

Christian: All this is true, and much more which thou hast left out; but the Prince whom I serve and honor is merciful, and ready to forgive. But besides, these infirmities possessed me in thy country, for there I sucked them in, and I have groaned under them, been sorry for them, and have obtained pardon of my Prince.

Apollyon: Then Apollyon broke out into a grievous rage, saying, I am an enemy to this Prince; I hate his person, his laws, and people: I am come out on purpose to withstand thee.

Christian: Apollyon, beware what you do, for I am in the King's highway, the way of holiness; therefore take heed to yourself.

Apollyon: Then Apollyon straddled quite over the whole breadth of the way, and said, I am void of fear in this matter. Prepare thyself to die; for I swear by my infernal den, that thou shalt go no farther: here will I spill thy soul. And with that he threw a flaming dart at his breast; but Christian had a shield in his hand, with which he caught it, and so prevented the danger of that.

Then did Christian draw, for he saw it was time to bestir him; and Apollyon as fast made at him, throwing darts as thick as hail; by the which, notwithstanding all that Christian could do to avoid it, Apollyon wounded him in his head, his hand, and foot. This made Christian give a little back: Apollyon, therefore, followed his work amain, and Christian again took courage, and resisted as manfully as he could. This sore combat lasted for above half a day, even till Christian was almost quite spent: for you must know, that Christian, by reason of his wounds, must needs grow weaker and weaker.

Then Apollyon, espying his opportunity, began to gather up close to Christian, and wrestling with him, gave him a dreadful fall; and with that Christian's sword flew out of his hand. Then said Apollyon, I am sure of thee now: and with that he had almost pressed him to death, so that Christian began to despair of life. But, as God would have it, while Apollyon was fetching his last blow, thereby to make a full end of this good man, Christian nimbly reached out his hand for his sword, and caught it, saying, Rejoice not against me, O mine enemy: when I fall, I shall arise, Mic. 7:8; and with that gave him a deadly thrust, which made him give back, as one that had received his mortal wound. Christian perceiving that, made at him again, saying, Nay, in all these things we are more than conquerors, through Him that loved us. Rom. 8:37. And with that Apollyon spread forth his dragon wings, and sped him away, that Christian saw him no more. James 4:7.

In this combat no man can imagine, unless he had seen and heard, as I did, what yelling and hideous roaring Apollyon made all the time of the fight; he spake like a dragon: and on the other side, what sighs and groans burst from Christian's heart. I never saw him all the while give so much as one pleasant look, till he perceived he had wounded Apollyon with his two-edged sword; then, indeed, he did smile, and look upward! But it was the dreadfullest sight that ever I saw.

So when the battle was over, Christian said, I will here give thanks to him that hath delivered me out of the mouth of the lion, to him that did help me against Apollyon. And so he did, saying,
\begin{verse} 
``Great Beelzebub, the captain of this fiend,\\
Designed my ruin; therefore to this end\\
He sent him harness'd out; and he, with rage\\
That hellish was, did fiercely me engage:\\
But blessed Michael helped me, and I,\\
By dint of sword, did quickly make him fly:\\
Therefore to Him let me give lasting praise,\\
And thank and bless his holy name always." \\
\end{verse} 
Then there came to him a hand with some of the leaves of the tree of life, the which Christian took and applied to the wounds that he had received in the battle, and was healed immediately. He also sat down in that place to eat bread, and to drink of the bottle that was given him a little before: so, being refreshed, he addressed himself to his journey with his sword drawn in his hand; for he said, I know not but some other enemy may be at hand. But he met with no other affront from Apollyon quite through this valley.

Now at the end of this valley was another, called the Valley of the Shadow of Death; and Christian must needs go through it, because the way to the Celestial City lay through the midst of it. Now, this valley is a very solitary place. The prophet Jeremiah thus describes it: ``A wilderness, a land of deserts and pits, a land of drought, and of the Shadow of Death, a land that no man" (but a Christian) ``passeth through, and where no man dwelt." Jer. 2:6.

Now here Christian was worse put to it than in his fight with Apollyon, as by the sequel you shall see.

I saw then in my dream, that when Christian was got to the borders of the Shadow of Death, there met him two men, children of them that brought up an evil report of the good land Num.13:32, making haste to go back; to whom Christian spake as follows.

Christian: Whither are you going?

The Two Men: They said, Back, back; and we would have you do so too, if either life or peace is prized by you.

Christian: Why, what's the matter? said Christian.

The Two Men: Matter! said they; we were going that way as you are going, and went as far as we durst: and indeed we were almost past coming back; for had we gone a little further, we had not been here to bring the news to thee.

Christian: But what have you met with? said Christian.

The Two Men: Why, we were almost in the Valley of the Shadow of Death, but that by good hap we looked before us, and saw the danger before we came to it. Psa. 44:19; 107:19.

Christian: But what have you seen? said Christian.

The Two Men: Seen! why the valley itself, which is as dark as pitch: we also saw there the hobgoblins, satyrs, and dragons of the pit: we heard also in that valley a continual howling and yelling, as of a people under unutterable misery, who there sat bound in affliction and irons: and over that valley hang the discouraging clouds of confusion: Death also doth always spread his wings over it. In a word, it is every whit dreadful, being utterly without order. Job 3:5; 10:22.

Christian: Then, said Christian, I perceive not yet, by what you have said, but that this is my way to the desired haven. Psalm 44:18,19; Jer. 2:6.

The Two Men: Be it thy way; we will not choose it for ours.

So they parted, and Christian went on his way, but still with his sword drawn in his hand, for fear lest he should be assaulted.

I saw then in my dream, so far as this valley reached, there was on the right hand a very deep ditch; that ditch is it into which the blind have led the blind in all ages, and have both there miserably perished. Again, behold, on the left hand there was a very dangerous quag, into which, if even a good man falls, he finds no bottom for his foot to stand on: into that quag king David once did fall, and had no doubt therein been smothered, had not He that is able plucked him out. Psa. 69:14.

The pathway was here also exceeding narrow, and therefore good Christian was the more put to it; for when he sought, in the dark, to shun the ditch on the one hand, he was ready to tip over into the mire on the other; also, when he sought to escape the mire, without great carefulness he would be ready to fall into the ditch. Thus he went on, and I heard him here sigh bitterly; for besides the danger mentioned above, the pathway was here so dark, that ofttimes when he lifted up his foot to go forward, he knew not where, or upon what he should set it next.

About the midst of this valley I perceived the mouth of hell to be, and it stood also hard by the wayside. Now, thought Christian, what shall I do? And ever and anon the flame and smoke would come out in such abundance, with sparks and hideous noises, (things that cared not for Christian's sword, as did Apollyon before,) that he was forced to put up his sword, and betake himself to another weapon, called All-prayer, Eph. 6:18; so he cried, in my hearing, O Lord, I beseech thee, deliver my soul. Psa. 116:4. Thus he went on a great while, yet still the flames would be reaching towards him; also he heard doleful voices, and rushings to and fro, so that sometimes he thought he should be torn in pieces, or trodden down like mire in the streets. This frightful sight was seen, and these dreadful noises were heard by him for several miles together; and coming to a place where he thought he heard a company of fiends coming forward to meet him, he stopped, and began to muse what he had best to do. Sometimes he had half a thought to go back; then again he thought he might be half-way through the valley. He remembered also, how he had already vanquished many a danger; and that the danger of going back might be much more than for to go forward. So he resolved to go on; yet the fiends seemed to come nearer and nearer. But when they were come even almost at him, he cried out with a most vehement voice, I will walk in the strength of the Lord God. So they gave back, and came no farther.

One thing I would not let slip. I took notice that now poor Christian was so confounded that he did not know his own voice; and thus I perceived it. Just when he was come over against the mouth of the burning pit, one of the wicked ones got behind him, and stepped up softly to him, and whisperingly suggested many grievous blasphemies to him, which he verily thought had proceeded from his own mind. This put Christian more to it than any thing that he met with before, even to think that he should now blaspheme Him that he loved so much before. Yet if he could have helped it, he would not have done it; but he had not the discretion either to stop his ears, or to know from whence these blasphemies came.

When Christian had travelled in this disconsolate condition some considerable time, he thought he heard the voice of a man, as going before him, saying, Though I walk through the Valley of the Shadow of Death, I will fear no evil, for thou art with me. Psa. 23:4.

Then was he glad, and that for these reasons:

First, Because he gathered from thence, that some who feared God were in this valley as well as himself.

Secondly, For that he perceived God was with them, though in that dark and dismal state. And why not, thought he, with me? though by reason of the impediment that attends this place, I cannot perceive it. Job 9:11.

Thirdly, For that he hoped (could he overtake them) to have company by and by.

So he went on, and called to him that was before; but he knew not what to answer, for that he also thought himself to be alone. And by and by the day broke: then said Christian, ``He hath turned the shadow of death into the morning." Amos 5:8.

Now morning being come, he looked back, not out of desire to return, but to see, by the light of the day, what hazards he had gone through in the dark. So he saw more perfectly the ditch that was on the one hand, and the quag that was on the other; also how narrow the way was which led betwixt them both. Also now he saw the hobgoblins, and satyrs, and dragons of the pit, but all afar off; for after break of day they came not nigh; yet they were discovered to him, according to that which is written, ``He discovereth deep things out of darkness, and bringeth out to light the shadow of death." Job 12:22.

Now was Christian much affected with this deliverance from all the dangers of his solitary way; which dangers, though he feared them much before, yet he saw them more clearly now, because the light of the day made them conspicuous to him. And about this time the sun was rising, and this was another mercy to Christian; for you must note, that though the first part of the Valley of the Shadow of Death was dangerous, yet this second part, which he was yet to go, was, if possible, far more dangerous; for, from the place where he now stood, even to the end of the valley, the way was all along set so full of snares, traps, gins, and nets here, and so full of pits, pitfalls, deep holes, and shelvings-down there, that had it now been dark, as it was when he came the first part of the way, had he had a thousand souls, they had in reason been cast away; but, as I said, just now the sun was rising. Then said he, ``His Candle shineth on my head, and by his light I go through darkness." Job 29:3.

In this light, therefore, he came to the end of the valley. Now I saw in my dream, that at the end of the valley lay blood, bones, ashes, and mangled bodies of men, even of pilgrims that had gone this way formerly; and while I was musing what should be the reason, I espied a little before me a cave, where two giants, Pope and Pagan, dwelt in old times; by whose power and tyranny the men whose bones, blood, ashes, etc., lay there, were cruelly put to death. But by this place Christian went without much danger, whereat I somewhat wondered; but I have learnt since, that Pagan has been dead many a day; and as for the other, though he be yet alive, he is, by reason of age, and also of the many shrewd brushes that he met with in his younger days, grown so crazy and stiff in his joints that he can now do little more than sit in his cave's mouth, grinning at pilgrims as they go by, and biting his nails because he cannot come at them.

So I saw that Christian went on his way; yet, at the sight of the old man that sat at the mouth of the cave, he could not tell what to think, especially because he spoke to him, though he could not go after him, saying, You will never mend, till more of you be burned. But he held his peace, and set a good face on it; and so went by, and catched no hurt. Then sang Christian,

\begin{verse} 
``O world of wonders, (I can say no less,)\\
That I should be preserved in that distress\\
That I have met with here! O blessed be\\
That hand that from it hath delivered me!\\
Dangers in darkness, devils, hell, and sin,\\
Did compass me, while I this vale was in;\\
Yea, snares, and pits, and traps, and nets did lie\\
My path about, that worthless, silly I\\
Might have been catch'd, entangled, and cast down;\\
But since I live, let Jesus wear the crown."\\
\end{verse} 
