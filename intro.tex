\newpage
\subsubsection[INTRODUCTION]{}
Courteous companions,

Some time since, to tell you my dream that I had of Christian the pilgrim, and of his dangerous journey towards the Celestial country, was pleasant to me and profitable to you. I told you then also what I saw concerning his wife and children, and how unwilling they were to go with him on pilgrimage; insomuch that he was forced to go on his progress without them; for he durst not run the danger of that destruction which he feared would come by staying with them in the City of Destruction: wherefore, as I then showed you, he left them and departed.

Now it hath so happened, through the multiplicity of business, that I have been much hindered and kept back from my wonted travels into those parts whence he went, and so could not, till now, obtain an opportunity to make further inquiry after those whom he left behind, that I might give you an account of them. But having had some concerns that way of late, I went down again thitherward. Now, having taken up my lodging in a wood about a mile off the place, as I slept, I dreamed again.

And as I was in my dream, behold, an aged gentleman came by where I lay; and, because he was to go some part of the way that I was traveling, methought I got up and went with him. So, as we walked, and as travelers usually do, I was as if we fell into a discourse; and our talk happened to be about Christian and his travels; for thus I began with the old man:

Sir, said I, what town is that there below, that lieth on the left hand of our way?

Then said Mr. Sagacity, (for that was his name,) It is the City of Destruction, a populous place, but possessed with a very ill-conditioned and idle sort of people.

I thought that was that city, quoth I; I went once myself through that town; and therefore know that this report you give of it is true.

Mr. Sagacity: Too true! I wish I could speak truth in speaking better of them that dwell therein.

Well, sir, quoth I, then I perceive you to be a well-meaning man, and so one that takes pleasure to hear and tell of that which is good. Pray, did you never hear what happened to a man some time ago of this town, (whose name was Christian,) that went on a pilgrimage up towards the higher regions?

Mr. Sagacity: Hear of him! Aye, and I also heard of the molestations, troubles, wars, captivities, cries, groans, frights, and fears, that he met with and had on his journey. Besides, I must tell you, all our country rings of him; there are but few houses that have heard of him and his doings, but have sought after and got the records of his pilgrimage; yea, I think I may say that his hazardous journey has got many well-wishers to his ways; for, though when he was here he was fool in every man's mouth, yet now he is gone he is highly commended of all. For tis said he lives bravely where he is: yea, many of them that are resolved never to run his hazards, yet have their mouths water at his gains.

They may, quoth I, well think, if they think any thing that is true, that he liveth well where he is; for he now lives at, and in the fountain of life, and has what he has without labor and sorrow, for there is no grief mixed therewith. But, pray what talk have the people about him?

Mr. Sagacity: Talk! the people talk strangely about him: some say that he now walks in white, Rev. 3:4; that he has a chain of gold about his neck; that he has a crown of gold, beset with pearls, upon his head: others say, that the shining ones, who sometimes showed themselves to him in his journey, are become his companions, and that he is as familiar with them where he is, as here one neighbor is with another. Besides, it is confidently affirmed concerning him, that the King of the place where he is has bestowed upon him already a very rich and pleasant dwelling at court, and that he every day eateth and drinketh, and walketh and talketh with him, and receiveth of the smiles and favors of him that is Judge of all there. Zech. 3:7; Luke 14:14,15. Moreover, it is expected of some, that his Prince, the Lord of that country, will shortly come into these parts, and will know the reason, if they can give any, why his neighbors set so little by him, and had him so much in derision, when they perceived that he would be a pilgrim. Jude, 14,15.

For they say, that now he is so in the affections of his Prince, that his Sovereign is so much concerned with the indignities that were cast upon Christian when he became a pilgrim, that he will look upon all as if done unto himself, Luke 10:16; and no marvel, for it was for the love that he had to his Prince that he ventured as he did.

I dare say, quoth I; I am glad on't; I am glad for the poor man's sake, for that now he has rest from his labor, and for that he now reapeth the benefit of his tears with joy; and for that he has got beyond the gun-shot of his enemies, and is out of the reach of them that hate him. Rev. 14:13; Psa. 126:5,6. I also am glad for that a rumor of these things is noised abroad in this country; who can tell but that it may work some good effect on some that are left behind? But pray, sir, while it is fresh in my mind, do you hear anything of his wife and children? Poor hearts! I wonder in my mind what they do.

Mr. Sagacity: Who? Christiana and her sons? They are like to do as well as Christian did himself; for though they all played the fool at first, and would by no means be persuaded by either the tears or entreaties of Christian, yet second thoughts have wrought wonderfully with them: so they have packed up, and are also gone after him.

Better and better, quoth I: but, what! wife and children, and all?

Mr. Sagacity: It is true: I can give you an account of the matter, for I was upon the spot at the instant, and was thoroughly acquainted with the whole affair.

Then, said I, a man, it seems, may report it for a truth.

Mr. Sagacity: You need not fear to affirm it: I mean, that they are all gone on pilgrimage, both the good woman and her four boys. And being we are, as I perceive, going some considerable way together, I will give you an account of the whole matter.

This Christiana, (for that was her name from the day that she with her children betook themselves to a pilgrim's life,) after her husband was gone over the river, and she could hear of him no more, her thoughts began to work in her mind. First, for that she had lost her husband, and for that the loving bond of that relation was utterly broken betwixt them. For you know, said he to me, nature can do no less but entertain the living with many a heavy cogitation, in the remembrance of the loss of loving relations. This, therefore, of her husband did cost her many a tear. But this was not all; for Christiana did also begin to consider with herself, whether her unbecoming behavior towards her husband was not one cause that she saw him no more, and that in such sort he was taken away from her. And upon this came into her mind, by swarms, all her unkind, unnatural, and ungodly carriage to her dear friend; which also clogged her conscience, and did load her with guilt. She was, moreover, much broken with recalling to remembrance the restless groans, brinish tears, and self-bemoanings of her husband, and how she did harden her heart against all his entreaties and loving persuasions of her and her sons to go with him; yea, there was not any thing that Christian either said to her, or did before her, all the while that his burden did hang on his back, but it returned upon her like a flash of lightning, and rent the caul of her heart in sunder; especially that bitter outcry of his, ``What shall I do to be saved?" did ring in her ears most dolefully.

Then said she to her children, Sons, we are all undone. I have sinned away your father, and he is gone: he would have had us with him, but I would not go myself: I also have hindered you of life. With that the boys fell into tears, and cried out to go after their father. Oh, said Christiana, that it had been but our lot to go with him! then had it fared well with us, beyond what it is like to do now. For, though I formerly foolishly imagined, concerning the troubles of your father, that they proceeded of a foolish fancy that he had, or for that he was overrun with melancholy humors; yet now it will not out of my mind, but that they sprang from another cause; to wit, for that the light of life was given him, James 1:23-25; John 8:12; by the help of which, as I perceive, he has escaped the snares of death. Prov. 14:27. Then they all wept again, and cried out, Oh, woe worth the day!

The next night Christiana had a dream; and, behold, she saw as if a broad parchment was opened before her, in which were recorded the sum of her ways; and the crimes, as she thought looked very black upon her. Then she cried out aloud in her sleep, ``Lord, have mercy upon me a sinner!" Luke 18:13; and the little children heard her.

After this she thought she saw two very ill-favored ones standing by her bedside, and saying, What shall we do with this woman? for she cries out for mercy, waking and sleeping: if she be suffered to go on as she begins, we shall lose her as we have lost her husband. Wherefore we must, by one way or other, seek to take her off from the thoughts of what shall be hereafter, else all the world cannot help but she will become a pilgrim.

Now she awoke in a great sweat, also a trembling was upon her: but after a while she fell to sleeping again. And then she thought she saw Christian, her husband, in a place of bliss among many immortals, with a harp in his hand, standing and playing upon it before One that sat on a throne with a rainbow about his head. She saw also, as if he bowed his head with his face to the paved work that was under his Prince's feet, saying, ``I heartily thank my Lord and King for bringing me into this place." Then shouted a company of them that stood round about, and harped with their harps; but no man living could tell what they said but Christian and his companions.

Next morning, when she was up, had prayed to God, and talked with her children a while, one knocked hard at the door; to whom she spake out, saying, ``If thou comest in God's name, come in." So he said, ``Amen;" and opened the door, and saluted her with, ``Peace be to this house." The which when he had done, he said, ``Christiana, knowest thou wherefore I am come?" Then she blushed and trembled; also her heart began to wax warm with desires to know from whence he came, and what was his errand to her. So he said unto her, ``My name is Secret; I dwell with those that are on high. It is talked of where I dwell as if thou hadst a desire to go thither: also there is a report that thou art aware of the evil thou hast formerly done to thy husband, in hardening of thy heart against his way, and in keeping of these babes in their ignorance. Christiana, the Merciful One has sent me to tell thee, that he is a God ready to forgive, and that he taketh delight to multiply the pardon of offences. He also would have thee to know, that he inviteth thee to come into his presence, to his table, and that he will feed thee with the fat of his house, and with the heritage of Jacob thy father.

``There is Christian, thy husband that was, with legions more, his companions, ever beholding that face that doth minister life to beholders; and they will all be glad when they shall hear the sound of thy feet step over thy Father's threshold."

Christiana at this was greatly abashed in herself, and bowed her head to the ground. This visitor proceeded, and said, ``Christiana, here is also a letter for thee, which I have brought from thy husband's King." So she took it, and opened it, but it smelt after the manner of the best perfume. Song 1:3. Also it was written in letters of gold. The contents of the letter were these, That the King would have her to do as did Christian her husband; for that was the way to come to his city, and to dwell in his presence with joy for ever. At this the good woman was quite overcome; so she cried out to her visitor, Sir, will you carry me and my children with you, that we also may go and worship the King?

Then said the visitor, Christiana, the bitter is before the sweet. Thou must through troubles, as did he that went before thee, enter this Celestial City. Wherefore I advise thee to do as did Christian thy husband: go to the Wicket-gate yonder, over the plain, for that stands at the head of the way up which thou must go; and I wish thee all good speed. Also I advise that thou put this letter in thy bosom, that thou read therein to thyself and to thy children until you have got it by heart; for it is one of the songs that thou must sing while thou art in this house of thy pilgrimage, Psalm 119:54; also this thou must deliver in at the further gate.

Now I saw in my dream, that this old gentleman, as he told me the story, did himself seem to be greatly affected therewith. He moreover proceeded, and said, So Christiana called her sons together, and began thus to address herself unto them: ``My sons, I have, as you may perceive, been of late under much exercise in my soul about the death of your father: not for that I doubt at all of his happiness, for I am satisfied now that he is well. I have also been much affected with the thoughts of my own state and yours, which I verily believe is by nature miserable. My carriage also to your father in his distress is a great load to my conscience; for I hardened both mine own heart and yours against him, and refused to go with him on pilgrimage.

The thoughts of these things would now kill me outright, but that for a dream which I had last night, and but that for the encouragement which this stranger has given me this morning. Come, my children, let us pack up, and begone to the gate that leads to the Celestial country, that we may see your father, and be with him and his companions in peace, according to the laws of that land.

Then did her children burst out into tears, for joy that the heart of their mother was so inclined. So their visitor bid them farewell; and they began to prepare to set out for their journey.

But while they were thus about to be gone, two of the women that were Christiana's neighbors came up to her house, and knocked at her door. To whom she said as before, If you come in God's name, come in. At this the women were stunned; for this kind of language they used not to hear, or to perceive to drop from the lips of Christiana. Yet they came in: but behold, they found the good woman preparing to be gone from her house.

So they began, and said, Neighbor, pray what is your meaning by this?

Christiana answered, and said to the eldest of them, whose name was Mrs. Timorous, I am preparing for a journey.

This Timorous was daughter to him that met Christian upon the Hill of Difficulty, and would have had him go back for fear of the lions.

Timorous: For what journey, I pray you?

Christiana: Even to go after my good husband. And with that she fell a weeping.

Timorous: I hope not so, good neighbor; pray, for your poor children's sake, do not so unwomanly cast away yourself.

Christiana: Nay, my children shall go with me; not one of them is willing to stay behind.

Timorous: I wonder in my very heart what or who has brought you into this mind!

Christiana: O neighbor, knew you but as much as I do, I doubt not but that you would go along with me.

Timorous: Prithee, what new knowledge hast thou got, that so worketh off thy mind from thy friends, and that tempteth thee to go nobody knows where?

Christiana: Then Christiana replied, I have been sorely afflicted since my husband's departure from me; but especially since he went over the river. But that which troubleth me most is, my churlish carriage to him when he was under his distress. Besides, I am now as he was then; nothing will serve me but going on pilgrimage. I was a dreaming last night that I saw him. O that my soul was with him! He dwelleth in the presence of the King of the country; he sits and eats with him at his table; he is become a companion of immortals, and has a house now given him to dwell in, to which the best palace on earth, if compared, seems to me but a dunghill. 2 Cor. 5:1-4. The Prince of the place has also sent for me, with promise of entertainment, if I shall come to him; his messenger was here even now, and has brought me a letter, which invites me to come. And with that she plucked out her letter, and read it, and said to them, What now will you say to this?

Timorous: Oh, the madness that has possessed thee and thy husband, to run yourselves upon such difficulties! You have heard, I am sure what your husband did meet with, even in a manner at the first step that he took on his way, as our neighbor Obstinate can yet testify, for he went along with him; yea, and Pliable too, until they, like wise men, were afraid to go any further. We also heard, over and above, how he met with the lions, Apollyon, the Shadow of Death, and many other things. Nor is the danger that he met with at Vanity Fair to be forgotten by thee. For if he, though a man, was so hard put to it, what canst thou, being but a poor woman, do? Consider also, that these four sweet babes are thy children, thy flesh and thy bones. Wherefore, though thou shouldest be so rash as to cast away thyself, yet, for the sake of the fruit of thy body, keep thou at home.

But Christiana said unto her, Tempt me not, my neighbor: I have now a price put into my hands to get gain, and I should be a fool of the greatest size if I should have no heart to strike in with the opportunity. And for that you tell me of all these troubles which I am like to meet with in the way, they are so far from being to me a discouragement, that they show I am in the right. The bitter must come before the sweet, and that also will make the sweet the sweeter. Wherefore, since you came not to my house in God's name, as I said, I pray you to be gone, and not to disquiet me further.

Then Timorous reviled her, and said to her fellow, Come, neighbor Mercy, let us leave her in her own hands, since she scorns our counsel and company. But Mercy was at a stand, and could not so readily comply with her neighbor; and that for a two fold reason. 1. Her bowels yearned over Christiana. So she said within herself, if my neighbor will needs be gone, I will go a little way with her, and help her. 2. Her bowels yearned over her own soul; for what Christiana had said had taken some hold upon her mind. Wherefore she said within herself again, I will yet have more talk with this Christiana; and, if I find truth and life in what she shall say, I myself with my heart shall also go with her. Wherefore Mercy began thus to reply to her neighbor Timorous:

Mercy: Neighbor, I did indeed come with you to see Christiana this morning; and since she is, as you see, taking of her last farewell of the country, I think to walk this sunshiny morning a little with her, to help her on her way. But she told her not of her second reason, but kept it to herself.

Timorous: Well, I see you have a mind to go a fooling too; but take heed in time, and be wise: while we are out of danger, we are out; but when we are in, we are in.

So Mrs. Timorous returned to her house, and Christiana betook herself to her journey. But when Timorous was got home to her house she sends for some of her neighbors, to wit, Mrs. Bat's-Eyes, Mrs. Inconsiderate, Mrs. Light-Mind, and Mrs. Know-Nothing. So when they were come to her house, she falls to telling of the story of Christiana, and of her intended journey. And thus she began her tale:

Timorous: Neighbors, having had little to do this morning, I went to give Christiana a visit; and when I came at the door I knocked, as you know it is our custom; and she answered, If you come in God's name, come in. So in I went, thinking all was well; but, when I came in I found her preparing herself to depart the town, she, and also her children. So I asked her what was her meaning by that. And she told me, in short, that she was now of a mind to go on pilgrimage, as did her husband. She told me also of a dream that she had, and how the King of the country where her husband was, had sent an inviting letter to come thither.

Then said Mrs. Know-Nothing, And what, do you think she will go?

Timorous: Aye, go she will, whatever comes on't; and methinks I know it by this; for that which was my great argument to persuade her to stay at home, (to wit, the troubles she was like to meet with on the way,) is one great argument with her to put her forward on her journey. For she told me in so many words, The bitter goes before the sweet; yea, and forasmuch as it doth, it makes the sweet the sweeter.

Mrs. Bat's-Eyes: Oh, this blind and foolish woman! said she; and will she not take warning by her husband's afflictions? For my part, I see, if he were here again, he would rest himself content in a whole skin, and never run so many hazards for nothing.

Mrs. Inconsiderate also replied, saying, Away with such fantastical fools from the town: a good riddance, for my part, I say, of her; should she stay where she dwells, and retain this her mind, who could live quietly by her? for she will either be dumpish, or unneighborly, or talk of such matters as no wise body can abide. Wherefore, for my part, I shall never be sorry for her departure; let her go, and let better come in her room: it was never a good world since these whimsical fools dwelt in it.

Then Mrs. Light-Mind added as followeth: Come, put this kind of talk away. I was yesterday at Madam Wanton's, where we were as merry as the maids. For who do you think should be there but I and Mrs. Love-the-Flesh, and three or four more, with Mrs. Lechery, Mrs. Filth, and some others: so there we had music and dancing, and what else was meet to fill up the pleasure. And I dare say, my lady herself is an admirable well-bred gentlewoman, and Mr. Lechery is as pretty a fellow. 
