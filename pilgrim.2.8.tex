\chapter[THE EIGHTH STAGE]{}

When they were gone from the shepherds, they quickly came to the place where Christian met with one Turn-away that dwelt in the town of Apostasy. Wherefore of him Mr. Great-Heart their guide now put them in mind, saying, This is the place where Christian met with one Turn-away, who carried with him the character of his rebellion at his back. And this I have to say concerning this man; he would hearken to no counsel, but once a falling, persuasion could not stop him. When he came to the place where the cross and sepulchre were, he did meet with one that did bid him look there; but he gnashed with his teeth, and stamped, and said he was resolved to go back to his own town. Before he came to the gate, he met with Evangelist, who offered to lay hands on him, to turn him into the way again; but this Turn-away resisted him, and having done much despite unto him, he got away over the wall, and so escaped his hand.

Then they went on; and just at the place where Little-Faith formerly was robbed, there stood a man with his sword drawn, and his face all over with blood. Then said Mr. Great-Heart, Who art thou? The man made answer, saying, I am one whose name is Valiant-for-truth. I am a pilgrim, and am going to the Celestial City. Now, as I was in my way, there were three men that did beset me, and propounded unto me these three things: 1. Whether I would become one of them. 2. Or go back from whence I came. 3. Or die upon the place. Prov. 1:11-14. To the first I answered, I had been a true man for a long season, and therefore it could not be expected that I should now cast in my lot with thieves. Then they demanded what I would say to the second. So I told them that the place from whence I came, had I not found incommodity there, I had not forsaken it at all; but finding it altogether unsuitable to me, and very unprofitable for me, I forsook it for this way. Then they asked me what I said to the third. And I told them my life cost far more dear than that I should lightly give it away. Besides, you have nothing to do thus to put things to my choice; wherefore at your peril be it if you meddle. Then these three, to wit, Wild-head, Inconsiderate, and Pragmatic, drew upon me, and I also drew upon them. So we fell to it, one against three, for the space of above three hours. They have left upon me, as you see, some of the marks of their valor, and have also carried away with them some of mine. They are but just now gone; I suppose they might, as the saying is, hear your horse dash, and so they betook themselves to flight.

Mr. Great-Heart: But here was great odds, three against one .

Valiant-for-Truth: Tis true; but little and more are nothing to him that has the truth on his side: ``Though an host should encamp against me," said one, Psa. 27:3, ``my heart shall not fear: though war should rise against me, in this will I be confident," etc. Besides, said he, I have read in some records, that one man has fought an army: and how many did Samson slay with the jawbone of an ass!

Mr. Great-Heart: Then said the guide, Why did you not cry out, that some might have come in for your succor?

Valiant-for-Truth: So I did to my King, who I knew could hear me, and afford invisible help, and that was sufficient for me.

Mr. Great-Heart: Then said Great-Heart to Mr. Valiant-for-truth, Thou hast worthily behaved thyself; let me see thy sword. So he showed it him.

When he had taken it in his hand, and looked thereon awhile, he said, Ha, it is a right Jerusalem blade.

Valiant-for-Truth: It is so. Let a man have one of these blades, with a hand to wield it, and skill to use it, and he may venture upon an angel with it. He need not fear its holding, if he can but tell how to lay on. Its edge will never blunt. It will cut flesh and bones, and soul, and spirit, and all. Heb. 4:12.

Mr. Great-Heart: But you fought a great while; I wonder you was not weary.

Valiant-for-Truth: I fought till my sword did cleave to my hand; and then they were joined together as if a sword grew out of my arm; and when the blood ran through my fingers, then I fought with most courage.

Mr. Great-Heart: Thou hast done well; thou hast resisted unto blood, striving against sin. Thou shalt abide by us, come in and go out with us; for we are thy companions. Then they took him and washed his wounds, and gave him of what they had, to refresh him: and so they went together.

Now, as they went on, because Mr. Great-Heart was delighted in him, (for he loved one greatly that he found to be a man of his hands,) and because there were in company those that were feeble and weak, therefore he questioned with him about many things; as first, what countryman he was.

Valiant-for-Truth: I am of Dark-land; for there was I born, and there my father and mother are still.

Mr. Great-Heart: Dark-land! said the guide; doth not that lie on the same coast with the City of Destruction?

Valiant-for-Truth: Yes, it doth. Now that which caused me to come on pilgrimage was this. We had one Mr. Tell-true come into our parts, and he told it about what Christian had done, that went from the City of Destruction; namely, how he had forsaken his wife and children, and had betaken himself to a pilgrim's life. It was also confidently reported, how he had killed a serpent that did come out to resist him in his journey; and how he got through to whither he intended. It was also told what welcome he had at all his Lord's lodgings, especially when he came to the gates of the Celestial City; for there, said the man, he was received with sound of trumpet by a company of shining ones. He told also how all the bells in the city did ring for joy at his reception, and what golden garments he was clothed with; with many other things that now I shall forbear to relate. In a word, that man so told the story of Christian and his travels that my heart fell into a burning haste to be gone after him; nor could father or mother stay me. So I got from them, and am come thus far on my way.

Mr. Great-Heart: You came in at the gate, did you not?

Valiant-for-Truth: Yes, yes; for the same man also told us, that all would be nothing if we did not begin to enter this way at the gate.

Mr. Great-Heart: Look you, said the guide to Christiana, the pilgrimage of your husband, and what he has gotten thereby, is spread abroad far and near.

Valiant-for-Truth: Why, is this Christian's wife?

Mr. Great-Heart: Yes, that it is; and these also are his four sons.

Valiant-for-Truth: What, and going on pilgrimage too?

Mr. Great-Heart: Yes, verily, they are following after.

Valiant-for-Truth: It glads me at the heart. Good man, how joyful will he be when he shall see them that would not go with him, yet to enter after him in at the gates into the Celestial City.

Mr. Great-Heart: Without doubt it will be a comfort to him; for, next to the joy of seeing himself there, it will be a joy to meet there his wife and children.

Valiant-for-Truth: But now you are upon that, pray let me hear your opinion about it. Some make a question whether we shall know one another when we are there.

Mr. Great-Heart: Do you think they shall know themselves then, or that they shall rejoice to see themselves in that bliss? And if they think they shall know and do this, why not know others, and rejoice in their welfare also? Again, since relations are our second self, though that state will be dissolved there, yet why may it not be rationally concluded that we shall be more glad to see them there than to see they are wanting?

Valiant-for-Truth: Well, I perceive whereabouts you are as to this. Have you any more things to ask me about my beginning to come on pilgrimage?

Mr. Great-Heart: Yes; were your father and mother willing that you should become a pilgrim?

Valiant-for-Truth: O no; they used all means imaginable to persuade me to stay at home.

Mr. Great-Heart: Why, what could they say against it?

Valiant-for-Truth: They said it was an idle life; and if I myself were not inclined to sloth and laziness, I would never countenance a pilgrim's condition.

Mr. Great-Heart: And what did they say else?

Valiant-for-Truth: Why, they told me that it was a dangerous way; yea, the most dangerous way in the world, said they, is that which the pilgrims go.

Mr. Great-Heart: Did they show you wherein this way is so dangerous?

Valiant-for-Truth: Yes; and that in many particulars.

Mr. Great-Heart: Name some of them.

Valiant-for-Truth: They told me of the Slough of Despond, where Christian was well-nigh smothered. They told me, that there were archers standing ready in Beelzebub-castle to shoot them who should knock at the Wicket-gate for entrance. They told me also of the wood and dark mountains; of the hill Difficulty; of the lions; and also of the three giants, Bloody-man, Maul, and Slay-good. They said, moreover, that there was a foul fiend haunted the Valley of Humiliation; and that Christian was by him almost bereft of life. Besides, said they, you must go over the Valley of the Shadow of Death, where the hobgoblins are, where the light is darkness, where the way is full of snares, pits, traps, and gins. They told me also of Giant Despair, of Doubting Castle, and of the ruin that the pilgrims met with here. Further they said I must go over the Enchanted Ground, which was dangerous; And that after all this I should find a river, over which there was no bridge; and that that river did lie betwixt me and the Celestial country.

Mr. Great-Heart: And was this all?

Valiant-for-Truth: No. They also told me that this way was full of deceivers, and of persons that lay in wait there to turn good men out of the path.

Mr. Great-Heart: But how did they make that out?

Valiant-for-Truth: They told me that Mr. Worldly Wiseman did lie there in wait to deceive. They said also, that there were Formality and Hypocrisy continually on the road. They said also, that By-ends, Talkative, or Demas, would go near to gather me up; that the Flatterer would catch me in his net; or that, with green-headed Ignorance, I would presume to go on to the gate, from whence he was sent back to the hole that was in the side of the hill, and made to go the by-way to hell.

Mr. Great-Heart: I promise you this was enough to discourage you; but did they make an end here?

Valiant-for-Truth: No, stay. They told me also of many that had tried that way of old, and that had gone a great way therein, to see if they could find something of the glory there that so many had so much talked of from time to time, and how they came back again, and befooled themselves for setting a foot out of doors in that path, to the satisfaction of all the country. And they named several that did so, as Obstinate and Pliable, Mistrust and Timorous, Turn-away and old Atheist, with several more; who, they said, had some of them gone far to see what they could find, but not one of them had found so much advantage by going as amounted to the weight of a feather.

Mr. Great-Heart: Said they any thing more to discourage you?

Valiant-for-Truth: Yes. They told me of one Mr. Fearing, who was a pilgrim, and how he found his way so solitary that he never had a comfortable hour therein; also, that Mr. Despondency had like to have been starved therein: yea, and also (which I had almost forgot) that Christian himself, about whom there has been such a noise, after all his adventures for a celestial crown, was certainly drowned in the Black River, and never went a foot further; however it was smothered up.

Mr. Great-Heart: And did none of these things discourage you?

Valiant-for-Truth: No; they seemed but as so many nothings to me.

Mr. Great-Heart: How came that about?

Valiant-for-Truth: Why, I still believed what Mr. Tell-true had said; and that carried me beyond them all.

Mr. Great-Heart: Then this was your victory, even your faith.

Valiant-for-Truth: It was so. I believed, and therefore came out, got into the way, fought all that set themselves against me, and, by believing, am come to this place.
\begin{verse}
 ``Who would true valor see,\\
Let him come hither;\\
One here will constant be,\\
Come wind, come weather\\
There's no discouragement\\
Shall make him once relent\\
His first avow'd intent\\
To be a pilgrim.\\
\end{verse}
\begin{verse}
Whoso beset him round\\
With dismal stories,\\
Do but themselves confound;\\
His strength the more is.\\
No lion can him fright,\\
He'll with a giant fight,\\
But he will have a right\\
To be a pilgrim.\\
\end{verse}
\begin{verse}
 Hobgoblin nor foul fiend\\
Can daunt his spirit;\\
He knows he at the end\\
Shall life inherit.\\
Then fancies fly away,\\
He'll not fear what men say;\\
He'll labor night and day\\
To be a pilgrim.\\
\end{verse}

By this time they were got to the Enchanted Ground, where the air naturally tended to make one drowsy. And that place was all grown over with briars and thorns, excepting here and there, where was an enchanted arbor, upon which if a man sits, or in which if a man sleeps, it is a question, some say, whether ever he shall rise or wake again in this world. Over this forest, therefore, they went, both one and another, and Mr. Great-Heart went before, for that he was the guide; and Mr. Valiant-for-truth came behind, being rear-guard, for fear lest peradventure some fiend, or dragon, or giant, or thief, should fall upon their rear, and so do mischief. They went on here, each man with his sword drawn in his hand; for they knew it was a dangerous place. Also they cheered up one another as well as they could. Feeble-mind, Mr. Great-Heart commanded should come up after him; and Mr. Despondency was under the eye of Mr. Valiant.

Now they had not gone far, but a great mist and darkness fell upon them all; so that they could scarce, for a great while, the one see the other. Wherefore they were forced, for some time, to feel one for another by words; for they walked not by sight. But any one must think, that here was but sorry going for the best of them all; but how much worse for the women and children, who both of feet and heart were but tender! Yet so it was, that through the encouraging words of him that led in the front, and of him that brought them up behind, they made a pretty good shift to wag along.

The way also here was very wearisome, through dirt and slabbiness. Nor was there, on all this ground, so much as one inn or victualling-house wherein to refresh the feebler sort. Here, therefore, was grunting, and puffing, and sighing, while one tumbleth over a bush, another sticks fast in the dirt, and the children, some of them, lost their shoes in the mire; while one cries out, I am down; and another, Ho, where are you? and a third, The bushes have got such fast hold on me, I think I cannot get away from them.

Then they came at an arbor, warm, and promising much refreshing to the pilgrims; for it was finely wrought above-head, beautified with greens, furnished with benches and settles. It also had in it a soft couch, whereon the weary might lean. This, you must think, all things considered, was tempting; for the pilgrims already began to be foiled with the badness of the way: but there was not one of them that made so much as a motion to stop there. Yea, for aught I could perceive, they continually gave so good heed to the advice of their guide, and he did so faithfully tell them of dangers, and of the nature of the dangers when they were at them, that usually, when they were nearest to them, they did most pluck up their spirits, and hearten one another to deny the flesh. This arbor was called The Slothful's Friend, and was made on purpose to allure, if it might be, some of the pilgrims there to take up their rest when weary.

I saw them in my dream, that they went on in this their solitary ground, till they came to a place at which a man is apt to lose his way. Now, though when it was light their guide could well enough tell how to miss those ways that led wrong, yet in the dark he was put to a stand. But he had in his pocket a map of all ways leading to or from the Celestial City; wherefore he struck a light (for he never goes without his tinder-box also), and takes a view of his book or map, which bids him to be careful in that place to turn to the right hand. And had he not been careful here to look in his map, they had all, in probability, been smothered in the mud; for just a little before them, and that at the end of the cleanest way too, was a pit, none knows how deep, full of nothing but mud, there made on purpose to destroy the pilgrims in.

Then thought I with myself, Who that goeth on pilgrimage but would have one of these maps about him, that he may look, when he is at a stand, which is the way he must take?

Then they went on in this Enchanted Ground till they came to where there was another arbor, and it was built by the highway-side. And in that arbor there lay two men, whose names were Heedless and Too-bold. These two went thus far on pilgrimage; but here, being wearied with their journey, they sat down to rest themselves, and so fell fast asleep. When the pilgrims saw them, they stood still, and shook their heads; for they knew that the sleepers were in a pitiful case. Then they consulted what to do, whether to go on and leave them in their sleep, or to step to them and try to awake them; so they concluded to go to them and awake them, that is, if they could; but with this caution, namely, to take heed that they themselves did not sit down nor embrace the offered benefit of that arbor.

So they went in, and spake to the men, and called each by his name, for the guide, it seems, did know them; but there was no voice nor answer. Then the guide did shake them, and do what he could to disturb them. Then said one of them, I will pay you when I take my money. At which the guide shook his head. I will fight so long as I can hold my sword in my hand, said the other. At that, one of the children laughed.

Then said Christiana, What is the meaning of this? The guide said, They talk in their sleep. If you strike them, beat them, or whatever else you do to them, they will answer you after this fashion; or, as one of them said in old time, when the waves of the sea did beat upon him, and he slept as one upon the mast of a ship, Prov. 23:34,35, When I awake, I will seek it yet again. You know, when men talk in their sleep, they say any thing; but their words are not governed either by faith or reason. There is an incoherency in their words now, as there was before betwixt their going on pilgrimage and sitting down here. This, then, is the mischief of it: when heedless ones go on pilgrimage, tis twenty to one but they are served thus. For this Enchanted Ground is one of the last refuges that the enemy to pilgrims has; wherefore it is, as you see, placed almost at the end of the way, and so it standeth against us with the more advantage. For when, thinks the enemy, will these fools be so desirous to sit down as when they are weary? and when so like to be weary as when almost at their journey's end? Therefore it is, I say, that the Enchanted Ground is placed so nigh to the land Beulah, and so near the end of their race. Wherefore let pilgrims look to themselves, lest it happen to them as it has done to these that, as you see, are fallen asleep, and none can awake them.

Then the pilgrims desired with trembling to go forward; only they prayed their guide to strike a light, that they might go the rest of their way by the help of the light of a lantern. So he struck a light, and they went by the help of that through the rest of this way, though the darkness was very great. 2 Pet. 1:19. But the children began to be sorely weary, and they cried out unto him that loveth pilgrims, to make their way more comfortable. So by that they had gone a little further, a wind arose that drove away the fog, so the air became more clear. Yet they were not off (by much) of the Enchanted Ground; only now they could see one another better, and the way wherein they should walk.

Now when they were almost at the end of this ground, they perceived that a little before them was a solemn noise, as of one that was much concerned. So they went on and looked before them: and behold they saw, as they thought, a man upon his knees, with hands and eyes lifted up, and speaking, as they thought, earnestly to one that was above. They drew nigh, but could not tell what he said; so they went softly till he had done. When he had done, he got up, and began to run towards the Celestial City. Then Mr. Great-Heart called after him, saying, Soho, friend, let us have your company, if you go, as I suppose you do, to the Celestial City. So the man stopped, and they came up to him. But as soon as Mr. Honest saw him, he said, I know this man. Then said Mr. Valiant-for-truth, Prithee, who is it? It is one, said he, that comes from whereabout I dwelt. His name is Standfast; he is certainly a right good pilgrim.

So they came up to one another; and presently Standfast said to old Honest, Ho, father Honest, are you there? Aye, said he, that I am, as sure as you are there. Right glad am I, said Mr. Standfast, that I have found you on this road. And as glad am I, said the other, that I espied you on your knees. Then Mr. Standfast blushed, and said, But why, did you see me? Yes, that I did, quoth the other, and with my heart was glad at the sight. Why, what did you think? said Standfast. Think! said old Honest; what could I think? I thought we had an honest man upon the road, and therefore should have his company by and by. If you thought not amiss, said Standfast, how happy am I! But if I be not as I should, t is I alone must bear it. That is true, said the other; but your fear doth further confirm me that things are right betwixt the Prince of pilgrims and your soul. For he saith, ``Blessed is the man that feareth always." Prov. 28:14.

Valiant-for-Truth: Well but, brother, I pray thee tell us what was it that was the cause of thy being upon thy knees even now: was it for that some special mercy laid obligations upon thee, or how?

Standfast: Why, we are, as you see, upon the Enchanted Ground; and as I was coming along, I was musing with myself of what a dangerous nature the road in this place was, and how many that had come even thus far on pilgrimage, had here been stopped and been destroyed. I thought also of the manner of the death with which this place destroyeth men. Those that die here, die of no violent distemper: the death which such die is not grievous to them. For he that goeth away in a sleep, begins that journey with desire and pleasure. Yea, such acquiesce in the will of that disease.

Mr. Honest: Then Mr. Honest interrupting him, said, Did you see the two men asleep in the arbor?

Standfast: Aye, aye, I saw Heedless and Too-bold there; and for ought I know, there they will lie till they rot. Prov. 10:7. But let me go on with my tale. As I was thus musing, as I said, there was one in very pleasant attire, but old, who presented herself to me, and offered me three things, to wit, her body, her purse, and her bed. Now the truth is, I was both weary and sleepy. I am also as poor as an owlet, and that perhaps the witch knew. Well, I repulsed her once and again, but she put by my repulses, and smiled. Then I began to be angry; but she mattered that nothing at all. Then she made offers again, and said, if I would be ruled by her, she would make me great and happy; for, said she, I am the mistress of the world, and men are made happy by me. Then I asked her name, and she told me it was Madam Bubble. This set me further from her; but she still followed me with enticements. Then I betook me, as you saw, to my knees, and with hands lifted up, and cries, I prayed to Him that had said he would help. So, just as you came up, the gentlewoman went her way. Then I continued to give thanks for this my

great deliverance; for I verily believe she intended no good, but rather sought to make stop of me in my journey.

Mr. Honest: Without doubt her designs were bad. But stay, now you talk of her, methinks I either have seen her, or have read some story of her.

Standfast: Perhaps you have done both.

Mr. Honest: Madam Bubble! Is she not a tall, comely dame, something of a swarthy complexion?

Standfast: Right, you hit it: she is just such a one.

Mr. Honest: Doth she not speak very smoothly, and give you a smile at the end of a sentence?

Standfast: You fall right upon it again, for these are her very actions.

Mr. Honest: Doth she not wear a great purse by her side, and is not her hand often in it, fingering her money, as if that was her heart's delight.

Standfast: Tis just so; had she stood by all this while, you could not more amply have set her forth before me, nor have better described her features.

Mr. Honest: Then he that drew her picture was a good limner, and he that wrote of her said true.

Mr. Great-Heart: This woman is a witch, and it is by virtue of her sorceries that this ground is enchanted. Whoever doth lay his head down in her lap, had as good lay it down on that block over which the axe doth hang; and whoever lay their eyes upon her beauty are counted the enemies of God. This is she that maintaineth in their splendor all those that are the enemies of pilgrims. James 4:4. Yea, this is she that has bought off many a man from a pilgrim's life. She is a great gossiper; she is always, both she and her daughters, at one pilgrim's heels or another, now commending, and then preferring the excellences of this life. She is a bold and impudent slut: she will talk with any man. She always laugheth poor pilgrims to scorn, but highly commends the rich. If there be one cunning to get money in a place, she will speak well of him from house to house. She loveth banqueting and feasting mainly well; she is always at one full table or another. She has given it out in some places that she is a goddess, and therefore some do worship her. She has her time, and open places of cheating; and she will say and avow it, that none can show a good comparable to hers. She promiseth to dwell with children's children, if they will but love her and make much of her. She will cast out of her purse gold like dust in some places and to some persons. She loves to be sought after, spoken well of, and to lie in the bosoms of men. She is never weary of commending her commodities, and she loves them most that think best of her. She will promise to some crowns and kingdoms, if they will but take her advice; yet many has she brought to the halter, and ten thousand times more to hell.

Standfast: Oh, said Standfast, what a mercy is it that I did resist her; for whither might she have drawn me!

Mr. Great-Heart: Whither? nay, none but God knows whither. But in general, to be sure, she would have drawn thee into many foolish and hurtful lusts, which drown men in destruction and perdition. 1 Tim. 6:9. T was she that set Absalom against his father, and Jeroboam against his master. T was she that persuaded Judas to sell his Lord; and that prevailed with Demas to forsake the godly pilgrim's life. None can tell of the mischief that she doth. She makes variance betwixt rulers and subjects, betwixt parents and children, betwixt neighbor and neighbor, betwixt a man and his wife, betwixt a man and himself, betwixt the flesh and the spirit. Wherefore, good Mr. Standfast, be as your name is, and when you have done all, stand.

At this discourse there was among the pilgrims a mixture of joy and trembling; but at length they broke out and sang,
\begin{verse}
 ``What danger is the Pilgrim in!\\
How many are his foes!\\
How many ways there are to sin\\
No living mortal knows.\\
\end{verse}
\begin{verse} 
Some in the ditch are spoiled, yea, can\\
Lie tumbling in the mire:\\
Some, though they shun the frying-pan\\
Do leap into the fire."\\
\end{verse}

After this, I beheld until they were come into the land of Beulah, where the sun shineth night and day. Here, because they were weary, they betook themselves a while to rest. And because this country was common for pilgrims, and because the orchards and vineyards that were here belonged to the King of the Celestial country, therefore they were licensed to make bold with any of his things. But a little while soon refreshed them here; for the bells did so ring, and the trumpets continually sound so melodiously, that they could not sleep, and yet they received as much refreshing as if they had slept their sleep ever so soundly. Here also all the noise of them that walked the streets was, More pilgrims are come to town! And another would answer, saying, And so many went over the water, and were let in at the golden gates to-day! They would cry again, There is now a legion of shining ones just come to town, by which we know that there are more pilgrims upon the road; for here they come to wait for them, and to comfort them after all their sorrow. Then the pilgrims got up, and walked to and fro. But how were their ears now filled with heavenly noises, and their eyes delighted with celestial visions! In this land they heard nothing, saw nothing, felt nothing, smelt nothing, tasted nothing that was offensive to their stomach or mind; only when they tasted of the water of the river over which they were to go, they thought that it tasted a little bitterish to the palate; but it proved sweeter when it was down.

In this place there was a record kept of the names of them that had been pilgrims of old, and a history of all the famous acts that they had done. It was here also much discoursed, how the river to some had had its flowings, and what ebbings it has had while others have gone over. It has been in a manner dry for some, while it has overflowed its banks for others.

In this place the children of the town would go into the King's gardens, and gather nosegays for the pilgrims, and bring them to them with much affection. Here also grew camphire, with spikenard and saffron, calamus and cinnamon, with all the trees of frankincense, myrrh, and aloes, with all chief spices. With these the pilgrims' chambers were perfumed while they stayed here; and with these were their bodies anointed, to prepare them to go over the river, when the time appointed was come.

Now, while they lay here, and waited for the good hour, there was a noise in the town that there was a post come from the Celestial City, with matter of great importance to one Christiana, the wife of Christian the pilgrim. So inquiry was made for her, and the house was found out where she was. So the post presented her with a letter. The contents were, Hail, good woman; I bring thee tidings that the Master calleth for thee, and expecteth that thou shouldst stand in his presence in clothes of immortality within these ten days.

When he had read this letter to her, he gave her therewith a sure token that he was a true messenger, and was come to bid her make haste to be gone. The token was, an arrow with a point sharpened with love, let easily into her heart, which by degrees wrought so effectually with her, that at the time appointed she must be gone.

When Christiana saw that her time was come, and that she was the first of this company that was to go over, she called for Mr. Great-Heart her guide, and told him how matters were. So he told her he was heartily glad of the news, and could have been glad had the post come for him. Then she bid him that he should give advice how all things should be prepared for her journey. So he told her, saying, Thus and thus it must be, and we that survive will accompany you to the river-side.

Then she called for her children, and gave them her blessing, and told them that she had read with comfort the mark that was set in their foreheads, and was glad to see them with her there, and that they had kept their garments so white. Lastly, she bequeathed to the poor that little she had, and commanded her sons and daughters to be ready against the messenger should come for them.

When she had spoken these words to her guide, and to her children, she called for Mr. Valiant-for-truth, and said unto him, Sir, you have in all places showed yourself true-hearted; be faithful unto death, and my King will give you a crown of life. Rev. 2:10. I would also entreat you to have an eye to my children; and if at any time you see them faint, speak comfortably to them. For my daughters, my sons' wives, they have been faithful, and a fulfilling of the promise upon them will be their end. But she gave Mr. Standfast a ring.

Then she called for old Mr. Honest, and said of him, ``Behold an Israelite indeed, in whom is no guile!" John 1:47. Then said he, I wish you a fair day when you set out for Mount Sion, and shall be glad to see that you go over the river dry-shod. But she answered, Come wet, come dry, I long to be gone; for however the weather is in my journey, I shall have time enough when I come there to sit down and rest me and dry me.

Then came in that good man Mr. Ready-to-halt, to see her. So she said to him, Thy travel hitherto has been with difficulty; but that will make thy rest the sweeter. Watch, and be ready; for at an hour when you think not, the messenger may come.

After him came Mr. Despondency and his daughter Much-afraid, to whom she said, You ought, with thankfulness, forever to remember your deliverance from the hands of Giant Despair, and out of Doubting Castle. The effect of that mercy is, that you are brought with safety hither. Be ye watchful, and cast away fear; be sober, and hope to the end.

Then she said to Mr. Feeble-mind, Thou wast delivered from the mouth of Giant Slay-good, that thou mightest live in the light of the living, and see thy King with comfort. Only I advise thee to repent of thine aptness to fear and doubt of his goodness, before he sends for thee; lest thou shouldst, when he comes, be forced to stand before him for that fault with blushing.

Now the day drew on that Christiana must be gone. So the road was full of people to see her take her journey. But behold, all the banks beyond the river were full of horses and chariots, which were come down from above to accompany her to the city gate. So she came forth, and entered the river, with a beckon of farewell to those that followed her. The last words that she was heard to say were, I come, Lord, to be with thee and bless thee! So her children and friends returned to their place, for those that waited for Christiana had carried her out of their sight. So she went and called, and entered in at the gate with all the ceremonies of joy that her husband Christian had entered with before her. At her departure, the children wept. But Mr. Great-Heart and Mr. Valiant played upon the welltuned cymbal and harp for joy. So all departed to their respective places.

In process of time there came a post to the town again, and his business was with Mr. Ready-to-halt. So he inquired him out, and said, I am come from Him whom thou hast loved and followed, though upon crutches; and my message is to tell thee, that he expects thee at his table to sup with him in his kingdom, the next day after Easter; wherefore prepare thyself for this journey. Then he also gave him a token that he was a true messenger, saying, ``I have broken thy golden bowl, and loosed thy silver cord." Eccles. 12:6.

After this, Mr. Ready-to-halt called for his fellow-pilgrims, and told them, saying, I am sent for, and God shall surely visit you also. So he desired Mr. Valiant to make his will. And because he had nothing to bequeath to them that should survive him but his crutches, and his good wishes, therefore thus he said, These crutches I bequeath to my son that shall tread in my steps, with a hundred warm wishes that he may prove better than I have been.

Then he thanked Mr. Great-Heart for his conduct and kindness, and so addressed himself to his journey. When he came to the brink of the river, he said, Now I shall have no more need of these crutches, since yonder are chariots and horses for me to ride on. The last words he was heard to say were, Welcome life! So he went his way.

After this, Mr. Feeble-mind had tidings brought him that the post sounded his horn at his chamber door. Then he came in, and told him, saying, I am come to tell thee that thy Master hath need of thee, and that in a very little time thou must behold his face in brightness. And take this as a token of the truth of my message: ``Those that look out at the windows shall be darkened." Eccles. 12:3. Then Mr. Feeble-mind called for his friends, and told them what errand had been brought unto him, and what token he had received of the truth of the message. Then he said, since I have nothing to bequeath to any, to what purpose should I make a will? As for my feeble mind, that I will leave behind me, for that I shall have no need of it in the place whither I go, nor is it worth bestowing upon the poorest pilgrims: wherefore, when I am gone, I desire that you, Mr. Valiant, would bury it in a dunghill. This done, and the day being come on which he was to depart, he entered the river as the rest. His last words were, Hold out, faith and patience! So he went over to the other side.

When days had many of them passed away, Mr. Despondency was sent for; for a post was come, and brought this message to him: Trembling man! these are to summon thee to be ready with the King by the next Lord's day, to shout for joy for thy deliverance from all thy doubtings. And, said the messenger, that my message is true, take this for a proof: so he gave him a grasshopper to be a burden unto him. Ecclesiastes 12:5.

Now Mr. Despondency's daughter, whose name was Much-afraid, said, when she heard what was done, that she would go with her father. Then Mr. Despondency said to his friends, Myself and my daughter, you know what we have been, and how troublesomely we have behaved ourselves in every company. My will and my daughter's is, that our desponds and slavish fears be by no man ever received, from the day of our departure, forever; for I know that after my death they will offer themselves to others. For, to be plain with you, they are ghosts which we entertained when we first began to be pilgrims, and could never shake them off after; and they will walk about, and seek entertainment of the pilgrims: but for our sakes, shut the doors upon them. When the time was come for them to depart, they went up to the brink of the river. The last words of Mr. Despondency were, Farewell, night; welcome, day! His daughter went through the river singing, but none could understand what she said.

Then it came to pass a while after, that there was a post in the town that inquired for Mr. Honest. So he came to the house where he was, and delivered to his hand these lines: Thou art commanded to be ready against this day seven-night, to present thyself before thy Lord at his Father's house. And for a token that my message is true, ``All the daughters of music shall be brought low." Eccles. 12:4. Then Mr. Honest called for his friends, and said unto them, I die, but shall make no will. As for my honesty, it shall go with me; let him that comes after be told of this. When the day that he was to be gone was come, he addressed himself to go over the river. Now the river at that time over-flowed its banks in some places; but Mr. Honest, in his lifetime, had spoken to one Good-conscience to meet him there, the which he also did, and lent him his hand, and so helped him over. The last words of Mr. Honest were, Grace reigns! So he left the world.

After this it was noised abroad that Mr. Valiant-for-truth was taken with a summons by the same post as the other, and had this for a token that the summons was true, ``That his pitcher was broken at the fountain." Eccl. 12:6. When he understood it, he called for his friends, and told them of it. Then said he, I am going to my Father's; and though with great difficulty I have got hither, yet now I do not repent me of all the trouble I have been at to arrive where I am. My sword I give to him that shall succeed me in my pilgrimage, and my courage and skill to him that can get it. My marks and scars I carry with me, to be a witness for me that I have fought His battles who will now be my rewarder. When the day that he must go hence was come, many accompanied him to the river-side, into which as he went, he said, ``Death, where is thy sting?" And as he went down deeper, he said, ``Grave, where is thy victory?" 1 Cor. 15:55. So he passed over, and all the trumpets sounded for him on the other side.

Then there came forth a summons for Mr. Standfast. This Mr. Standfast was he whom the rest of the pilgrims found upon his knees in the Enchanted Ground. And the post brought it him open in his hands: the contents thereof were, that he must prepare for a change of life, for his Master was not willing that he should be so far from him any longer. At this Mr. Standfast was put into a muse. Nay, said the messenger, you need not doubt of the truth of my message; for here is a token of the truth thereof, ``Thy wheel is broken at the cistern." Eccles. 12:6. Then he called to him Mr. Great-Heart, who was their guide, and said unto him, Sir, although it was not my hap to be much in your good company during the days of my pilgrimage, yet, since the time I knew you, you have been profitable to me. When I came from home, I left behind me a wife and five small children; let me entreat you, at your return, (for I know that you go and return to your Master's house, in hopes that you may yet be a conductor to more of the holy pilgrims,) that you send to my family, and let them be acquainted with all that hath and shall happen unto me. Tell them moreover of my happy arrival at this place, and of the present and late blessed condition I am in. Tell them also of Christian and Christiana his wife, and how she and her children came after her husband. Tell them also of what a happy end she made, and whither she is gone. I have little or nothing to send to my family, unless it be prayers and tears for them; of which it will suffice that you acquaint them, if peradventure they may prevail. When Mr. Standfast had thus set things in order, and the time being come for him to haste him away, he also went down to the river. Now there was a great calm at that time in the river; wherefore Mr. Standfast, when he was about half-way in, stood a while, and talked with his companions that had waited upon him thither. And he said, This river has been a terror to many; yea, the thoughts of it also have often frightened me; but now methinks I stand easy; my foot is fixed upon that on which the feet of the priests that bare the ark of the covenant stood while Israel went over Jordan. Josh. 3:17. The waters indeed are to the palate bitter, and to the stomach cold; yet the thoughts of what I am going to, and of the convoy that waits for me on the other side, do lie as a glowing coal at my heart. I see myself now at the end of my journey; my toilsome days are ended. I am going to see that head which was crowned with thorns, and that face which was spit upon for me. I have formerly lived by hearsay and faith; but now I go where I shall live by sight, and shall be with him in whose company I delight myself. I have loved to hear my Lord spoken of; and wherever I have seen the print of his shoe in the earth, there I have coveted to set my foot too. His name has been to me as a civet-box; yea, sweeter than all perfumes. His voice to me has been most sweet, and his countenance I have more desired than they that have most desired the light of the sun. His words I did use to gather for my food, and for antidotes against my faintings. He hath held me, and hath kept me from mine iniquities; yea, my steps hath he strengthened in his way.

Now, while he was thus in discourse, his countenance changed; his strong man bowed under him: and after he had said, Take me, for I come unto thee, he ceased to be seen of them.

But glorious it was to see how the open region was filled with horses and chariots, with trumpeters and pipers, with singers and players upon stringed instruments, to welcome the pilgrims as they went up, and followed one another in at the beautiful gate of the city.

As for Christiana's children, the four boys that Christiana brought, with their wives and children, I did not stay where I was till they were gone over. Also, since I came away, I heard one say that they were yet alive, and so would be for the increase of the church, in that place where they were, for a time.

Should it be my lot to go that way again, I may give those that desire it an account of what I here am silent about: meantime I bid my reader

FAREWELL.

THE END. 
