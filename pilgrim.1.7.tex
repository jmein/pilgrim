\chapter[THE SEVENTH STAGE]{}

Now I saw in my dream, that Christian went not forth alone; for there was one whose name was Hopeful, (being so made by the beholding of Christian and Faithful in their words and behavior, in their sufferings at the fair,) who joined himself unto him, and entering into a brotherly covenant, told him that he would be his companion. Thus one died to bear testimony to the truth, and another rises out of his ashes to be a companion with Christian in his pilgrimage. This Hopeful also told Christian, that there were many more of the men in the fair that would take their time, and follow after.

So I saw, that quickly after they were got out of the fair, they overtook one that was going before them, whose name was By-ends; so they said to him, What countryman, sir? and how far go you this way? He told them, that he came from the town of Fair-speech, and he was going to the Celestial City; but told them not his name.

From Fair-speech? said Christian; is there any good that lives there? Prov. 26:25.

By-Ends: Yes, said By-ends, I hope so.

Christian: Pray, sir, what may I call you? said Christian.

By-Ends: I am a stranger to you, and you to me: if you be going this way, I shall be glad of your company; if not, I must be content.

Christian: This town of Fair-speech, said Christian, I have heard of; and, as I remember, they say it's a wealthy place.

By-Ends: Yes, I will assure you that it is; and I have very many rich kindred there.

Christian: Pray, who are your kindred there, if a man may be so bold?

By-Ends: Almost the whole town; and in particular my Lord Turn-about, my Lord Time-server, my Lord Fair-speech, from whose ancestors that town first took its name; also, Mr. Smooth-man, Mr. Facing-both-ways, Mr. Any-thing; and the parson of our parish, Mr. Two-tongues, was my mother's own brother, by father's side; and, to tell you the truth, I am become a gentleman of good quality; yet my great-grandfather was but a waterman, looking one way and rowing another, and I got most of my estate by the same occupation.

Christian: Are you a married man.

By-Ends: Yes, and my wife is a very virtuous woman, the daughter of a virtuous woman; she was my Lady Feigning's daughter; therefore she came of a very honorable family, and is arrived to such a pitch of breeding, that she knows how to carry it to all, even to prince and peasant. Tis true, we somewhat differ in religion from those of the stricter sort, yet but in two small points: First, we never strive against wind and tide. Secondly, we are always most zealous when religion goes in his silver slippers; we love much to walk with him in the street, if the sun shines and the people applaud him.

Then Christian stepped a little aside to his fellow Hopeful, saying, it runs in my mind that this is one By-ends, of Fair-speech; and if it be he, we have as very a knave in our company as dwelleth in all these parts. Then said Hopeful, Ask him; methinks he should not be ashamed of his name. So Christian came up with him again, and said, Sir, you talk as if you knew something more than all the world doth; and, if I take not my mark amiss, I deem I have half a guess of you. Is not your name Mr. By-ends of Fair-speech?

By-Ends: This is not my name, but indeed it is a nickname that is given me by some that cannot abide me, and I must be content to bear it as a reproach, as other good men have borne theirs before me.

Christian: But did you never give an occasion to men to call you by this name?

By-Ends: Never, never! The worst that ever I did to give them an occasion to give me this name was, that I had always the luck to jump in my judgment with the present way of the times, whatever it was, and my chance was to get thereby: but if things are thus cast upon me, let me count them a blessing; but let not the malicious load me therefore with reproach.

Christian: I thought, indeed, that you were the man that I heard of; and to tell you what I think, I fear this name belongs to you more properly than you are willing we should think it doth.

By-Ends: Well if you will thus imagine, I cannot help it; you shall find me a fair company-keeper, if you will still admit me your associate.

Christian: If you will go with us, you must go against wind and tide; the which, I perceive, is against your opinion: you must also own Religion in his rags, as well as when in his silver slippers; and stand by him, too, when bound in irons, as well as when he walketh the streets with applause.

By-Ends: You must not impose, nor lord it over my faith; leave me to my liberty, and let me go with you.

Christian: Not a step farther, unless you will do, in what I propound, as we.

Then said By-ends, I shall never desert my old principles, since they are harmless and profitable. If I may not go with you, I must do as I did before you overtook me, even go by myself, until some overtake me that will be glad of my company.

Now I saw in my dream, that Christian and Hopeful forsook him, and kept their distance before him; but one of them, looking back, saw three men following Mr. By-ends; and, behold, as they came up with him, he made them a very low congee; and they also gave him a compliment. The men's names were, Mr. Hold-the-world, Mr. Money-love, and Mr. Save-all, men that Mr. By-ends had formerly been acquainted with; for in their minority they were schoolfellows, and taught by one Mr. Gripeman, a schoolmaster in Lovegain, which is a market-town in the county of Coveting, in the North. This Schoolmaster taught them the art of getting, either by violence, cozenage, flattering, lying, or by putting on a guise of religion; and these four gentlemen had attained much of the art of their master, so that they could each of them have kept such a school themselves.

Well, when they had, as I said, thus saluted each other, Mr. Money-love said to Mr. By-ends, Who are they upon the road before us? For Christian and Hopeful were yet within view.

By-Ends: They are a couple of far country-men, that, after their mode, are going on pilgrimage.

Mr. Money-Love: Alas! why did they not stay, that we might have had their good company? for they, and we, and you, sir, I hope, are all going on pilgrimage.

By-Ends: We are so, indeed; but the men before us are so rigid, and love so much their own notions, and do also so lightly esteem the opinions of others, that let a man be ever so godly, yet if he jumps not with them in all things, they thrust him quite out of their company.

Mr. Save-All: That is bad; but we read of some that are righteous overmuch, and such men's rigidness prevails with them to judge and condemn all but themselves. But I pray, what, and how many, were the things wherein you differed?

By-Ends: Why, they, after their headstrong manner, conclude that it is their duty to rush on their journey all weathers, and I am for waiting for wind and tide. They are for hazarding all for God at a clap; and I am for taking all advantages to secure my life and estate. They are for holding their notions, though all other men be against them; but I am for religion in what, and so far as the times and my safety will bear it. They are for religion when in rags and contempt; but I am for him when he walks in his silver slippers, in the sunshine, and with applause.

Mr. Hold-the-World: Aye, and hold you there still, good Mr. By-ends; for, for my part, I can count him but a fool, that having the liberty to keep what he has, shall be so unwise as to lose it. Let us be wise as serpents. It is best to make hay while the sun shines. You see how the bee lieth still in winter, and bestirs her only when she can have profit with pleasure. God sends sometimes rain, and sometimes sunshine: if they be such fools to go through the first, yet let us be content to take fair weather along with us. For my part, I like that religion best that will stand with the security of God's good blessings unto us; for who can imagine, that is ruled by his reason, since God has bestowed upon us the good things of this life, but that he would have us keep them for his sake? Abraham and Solomon grew rich in religion; and Job says, that a good man shall lay up gold as dust; but he must not be such as the men before us, if they be as you have described them.

Mr. Save-All: I think that we are all agreed in this matter; and therefore there needs no more words about it.

Mr. Money-Love: No, there needs no more words about this matter, indeed; for he that believes neither Scripture nor reason, (and you see we have both on our side,) neither knows his own liberty nor seeks his own safety.

By-Ends: My brethren, we are, as you see, going all on pilgrimage; and for our better diversion from things that are bad, give me leave to propound unto you this question.

Suppose a man, a minister, or a tradesman, etc., should have an advantage lie before him to get the good blessings of this life, yet so as that he can by no means come by them, except, in appearance at least, he becomes extraordinary zealous in some points of religion that he meddled not with before; may he not use this means to attain his end, and yet be a right honest man?

Mr. Money-Love: I see the bottom of your question; and with these gentlemen's good leave, I will endeavor to shape you an answer. And first, to speak to your question as it concerneth a minister himself: suppose a minister, a worthy man, possessed but of a very small benefice, and has in his eye a greater, more fat and plump by far; he has also now an opportunity of getting it, yet so as by being more studious, by preaching more frequently and zealously, and, because the temper of the people requires it, by altering of some of his principles; for my part, I see no reason why a man may not do this, provided he has a call, aye, and more a great deal besides, and yet be an honest man. For why?

1. His desire of a greater benefice is lawful, (this cannot be contradicted,) since it is set before him by Providence; so then he may get it if he can, making no question for conscience' sake.

2. Besides, his desire after that benefice makes him more studious, a more zealous preacher, etc., and so makes him a better man, yea, makes him better improve his parts, which is according to the mind of God.

3. Now, as for his complying with the temper of his people, by deserting, to serve them, some of his principles, this argueth, 1. That he is of a self-denying temper. 2. Of a sweet and winning deportment. And, 3. So more fit for the ministerial function.

4. I conclude, then, that a minister that changes a small for a great, should not, for so doing, be judged as covetous; but rather, since he is improved in his parts and industry thereby, be counted as one that pursues his call, and the opportunity put into his hand to do good.

And now to the second part of the question, which concerns the tradesman you mentioned. Suppose such an one to have but a poor employ in the world, but by becoming religious he may mend his market, perhaps get a rich wife, or more and far better customers to his shop; for my part, I see no reason but this may be lawfully done. For why?

1. To become religious is a virtue, by what means soever a man becomes so.

2. Nor is it unlawful to get a rich wife, or more custom to my shop.

3. Besides, the man that gets these by becoming religious, gets that which is good of them that are good, by becoming good himself; so then here is a good wife, and good customers, and good gain, and all these by becoming religious, which is good: therefore, to become religious to get all these is a good and profitable design.

This answer, thus made by Mr. Money-love to Mr. By-ends' question, was highly applauded by them all; wherefore they concluded, upon the whole, that it was most wholesome and advantageous. And because, as they thought, no man was able to contradict it; and because Christian and Hopeful were yet within call, they jointly agreed to assault them with the question as soon as they overtook them; and the rather, because they had opposed Mr. By-ends before. So they called after them, and they stopped and stood still till they came up to them; but they concluded, as they went, that not Mr. By-ends, but old Mr. Hold-the-world should propound the question to them, because, as they supposed, their answer to him would be without the remainder of that heat that was kindled betwixt Mr. By-ends and them at their parting a little before.

So they came up to each other, and after a short salutation, Mr. Hold-the-world propounded the question to Christian and his fellow, and then bid them to answer if they could.

Then said Christian, Even a babe in religion may answer ten thousand such questions. For if it be unlawful to follow Christ for loaves, as it is, John 6:26; how much more abominable is it to make of him and religion a stalking-horse to get and enjoy the world! Nor do we find any other than heathens, hypocrites, devils, and wizards, that are of this opinion.

1. Heathens: for when Hamor and Shechem had a mind to the daughter and cattle of Jacob, and saw that there was no way for them to come at them but by being circumcised, they said to their companions, If every male of us be circumcised, as they are circumcised, shall not their cattle, and their substance, and every beast of theirs be ours? Their daughters and their cattle were that which they sought to obtain, and their religion the stalking-horse they made use of to come at them. Read the whole story, Gen. 34:20-24.

2. The hypocritical Pharisees were also of this religion: long prayers were their pretence, but to get widows' houses was their intent; and greater damnation was from God their judgment. Luke 20:46,47.

3. Judas the devil was also of this religion: he was religious for the bag, that he might be possessed of what was put therein; but he was lost, cast away, and the very son of perdition.

4. Simon the wizard was of this religion too; for he would have had the Holy Ghost, that he might have got money therewith: and his sentence from Peter's mouth was according. Acts 8:19-22.

5. Neither will it go out of my mind, but that that man who takes up religion for the world, will throw away religion for the world; for so surely as Judas designed the world in becoming religious, so surely did he also sell religion and his Master for the same. To answer the question, therefore, affirmatively, as I perceive you have done, and to accept of, as authentic, such answer, is heathenish, hypocritical, and devilish; and your reward will be according to your works.

Then they stood staring one upon another, but had not wherewith to answer Christian. Hopeful also approved of the soundness of Christian's answer; so there was a great silence among them. Mr. By-ends and his company also staggered and kept behind, that Christian and Hopeful might outgo them. Then said Christian to his fellow, If these men cannot stand before the sentence of men, what will they do with the sentence of God? And if they are mute when dealt with by vessels of clay, what will they do when they shall be rebuked by the flames of a devouring fire?

Then Christian and Hopeful outwent them again, and went till they came at a delicate plain, called Ease, where they went with much content; but that plain was but narrow, so they were quickly got over it. Now at the farther side of that plain was a little hill, called Lucre, and in that hill a silver-mine, which some of them that had formerly gone that way, because of the rarity of it, had turned aside to see; but going too near the brim of the pit, the ground, being deceitful under them, broke, and they were slain: some also had been maimed there, and could not, to their dying day, be their own men again.

Then I saw in my dream, that a little off the road, over against the silver-mine, stood Demas (gentleman-like) to call passengers to come and see; who said to Christian and his fellow, Ho! turn aside hither, and I will show you a thing.

Christian: What thing so deserving as to turn us out of the way to see it?

Demas: Here is a silver-mine, and some digging in it for treasure; if you will come, with a little pains you may richly provide for yourselves.

Hopeful: Then said Hopeful, let us go see.

Christian: Not I, said Christian: I have heard of this place before now, and how many there have been slain; and besides, that treasure is a snare to those that seek it, for it hindereth them in their pilgrimage.

Then Christian called to Demas, saying, Is not the place dangerous? Hath it not hindered many in their pilgrimage? Hosea 9:6.

Demas: Not very dangerous, except to those that are careless; but withal he blushed as he spake.

Christian: Then said Christian to Hopeful, Let us not stir a step, but still keep on our way.

Hopeful: I will warrant you, when By-ends comes up, if he hath the same invitation as we, he will turn in thither to see.

Christian: No doubt thereof, for his principles lead him that way, and a hundred to one but he dies there.

Demas: Then Demas called again, saying, But will you not come over and see?

Christian: Then Christian roundly answered, saying, Demas, thou art an enemy to the right ways of the Lord of this way, and hast been already condemned for thine own turning aside, by one of his Majesty's judges, 2 Tim. 4:10; and why seekest thou to bring us into the like condemnation? Besides, if we at all turn aside, our Lord the King will certainly hear thereof, and will there put us to shame, where we would stand with boldness before him.

Demas cried again, that he also was one of their fraternity; and that if they would tarry a little, he also himself would walk with them.

Christian: Then said Christian, What is thy name? Is it not the same by which I have called thee?

Demas: Yes, my name is Demas; I am the son of Abraham.

Christian: I know you; Gehazi was your great-grandfather, and Judas your father, and you have trod in their steps; it is but a devilish prank that thou usest: thy father was hanged for a traitor, and thou deservest no better reward. 2 Kings 5:20-27; Matt.26:14,15; 27:3-5. Assure thyself, that when we come to the King, we will tell him of this thy behavior. Thus they went their way.

By this time By-ends and his companions were come again within sight, and they at the first beck went over to Demas. Now, whether they fell into the pit by looking over the brink thereof, or whether they went down to dig, or whether they were smothered in the bottom by the damps that commonly arise, of these things I am not certain; but this I observed, that they were never seen again in the way. Then sang Christian,
\begin{verse} 
``By-ends and silver Demas both agree;\\
One calls, the other runs, that he may be\\
A sharer in his lucre: so these two\\
Take up in this world, and no farther go."\\
\end{verse}
Now I saw that, just on the other side of this plain, the pilgrims came to a place where stood an old monument, hard by the highway-side, at the sight of which they were both concerned, because of the strangeness of the form thereof; for it seemed to them as if it had been a woman transformed into the shape of a pillar. Here, therefore, they stood looking and looking upon it, but could not for a time tell what they should make thereof. At last Hopeful espied, written above upon the head thereof, a writing in an unusual hand; but he being no scholar, called to Christian (for he was learned) to see if he could pick out the meaning: so he came, and after a little laying of letters together, he found the same to be this, ``Remember Lot's wife." So he read it to his fellow; after which they both concluded that that was the pillar of salt into which Lot's wife was turned, for her looking back with a covetous heart when she was going from Sodom for safety. Gen. 19:26. Which sudden and amazing sight gave them occasion for this discourse.

Christian: Ah, my brother, this is a seasonable sight: it came opportunely to us after the invitation which Demas gave us to come over to view the hill Lucre; and had we gone over, as he desired us, and as thou wast inclined to do, my brother, we had, for aught I know, been made, like this woman, a spectacle for those that shall come after to behold.

Hopeful: I am sorry that I was so foolish, and am made to wonder that I am not now as Lot's wife; for wherein was the difference betwixt her sin and mine? She only looked back, and I had a desire to go see. Let grace be adored; and let me be ashamed that ever such a thing should be in mine heart.

Christian: Let us take notice of what we see here, for our help from time to come. This woman escaped one judgment, for she fell not by the destruction of Sodom; yet she was destroyed by another, as we see: she is turned into a pillar of salt.

Hopeful: True, and she may be to us both caution and example; caution, that we should shun her sin; or a sign of what judgment will overtake such as shall not be prevented by this caution: so Korah, Dathan, and Abiram, with the two hundred and fifty men that perished in their sin, did also become a sign or example to others to beware. Numb. 16:31,32; 26:9,10. But above all, I muse at one thing, to wit, how Demas and his fellows can stand so confidently yonder to look for that treasure, which this woman but for looking behind her after, (for we read not that she stepped one foot out of the way,) was turned into a pillar of salt; especially since the judgment which overtook her did make her an example within sight of where they are; for they cannot choose but see her, did they but lift up their eyes.

Christian: It is a thing to be wondered at, and it argueth that their hearts are grown desperate in the case; and I cannot tell who to compare them to so fitly, as to them that pick pockets in the presence of the judge, or that will cut purses under the gallows. It is said of the men of Sodom, that they were ``sinners exceedingly," because they were sinners ``before the Lord," that is, in his eyesight, and notwithstanding the kindnesses that he had shown them; for the land of Sodom was now like the garden of Eden as heretofore. Gen. 13:10-13. This, therefore, provoked him the more to jealousy, and made their plague as hot as the fire of the Lord out of heaven could make it. And it is most rationally to be concluded, that such, even such as these are, that shall sin in the sight, yea, and that too in despite of such examples that are set continually before them, to caution them to the contrary, must be partakers of severest judgments.

Hopeful: Doubtless thou hast said the truth; but what a mercy is it, that neither thou, but especially I, am not made myself this example! This ministereth occasion to us to thank God, to fear before him, and always to remember Lot's wife.

I saw then that they went on their way to a pleasant river, which David the king called ``the river of God;" but John, ``the river of the water of life." Psa. 65:9; Rev. 22:1; Ezek. 47:1-9. Now their way lay just upon the bank of this river: here, therefore, Christian and his companion walked with great delight; they drank also of the water of the river, which was pleasant and enlivening to their weary spirits. Besides, on the banks of this river, on either side, were green trees with all manner of fruit; and the leaves they ate to prevent surfeits, and other diseases that are incident to those that heat their blood by travel. On either side of the river was also a meadow, curiously beautified with lilies; and it was green all the year long. In this meadow they lay down and slept, for here they might lie down safely. Psa. 23:2; Isa. 14:30. When they awoke they gathered again of the fruit of the trees, and drank again of the water of the river, and then lay down again to sleep. Thus they did several days and nights. Then they sang;
\begin{verse} 
``Behold ye, how these Crystal Streams do glide,\\
To comfort pilgrims by the highway-side.\\
The meadows green, besides their fragrant smell,\\
Yield dainties for them; And he that can tell\\
What pleasant fruit, yea, leaves these trees do yield,\\
Will soon sell all, that he may buy this field."\\ 
\end{verse}
So when they were disposed to go on, (for they were not as yet at their journey's end,) they ate, and drank, and departed.

Now I beheld in my dream, that they had not journeyed far, but the river and the way for a time parted, at which they were not a little sorry; yet they durst not go out of the way. Now the way from the river was rough, and their feet tender by reason of their travels; so the souls of the pilgrims were much discouraged because of the way. Numb. 21:4. Wherefore, still as they went on, they wished for a better way. Now, a little before them, there was on the left hand of the road a meadow, and a stile to go over into it, and that meadow is called By-path meadow. Then said Christian to his fellow, If this meadow lieth along by our wayside, let's go over into it. Then he went to the stile to see, and behold a path lay along by the way on the other side of the fence. It is according to my wish, said Christian; here is the easiest going; come, good Hopeful, and let us go over.

Hopeful: But how if this path should lead us out of the way?

Christian: That is not likely, said the other. Look, doth it not go along by the wayside? So Hopeful, being persuaded by his fellow, went after him over the stile. When they were gone over, and were got into the path, they found it very easy for their feet; and withal, they, looking before them, espied a man walking as they did, and his name was Vain-Confidence: so they called after him, and asked him whither that way led. He said, To the Celestial Gate. Look, said Christian, did not I tell you so? by this you may see we are right. So they followed, and he went before them. But behold the night came on, and it grew very dark; so that they that went behind lost the sight of him that went before.

He therefore that went before, (Vain-Confidence by name,) not seeing the way before him, fell into a deep pit, which was on purpose there made, by the prince of those grounds, to catch vain-glorious fools withal, and was dashed in pieces with his fall. Isa. 9:16.

Now, Christian and his fellow heard him fall. So they called to know the matter, but there was none to answer, only they heard a groaning. Then said Hopeful, Where are we now? Then was his fellow silent, as mistrusting that he had led him out of the way; and now it began to rain, and thunder, and lighten in a most dreadful manner, and the water rose amain.

Then Hopeful groaned in himself, saying, Oh that I had kept on my way!

Christian: Who could have thought that this path should have led us out of the way?

Hopeful: I was afraid on't at the very first, and therefore gave you that gentle caution. I would have spoke plainer, but that you are older than I.

Christian: Good brother, be not offended; I am sorry I have brought thee out of the way, and that I have put thee into such imminent danger. Pray, my brother, forgive me; I did not do it of an evil intent.

Hopeful: Be comforted, my brother, for I forgive thee; and believe, too, that this shall be for our good.

Christian: I am glad I have with me a merciful brother: but we must not stand here; let us try to go back again.

Hopeful: But, good brother, let me go before.

Christian: No, if you please, let me go first, that if there be any danger, I may be first therein, because by my means we are both gone out of the way.

Hopeful: No, said Hopeful, you shall not go first, for your mind being troubled may lead you out of the way again. Then for their encouragement they heard the voice of one saying, ``Let thine heart be toward the highway, even the way that thou wentest: turn again." Jer. 31:21. But by this time the waters were greatly risen, by reason of which the way of going back was very dangerous. (Then I thought that it is easier going out of the way when we are in, than going in when we are out.) Yet they adventured to go back; but it was so dark, and the flood was so high, that in their going back they had like to have been drowned nine or ten times.

Neither could they, with all the skill they had, get again to the stile that night. Wherefore at last, lighting under a little shelter, they sat down there till the day brake; but being weary, they fell asleep. Now there was, not far from the place where they lay, a castle, called Doubting Castle, the owner whereof was Giant Despair, and it was in his grounds they now were sleeping: wherefore he, getting up in the morning early, and walking up and down in his fields, caught Christian and Hopeful asleep in his grounds. Then with a grim and surly voice, he bid them awake, and asked them whence they were, and what they did in his grounds. They told him they were pilgrims, and that they had lost their way. Then said the giant, You have this night trespassed on me by trampling in and lying on my grounds, and therefore you must go along with me. So they were forced to go, because he was stronger than they. They also had but little to say, for they knew themselves in a fault. The giant, therefore, drove them before him, and put them into his castle, into a very dark dungeon, nasty and stinking to the spirits of these two men. Here, then, they lay from Wednesday morning till Saturday night, without one bit of bread, or drop of drink, or light, or any to ask how they did; they were, therefore, here in evil case, and were far from friends and acquaintance. Psa. 88:18. Now in this place Christian had double sorrow, because it was through his unadvised counsel that they were brought into this distress.

Now Giant Despair had a wife, and her name was Diffidence: so when he was gone to bed he told his wife what he had done, to wit, that he had taken a couple of prisoners, and cast them into his dungeon for trespassing on his grounds. Then he asked her also what he had best do further to them. So she asked him what they were, whence they came, and whither they were bound, and he told her. Then she counseled him, that when he arose in the morning he should beat them without mercy. So when he arose, he getteth him a grievous crab-tree cudgel, and goes down into the dungeon to them, and there first falls to rating of them as if they were dogs, although they gave him never a word of distaste. Then he falls upon them, and beats them fearfully, in such sort that they were not able to help themselves, or to turn them upon the floor. This done, he withdraws and leaves them there to condole their misery, and to mourn under their distress: so all that day they spent the time in nothing but sighs and bitter lamentations. The next night, she, talking with her husband further about them, and understanding that they were yet alive, did advise him to counsel them to make away with themselves. So when morning was come, he goes to them in a surly manner, as before, and perceiving them to be very sore with the stripes that he had given them the day before, he told them, that since they were never like to come out of that place, their only way would be forthwith to make an end of themselves, either with knife, halter, or poison; for why, said he, should you choose to live, seeing it is attended with so much bitterness? But they desired him to let them go. With that he looked ugly upon them, and rushing to them, had doubtless made an end of them himself, but that he fell into one of his fits, (for he sometimes in sunshiny weather fell into fits,) and lost for a time the use of his hands; wherefore he withdrew, and left them as before to consider what to do. Then did the prisoners consult between themselves whether it was best to take his counsel or no; and thus they began to discourse:

Christian: Brother, said Christian, what shall we do? The life that we now live is miserable. For my part, I know not whether it is best to live thus, or to die out of hand. My soul chooseth strangling rather than life, and the grave is more easy for me than this dungeon. Job. 7:15. Shall we be ruled by the giant?

Hopeful: Indeed our present condition is dreadful, and death would be far more welcome to me than thus for ever to abide; but yet, let us consider, the Lord of the country to which we are going hath said, ``Thou shalt do no murder," no, not to another man's person; much more, then, are we forbidden to take his counsel to kill ourselves. Besides, he that kills another, can but commit murder upon his body; but for one to kill himself, is to kill body and soul at once. And moreover, my brother, thou talkest of ease in the grave; but hast thou forgotten the hell whither for certain the murderers go? for ``no murderer hath eternal life," etc. And let us consider again, that all the law is not in the hand of Giant Despair: others, so far as I can understand, have been taken by him as well as we, and yet have escaped out of his hands. Who knows but that God, who made the world, may cause that Giant Despair may die; or that, at some time or other, he may forget to lock us in; or that he may, in a short time, have another of his fits before us, and may lose the use of his limbs? And if ever that should come to pass again, for my part, I am resolved to pluck up the heart of a man, and to try my utmost to get from under his hand. I was a fool that I did not try to do it before. But, however, my brother, let us be patient, and endure a while: the time may come that may give us a happy release; but let us not be our own murderers. With these words Hopeful at present did moderate the mind of his brother; so they continued together in the dark that day, in their sad and doleful condition.

Well, towards evening the giant goes down into the dungeon again, to see if his prisoners had taken his counsel. But when he came there he found them alive; and truly, alive was all; for now, what for want of bread and water, and by reason of the wounds they received when he beat them, they could do little but breathe. But I say, he found them alive; at which he fell into a grievous rage, and told them, that seeing they had disobeyed his counsel, it should be worse with them than if they had never been born.

At this they trembled greatly, and I think that Christian fell into a swoon; but coming a little to himself again, they renewed their discourse about the giant's counsel, and whether yet they had best take it or no. Now Christian again seemed for doing it; but Hopeful made his second reply as followeth:

Hopeful: My brother, said he, rememberest thou not how valiant thou hast been heretofore? Apollyon could not crush thee, nor could all that thou didst hear, or see, or feel, in the Valley of the Shadow of Death. What hardship, terror, and amazement hast thou already gone through; and art thou now nothing but fears! Thou seest that I am in the dungeon with thee, a far weaker man by nature than thou art. Also this giant hath wounded me as well as thee, and hath also cut off the bread and water from my mouth, and with thee I mourn without the light. But let us exercise a little more patience. Remember how thou playedst the man at Vanity Fair, and wast neither afraid of the chain nor cage, nor yet of bloody death: wherefore let us (at least to avoid the shame that it becomes not a Christian to be found in) bear up with patience as well as we can.

Now night being come again, and the giant and his wife being in bed, she asked him concerning the prisoners, and if they had taken his counsel: to which he replied, They are sturdy rogues; they choose rather to bear all hardships than to make away with themselves. Then said she, Take them into the castle-yard to-morrow, and show them the bones and skulls of those that thou hast already dispatched, and make them believe, ere a week comes to an end, thou wilt tear them in pieces, as thou hast done their fellows before them.

So when the morning was come, the giant goes to them again, and takes them into the castle-yard, and shows them as his wife had bidden him. These, said he, were pilgrims, as you are, once, and they trespassed on my grounds, as you have done; and when I thought fit I tore them in pieces; and so within ten days I will do you: get you down to your den again. And with that he beat them all the way thither. They lay, therefore, all day on Saturday in a lamentable case, as before. Now, when night was come, and when Mrs. Diffidence and her husband the giant was got to bed, they began to renew their discourse of their prisoners; and withal, the old giant wondered that he could neither by his blows nor counsel bring them to an end. And with that his wife replied, I fear, said she, that they live in hopes that some will come to relieve them; or that they have picklocks about them, by the means of which they hope to escape. And sayest thou so, my dear? said the giant; I will therefore search them in the morning.

Well, on Saturday, about midnight they began to pray, and continued in prayer till almost break of day.

Now, a little before it was day, good Christian, as one half amazed, brake out into this passionate speech: What a fool, quoth he, am I, thus to lie in a stinking dungeon, when I may as well walk at liberty! I have a key in my bosom, called Promise, that will, I am persuaded, open any lock in Doubting Castle. Then said Hopeful, That is good news; good brother, pluck it out of thy bosom, and try.

Then Christian pulled it out of his bosom, and began to try at the dungeon-door, whose bolt, as he turned the key, gave back, and the door flew open with ease, and Christian and Hopeful both came out. Then he went to the outward door that leads into the castle-yard, and with his key opened that door also. After he went to the iron gate, for that must be opened too; but that lock went desperately hard, yet the key did open it. They then thrust open the gate to make their escape with speed; but that gate, as it opened, made such a creaking, that it waked Giant Despair, who hastily rising to pursue his prisoners, felt his limbs to fail, for his fits took him again, so that he could by no means go after them. Then they went on, and came to the King's highway, and so were safe, because they were out of his jurisdiction.

Now, when they were gone over the stile, they began to contrive with themselves what they should do at that stile, to prevent those that shall come after from falling into the hands of Giant Despair. So they consented to erect there a pillar, and to engrave upon the side thereof this sentence: ``Over this stile is the way to Doubting Castle, which is kept by Giant Despair, who despiseth the King of' the Celestial country, and seeks to destroy his holy pilgrims." Many, therefore, that followed after, read what was written, and escaped the danger. This done, they sang as follows:

\begin{verse} 
``Out of the way we went, and then we found\\
What twas to tread upon forbidden ground:\\
And let them that come after have a care,\\
Lest heedlessness makes them as we to fare;\\
Lest they, for trespassing, his prisoners are,\\
Whose castle's Doubting, and whose name's Despair."\\ 
\end{verse} 
