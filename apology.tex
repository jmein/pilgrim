\subsubsection[APOLOGY]{THE AUTHOR'S APOLOGY FOR HIS BOOK}
\begin{verse}
WHEN at the first I took my pen in hand\\
Thus for to write, I did not understand\\
That I at all should make a little book\\
In such a mode: nay, I had undertook\\
To make another; which, when almost done,\\
Before I was aware I this begun.\\
\end{verse}
\begin{verse}
And thus it was: I, writing of the way\\
And race of saints in this our gospel-day,\\
Fell suddenly into an allegory\\
About their journey, and the way to glory,\\
In more than twenty things which I set down\\
This done, I twenty more had in my crown,\\
And they again began to multiply,\\
Like sparks that from the coals of fire do fly.\\
Nay, then, thought I, if that you breed so fast,\\
I'll put you by yourselves, lest you at last\\
Should prove ad infinitum,\footnote{Without end.} and eat out\\
The book that I already am about.\\
Well, so I did; but yet I did not think\\
To show to all the world my pen and ink\\
In such a mode; I only thought to make\\
I knew not what: nor did I undertake\\
Thereby to please my neighbor; no, not I;\\
I did it my own self to gratify.\\
\end{verse}
\begin{verse}
Neither did I but vacant seasons spend\\
In this my scribble; nor did I intend\\
But to divert myself, in doing this,\\
From worser thoughts, which make me do amiss.\\
Thus I set pen to paper with delight,\\
And quickly had my thoughts in black and white;\\
For having now my method by the end,\\
Still as I pull'd, it came; and so I penned\\
It down; until it came at last to be,\\
For length and breadth, the bigness which you see.\\
\end{verse}
\begin{verse}
Well, when I had thus put mine ends together\\
I show'd them others, that I might see whether\\
They would condemn them, or them justify:\\
And some said, let them live; some, let them die:\\
Some said, John, print it; others said, Not so:\\
Some said, It might do good; others said, No.\\

\end{verse}
\newpage
\begin{verse}
Now was I in a strait, and did not see\\
Which was the best thing to be done by me:\\
At last I thought, Since ye are thus divided,\\
I print it will; and so the case decided.\\
\end{verse}
\begin{verse}
For, thought I, some I see would have it done,\\
Though others in that channel do not run:\\
To prove, then, who advised for the best,\\
Thus I thought fit to put it to the test.\\
\end{verse}
\begin{verse}
I further thought, if now I did deny\\
Those that would have it, thus to gratify;\\
I did not know, but hinder them I might\\
Of that which would to them be great delight.\\
For those which were not for its coming forth,\\
I said to them, Offend you, I am loath;\\
Yet since your brethren pleased with it be,\\
Forbear to judge, till you do further see.\\
\end{verse}
\begin{verse}
If that thou wilt not read, let it alone;\\
Some love the meat, some love to pick the bone.\\
Yea, that I might them better palliate,\\
I did too with them thus expostulate:\\
\end{verse}
\begin{verse}
May I not write in such a style as this?\\
In such a method too, and yet not miss\\
My end-thy good? Why may it not be done?\\
Dark clouds bring waters, when the bright bring none.\\
Yea, dark or bright, if they their silver drops\\
Cause to descend, the earth, by yielding crops,\\
Gives praise to both, and carpeth not at either,\\
But treasures up the fruit they yield together;\\
Yea, so commixes both, that in their fruit\\
None can distinguish this from that; they suit\\
Her well when hungry; but if she be full,\\
She spews out both, and makes their blessing null.\\
\end{verse}
\begin{verse}
You see the ways the fisherman doth take\\
To catch the fish; what engines doth he make!\\
Behold how he engageth all his wits;\\
Also his snares, lines, angles, hooks, and nets:\\
Yet fish there be, that neither hook nor line,\\
Nor snare, nor net, nor engine can make thine:\\
They must be groped for, and be tickled too,\\
Or they will not be catch'd, whate'er you do.\\
\end{verse}
\newpage
\begin{verse}
How does the fowler seek to catch his game\\
By divers means! all which one cannot name.\\
His guns, his nets, his lime-twigs, light and bell:\\
He creeps, he goes, he stands; yea, who can tell\\
Of all his postures? yet there's none of these\\
Will make him master of what fowls he please.\\
Yea, he must pipe and whistle, to catch this;\\
Yet if he does so, that bird he will miss.\\
If that a pearl may in toad's head dwell,\\
And may be found too in an oyster-shell;\\
If things that promise nothing, do contain\\
What better is than gold; who will disdain,\\
That have an inkling\footnote{Hint, whisper, insinuation.} of it, there to look,\\
That they may find it. Now my little book,\\
(Though void of all these paintings that may make
It with this or the other man to take,)\\
Is not without those things that do excel\\
What do in brave but empty notions dwell.\\
``Well, yet I am not fully satisfied\\
That this your book will stand, when soundly tried."\\
\end{verse}
\begin{verse}
Why, what's the matter? ``It is dark." What though?\\
``But it is feigned." What of that? I trow\\
Some men by feigned words, as dark as mine,\\
Make truth to spangle, and its rays to shine.\\
``But they want solidness." Speak, man, thy mind.\\
``They drown the weak; metaphors make us blind."\\
\end{verse}
\begin{verse}
Solidity, indeed, becomes the pen\\
Of him that writeth things divine to men:\\
But must I needs want solidness, because\\
By metaphors I speak? Were not God's laws,\\
His gospel laws, in olden time held forth\\
By types, shadows, and metaphors? Yet loth\\
Will any sober man be to find fault\\
With them, lest he be found for to assault\\
The highest wisdom! No, he rather stoops,\\
And seeks to find out what, by pins and loops,\\
By calves and sheep, by heifers, and by rams,\\
By birds and herbs, and by the blood of lambs,\\
God speaketh to him; and happy is he\\
That finds the light and grace that in them be.\\
\end{verse}
\begin{verse}
But not too forward, therefore, to conclude\\
That I want solidness--that I am rude;\\
All things solid in show, not solid be;\\
All things in parable despise not we,\\
Lest things most hurtful lightly we receive,\\
And things that good are, of our souls bereave.\\
My dark and cloudy words they do but hold\\
The truth, as cabinets inclose the gold.\\
\end{verse}
\begin{verse}
The prophets used much by metaphors\\
To set forth truth: yea, who so considers\\
Christ, his apostles too, shall plainly see,\\
That truths to this day in such mantles be.\\
\end{verse}
\begin{verse}
Am I afraid to say, that holy writ,\\
Which for its style and phrase puts down all wit,\\
Is everywhere so full of all these things,\\
Dark figures, allegories? Yet there springs\\
From that same book, that lustre, and those rays\\
Of light, that turn our darkest nights to days.\\
\end{verse}
\begin{verse}
Come, let my carper to his life now look,\\
And find there darker lines than in my book\\
He findeth any; yea, and let him know,\\
That in his best things there are worse lines too.\\
\end{verse}
\begin{verse}
May we but stand before impartial men,\\
To his poor one I durst adventure ten,\\
That they will take my meaning in these lines\\
Far better than his lies in silver shrines.\\
Come, truth, although in swaddling-clothes, I find\\
Informs the judgment, rectifies the mind;\\
Pleases the understanding, makes the will\\
Submit, the memory too it doth fill\\
With what doth our imagination please;\\
Likewise it tends our troubles to appease.\\
\end{verse}
\begin{verse}
Sound words, I know, Timothy is to use,\\
And old wives' fables he is to refuse;\\
But yet grave Paul him nowhere doth forbid\\
The use of parables, in which lay hid\\
That gold, those pearls, and precious stones that were\\
Worth digging for, and that with greatest care.\\
\end{verse}
\begin{verse}
Let me add one word more. O man of God,\\
Art thou offended? Dost thou wish I had\\
Put forth my matter in another dress?\\
Or that I had in things been more express?\\
Three things let me propound; then I submit\\
To those that are my betters, as is fit.\\
\end{verse}
\begin{verse}
1. I find not that I am denied the use
Of this my method, so I no abuse\\
Put on the words, things, readers, or be rude\\
In handling figure or similitude,\\
In application; but all that I may\\
Seek the advance of truth this or that way.\\
Denied, did I say? Nay, I have leave,\\
(Example too, and that from them that have
God better pleased, by their words or ways,\\
Than any man that breatheth now-a-days,)\\
Thus to express my mind, thus to declare\\
Things unto thee that excellentest are.\\
\end{verse}
\begin{verse}
2. I find that men as high as trees will write
Dialogue-wise; yet no man doth them slight\\
For writing so. Indeed, if they abuse\\
Truth, cursed be they, and the craft they use\\
To that intent; but yet let truth be free\\
To make her sallies upon thee and me,\\
Which way it pleases God: for who knows how,\\
Better than he that taught us first to plough,\\
To guide our minds and pens for his designs?\\
And he makes base things usher in divine.\\
\end{verse}
\begin{verse}
3. I find that holy writ, in many places,
Hath semblance with this method, where the cases\\
Do call for one thing to set forth another:\\
Use it I may then, and yet nothing smother\\
Truth's golden beams: nay, by this method may\\
Make it cast forth its rays as light as day.\\
\end{verse}
\begin{verse}
And now, before I do put up my pen,\\
I'll show the profit of my book; and then\\
Commit both thee and it unto that hand\\
That pulls the strong down, and makes weak ones stand.\\
\end{verse}
\begin{verse}
This book it chalketh out before thine eyes\\
The man that seeks the everlasting prize:\\
It shows you whence he comes, whither he goes,\\
What he leaves undone; also what he does:\\
It also shows you how he runs, and runs,\\
Till he unto the gate of glory comes.\\
It shows, too, who set out for life amain,\\
As if the lasting crown they would obtain;\\
Here also you may see the reason why\\
They lose their labor, and like fools do die.\\
\end{verse}
\begin{verse}
This book will make a traveler of thee,\\
If by its counsel thou wilt ruled be;\\
It will direct thee to the Holy Land,\\
If thou wilt its directions understand\\
Yea, it will make the slothful active be;\\
The blind also delightful things to see.\\
\end{verse}
\begin{verse}
Art thou for something rare and profitable?\\
Or would'st thou see a truth within a fable?\\
Art thou forgetful? Wouldest thou remember\\
From New-Year's day to the last of December?\\
Then read my fancies; they will stick like burs,\\
And may be, to the helpless, comforters.\\
\end{verse}
\begin{verse}
This book is writ in such a dialect\\
As may the minds of listless men affect:\\
It seems a novelty, and yet contains\\
Nothing but sound and honest gospel strains.\\
\end{verse}
\begin{verse}
Would'st thou divert thyself from melancholy?\\
Would'st thou be pleasant, yet be far from folly?\\
Would'st thou read riddles, and their explanation?\\
Or else be drowned in thy contemplation?\\
Dost thou love picking meat? Or would'st thou see\\
A man i' the clouds, and hear him speak to thee?\\
Would'st thou be in a dream, and yet not sleep?\\
Or would'st thou in a moment laugh and weep?\\
Would'st thou lose thyself and catch no harm,\\
And find thyself again without a charm?\\
Would'st read thyself, and read thou know'st not what,\\
And yet know whether thou art blest or not,\\
By reading the same lines? O then come hither,\\
And lay my book, thy head, and heart together.\\
\end{verse}


JOHN BUNYAN.\\
