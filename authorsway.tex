\newpage
\subsubsection[SENDING FORTH]{THE AUTHOR'S WAY OF SENDING FORTH HIS SECOND  OF THE PILGRIM}
\begin{verse}
Go, now, my little Book, to every place\\
Where my first Pilgrim has but shown his face:\\
Call at their door: if any say, Who's there?\\
Then answer thou, Christiana is here.\\
If they bid thee come in, then enter thou,\\
With all thy boys; and then, as thou know'st how,\\
Tell who they are, also from whence they came;\\
Perhaps they'll know them by their looks, or name:\\
But if they should not, ask them yet again,\\
If formerly they did not entertain\\
One Christian, a Pilgrim? If they say\\
They did, and were delighted in his way;\\
Then let them know that these related were\\
Unto him; yea, his wife and children are.\\
\end{verse}
\begin{verse}
Tell them, that they have left their house and home;\\
Are turned Pilgrims; seek a world to come;\\
That they have met with hardships in the way;\\
That they do meet with troubles night and day;\\
That they have trod on serpents; fought with devils;\\
Have also overcome a many evils;\\
Yea, tell them also of the next who have,\\
Of love to pilgrimage, been stout and brave\\
Defenders of that way; and how they still\\
Refuse this world to do their Father's will.\\
Go tell them also of those dainty things\\
That pilgrimage unto the Pilgrim brings.\\
Let them acquainted be, too, how they are\\
Beloved of their King, under his care;\\
What goodly mansions he for them provides;\\
Though they meet with rough winds and swelling tides,\\
How brave a calm they will enjoy at last,\\
Who to their Lord, and by his ways hold fast.\\
\end{verse}
\begin{verse}
Perhaps with heart and hand they will embrace\\
Thee, as they did my firstling; and will grace\\
Thee and thy fellows with such cheer and fare,\\
As show well, they of Pilgrims lovers are.\\
\end{verse}
Objection I
\begin{verse}
But how if they will not believe of me\\
That I am truly thine? cause some there be\\
That counterfeit the Pilgrim and his name,\\
Seek, by disguise, to seem the very same;\\
And by that means have wrought themselves into\\
The hands and houses of I know not who.\\
\end{verse}
Answer
\begin{verse}
'Tis true, some have, of late, to counterfeit\\
My Pilgrim, to their own my title set;\\
Yea, others half my name, and title too,\\
Have stitched to their books, to make them do.\\
But yet they, by their features, do declare\\
Themselves not mine to be, whose'er they are.\\
\end{verse}
\begin{verse}
If such thou meet'st with, then thine only way\\
Before them all, is, to say out thy say\\
In thine own native language, which no man\\
Now useth, nor with ease dissemble can.\\
\end{verse}
\begin{verse}
If, after all, they still of you shall doubt,\\
Thinking that you, like gypsies, go about,\\
In naughty wise the country to defile;\\
Or that you seek good people to beguile\\
With things unwarrantable; send for me,\\
And I will testify you pilgrims be;\\
Yea, I will testify that only you\\
My Pilgrims are, and that alone will do.\\
\end{verse}
Objection II
\begin{verse}
But yet, perhaps, I may enquire for him\\
Of those who wish him damned life and limb.\\
What shall I do, when I at such a door\\
For Pilgrims ask, and they shall rage the more?\\
\end{verse}
Answer
\begin{verse}
Fright not thyself, my Book, for such bugbears\\
Are nothing else but groundless fears.\\
My Pilgrim's book has traveled sea and land,\\
Yet could I never come to understand\\
That it was slighted or turned out of door\\
By any Kingdom, were they rich or poor.\\
In France and Flanders, where men kill each other,\\
My Pilgrim is esteemed a friend, a brother.\\
\end{verse}
\begin{verse}
In Holland, too, tis said, as I am told,\\
My Pilgrim is with some, worth more than gold.\\
Highlanders and wild Irish can agree\\
My Pilgrim should familiar with them be.\\
\end{verse}
\begin{verse}
'Tis in New England under such advance,\\
Receives there so much loving countenance,\\
As to be trimm'd, newcloth'd, and deck'd with gems,\\
That it might show its features, and its limbs.\\
Yet more: so comely doth my Pilgrim walk,\\
That of him thousands daily sing and talk.\\
\end{verse}
\begin{verse}
If you draw nearer home, it will appear\\
My Pilgrim knows no ground of shame or fear:\\
City and country will him entertain,\\
With Welcome, Pilgrim; yea, they can't refrain\\
From smiling, if my Pilgrim be but by,\\
Or shows his head in any company.\\
\end{verse}
\begin{verse}
Brave gallants do my Pilgrim hug and love,\\
Esteem it much, yea, value it above\\
Things of greater bulk; yea, with delight\\
Say, my lark's leg is better than a kite.\\
Young ladies, and young gentlewomen too,\\
Do not small kindness to my Pilgrim show;\\
Their cabinets, their bosoms, and their hearts,\\
My Pilgrim has; 'cause he to them imparts\\
His pretty riddles in such wholsome strains,\\
As yield them profit double to thetr pains\\
Of reading; yea, I think I may be bold\\
To say some prize him far above their gold.\\
The very children that do walk the street,\\
If they do but my holy Pilgrim meet,\\
Salute him will; will wish him well, and say,\\
He is the only stripling of the day.\\
\end{verse}
\begin{verse}
They that have never seen him, yet admire\\
What they have heard of him, and much desire\\
To have his company, and hear him tell\\
Those Pilgrim stories which he knows so well.\\
\end{verse}
\begin{verse}
Yea, some that did not love him at first,\\
But call'd him fool and noddy, say they must,\\
Now they have seen and heard him, him commend\\
And to those whom they love they do him send.\\
\end{verse}
\begin{verse}
Wherefore, my Second Part, thou need'st not be\\
Afraid to show thy head: none can hurt thee,\\
That wish but well to him that went before;\\
'Cause thou com'st after with a second store\\
Of things as good, as rich, as profitable,\\
For young, for old, for stagg'ring, and for stable.\\
\end{verse}
Objection III
\begin{verse}
But some there be that say, He laughs too loud\\
And some do say, His Head is in a cloud.\\
Some say, His words and stories are so dark,\\
They know not how, by them, to find his mark.\\
\end{verse}
Answer
\begin{verse}
One may, I think, say, Both his laughs and cries\\
May well be guess'd at by his wat'ry eyes.\\
Some things are of that nature, as to make\\
One's fancy chuckle, while his heart doth ache:\\
When Jacob saw his Rachel with the sheep,\\
He did at the same time both kiss and weep.\\
\end{verse}
\begin{verse}
Whereas some say, A cloud is in his head;\\
That doth but show his wisdom's covered\\
With its own mantles--and to stir the mind\\
To search well after what it fain would find,\\
Things that seem to be hid in words obscure\\
Do but the godly mind the more allure\\
To study what those sayings should contain,\\
That speak to us in such a cloudy strain.\\
I also know a dark similitude\\
Will on the curious fancy more intrude,\\
And will stick faster in the heart and head,\\
Than things from similes not borrowed.\\
\end{verse}
\begin{verse}
Wherefore, my Book, let no discouragement\\
Hinder thy travels. Behold, thou art sent\\
To friends, not foes; to friends that will give place\\
To thee, thy pilgrims, and thy words embrace.\\
\end{verse}
\begin{verse}
Besides, what my first Pilgrim left conceal'd,\\
Thou, my brave second Pilgrim, hast reveal'd;\\
What Christian left lock'd up, and went his way,\\
Sweet Christiana opens with her key.\\
\end{verse}
Objection IV
\begin{verse}
But some love not the method of your first:\\
Romance they count it; throw't away as dust.\\
If I should meet with such, what should I say?\\
Must I slight them as they slight me, or nay?\\
\end{verse}
\newpage
Answer
\begin{verse}
My Christiana, if with such thou meet,\\
By all means, in all loving wise them greet;\\
Render them not reviling for revile,\\
But, if they frown, I prithee on them smile:\\
Perhaps tis nature, or some ill report,\\
Has made them thus despise, or thus retort.\\
\end{verse}
\begin{verse}
Some love no fish, some love no cheese, and some\\
Love not their friends, nor their own house or home;\\
Some start at pig, slight chicken, love not fowl\\
More than they love a cuckoo or an owl.\\
Leave such, my Christiana, to their choice,\\
And seek those who to find thee will rejoice;\\
By no means strive, but, in most humble wise,\\
Present thee to them in thy Pilgrim's guise.\\
\end{verse}
\begin{verse}
Go then, my little Book, and show to all\\
That entertain and bid thee welcome shall,\\
What thou shalt keep close shut up from the rest;\\
And wish what thou shalt show them may be bless'd\\
To them for good, and make them choose to be\\
Pilgrims, by better far than thee or me.\\
Go, then, I say, tell all men who thou art:\\
Say, I am Christiana; and my part\\
Is now, with my four sons, to tell you what\\
It is for men to take a Pilgrim's lot.\\
\end{verse}
\begin{verse}
Go, also, tell them who and what they be\\
That now do go on pilgrimage with thee;\\
Say, Here's my neighbor Mercy: she is one\\
That has long time with me a pilgrim gone:\\
Come, see her in her virgin face, and learn\\
'Twixt idle ones and pilgrims to discern.\\
Yea, let young damsels learn of her to prize\\
The world which is to come, in any wise.\\
When little tripping maidens follow God,\\
And leave old doting sinners to his rod,\\
'Tis like those days wherein the young ones cried\\
Hosanna! when the old ones did deride.\\
\end{verse}
\begin{verse}
Next tell them of old Honest, whom you found\\
With his white hairs treading the Pilgrim's ground;\\
Yea, tell them how plain-hearted this man was;\\
How after his good Lord he bare the cross.\\
Perhaps with some gray head, this may prevail\\
With Christ to fall in love, and sin bewail.\\
\end{verse}
\begin{verse}
Tell them also, how Master Fearing went\\
On pilgrimage, and how the time he spent\\
In solitariness, with fears and cries;\\
And how, at last, he won the joyful prize.\\
He was a good man, though much down in spirit;\\
He is a good man, and doth life inherit.\\
\end{verse}
\begin{verse}
Tell them of Master Feeble-mind also,\\
Who not before, but still behind would go.\\
Show them also, how he had like been slain,\\
And how one Great-Heart did his life regain.\\
This man was true of heart; though weak in grace,\\
One might true godliness read in his face.\\
\end{verse}
\begin{verse}
Then tell them of Master Ready-to-Halt,\\
A man with crutches, but much without fault.\\
Tell them how Master Feeble-mind and he\\
Did love, and in opinion much agree.\\
And let all know, though weakness was their chance,\\
Yet sometimes one could sing, the other dance.\\
\end{verse}
\begin{verse}
Forget not Master Valiant-for-the-Truth,\\
That man of courage, though a very youth:\\
Tell every one his spirit was so stout,\\
No man could ever make him face about;\\
And how Great-Heart and he could not forbear,\\
But pull down Doubting-Castle, slay Despair!\\
\end{verse}
\begin{verse}
Overlook not Master Despondency,\\
Nor Much-afraid, his daughter, though they lie\\
Under such mantles, as may make them look\\
(With some) as if their God had them forsook.\\
They softly went, but sure; and, at the end,\\
Found that the Lord of Pilgrims was their friend.\\
When thou hast told the world of all these things,\\
Then turn about, my Book, and touch these strings;\\
Which, if but touched, will such music make,\\
They'll make a cripple dance, a giant quake.\\
\end{verse}
\begin{verse}
Those riddles that lie couched within thy breast,\\
Freely propound, expound; and for the rest\\
Of thy mysterious lines, let them remain\\
For those whose nimble fancies shall them gain.\\
\end{verse}
\begin{verse}
Now may this little Book a blessing be\\
To those who love this little Book and me;\\
And may its buyer have no cause to say,\\
His money is but lost or thrown away.\\
Yea, may this second Pilgrim yield that fruit\\
As may with each good Pilgrim's fancy suit;\\
And may it some persuade, that go astray,\\
To turn their feet and heart to the right way,\\
\end{verse}

Is the hearty prayer of The Author, 

JOHN BUNYAN.
