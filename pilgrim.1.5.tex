\chapter[THE FIFTH STAGE]{}

Now, as Christian went on his way, he came to a little ascent, which was cast up on purpose that pilgrims might see before them: up there, therefore, Christian went; and looking forward, he saw Faithful before him upon his journey: Then said Christian aloud, Ho, ho; so-ho; stay, and I will be your companion. At that Faithful looked behind him; to whom Christian cried again, Stay, stay, till I come up to you. But Faithful answered, No, I am upon my life, and the avenger of blood is behind me.

At this Christian was somewhat moved, and putting to all his strength, he quickly got up with Faithful, and did also overrun him; so the last was first. Then did Christian vaingloriously smile, because he had gotten the start of his brother; but not taking good heed to his feet, he suddenly stumbled and fell, and could not rise again until Faithful came up to help him.

Then I saw in my dream, they went very lovingly on together, and had sweet discourse of all things that had happened to them in their pilgrimage; and thus Christian began.

Christian: My honored and well-beloved brother Faithful, I am glad that I have overtaken you, and that God has so tempered our spirits that we can walk as companions in this so pleasant a path.

Faithful: I had thought, my dear friend, to have had your company quite from our town, but you did get the start of me; wherefore I was forced to come thus much of the way alone.

Christian: How long did you stay in the city of Destruction before you set out after me on your pilgrimage?

Faithful: Till I could stay no longer; for there was a great talk presently after you were gone out, that our city would, in a short time, with fire from heaven, be burnt down to the ground.

Christian: What, did your neighbors talk so?

Faithful: Yes, it was for a while in every body's mouth.

Christian: What, and did no more of them but you come out to escape the danger?

Faithful: Though there was, as I said, a great talk thereabout, yet I do not think they did firmly believe it; for, in the heat of the discourse, I heard some of them deridingly speak of you and of your desperate journey, for so they called this your pilgrimage. But I did believe, and do still, that the end of our city will be with fire and brimstone from above; and therefore I have made my escape.

Christian: Did you hear no talk of neighbor Pliable?

Faithful: Yes, Christian, I heard that he followed you till he came to the Slough of Despond, where, as some said, he fell in; but he would not be known to have so done: but I am sure he was soundly bedabbled with that kind of dirt.

Christian: And what said the neighbors to him?

Faithful: He hath, since his going back, been had greatly in derision, and that among all sorts of people: some do mock and despise him, and scarce will any set him on work. He is now seven times worse than if he had never gone out of the city.

Christian: But why should they be so set against him, since they also despise the way that he forsook?

Faithful: O, they say, Hang him; he is a turncoat; he was not true to his profession! I think God has stirred up even His enemies to hiss at him, and make him a proverb, because he hath forsaken the way. Jer. 29:18,19.

Christian: Had you no talk with him before you came out?

Faithful: I met him once in the streets, but he leered away on the other side, as one ashamed of what he had done; So I spake not to him.

Christian: Well, at my first setting out I had hopes of that man; but now I fear he will perish in the overthrow of the city. For it has happened to him according to the true proverb, The dog is turned to his vomit again, and the sow that was washed to her wallowing in the mire. 2 Pet. 2:22.

Faithful: These are my fears of him too; but who can hinder that which will be?

Christian: Well, neighbor Faithful, said Christian, let us leave him, and talk of things that more immediately concern ourselves. Tell me now what you have met with in the way as you came; for I know you have met with some things, or else it may be writ for a wonder.

Faithful: I escaped the slough that I perceive you fell into, and got up to the gate without that danger; only I met with one whose name was Wanton, that had like to have done me mischief.

Christian: It was well you escaped her net: Joseph was hard put to it by her, and he escaped her as you did; but it had like to have cost him his life. Gen. 39:11-13. But what did she do to you?

Faithful: You cannot think (but that you know something) what a flattering tongue she had; she lay at me hard to turn aside with her, promising me all manner of content.

Christian: Nay, she did not promise you the content of a good conscience.

Faithful: You know what I mean; all carnal and fleshly content.

Christian: Thank God that you escaped her: the abhorred of the Lord shall fall into her pit. Prov. 22:14.

Faithful: Nay, I know not whether I did wholly escape her or no.

Christian: Why, I trow you did not consent to her desires?

Faithful: No, not to defile myself; for I remembered an old writing that I had seen, which said, ``Her steps take hold on Hell." Prov. 5:5. So I shut mine eyes, because I would not be bewitched with her looks. Job 31:1. Then she railed on me, and I went my way.

Christian: Did you meet with no other assault as you came?

Faithful: When I came to the foot of the hill called Difficulty, I met with a very aged man, who asked me what I was, and whither bound. I told him that I was a pilgrim, going to the Celestial City. Then said the old man, Thou lookest like an honest fellow; wilt thou be content to dwell with me for the wages that I shall give thee? Then I asked his name, and where he dwelt? He said his name was Adam the First, and that he dwelt in the town of Deceit. Eph. 4:22. I asked him then what was his work, and what the wages that he would give. He told me that his work was many delights; and his wages, that I should be his heir at last. I further asked him, what house he kept, and what other servants he had. So he told me that his house was maintained with all the dainties of the world, and that his servants were those of his own begetting. Then I asked how many children he had. He said that he had but three daughters, the Lust of the Flesh, the Lust of the Eyes, and the Pride of Life, 1 John, 2:16; and that I should marry them if I would. Then I asked, how long time he would have me live with him; And he told me, as long as he lived himself.

Christian: Well, and what conclusion came the old man and you to at last?

Faithful: Why, at first I found myself somewhat inclinable to go with the man, for I thought he spake very fair; but looking in his forehead, as I talked with him, I saw there written, ``Put off the old man with his deeds."

Christian: And how then?

Faithful: Then it came burning hot into my mind, that, whatever he said, and however he flattered, when he got me home to his house he would sell me for a slave. So I bid him forbear to talk, for I would not come near the door of his house. Then he reviled me, and told me that he would send such a one after me that should make my way bitter to my soul. So I turned to go away from him; but just as I turned myself to go thence, I felt him take hold of my flesh, and give me such a deadly twitch back, that I thought he had pulled part of me after himself: this made me cry, ``O wretched man." Rom. 7:24. So I went on my way up the hill.

Now, when I had got above half-way up, I looked behind me, and saw one coming after me, swift as the wind; so he overtook me just about the place where the settle stands.

Christian: Just there, said Christian, did I sit down to rest me; but being overcome with sleep, I there lost this roll out of my bosom.

Faithful: But, good brother, hear me out. So soon as the man overtook me, it was but a word and a blow; for down he knocked me, and laid me for dead. But when I was a little come to myself again I asked him wherefore he served me so. He said because of my secret inclining to Adam the First. And with that he struck me another deadly blow on the breast, and beat me down backward; so I lay at his foot as dead as before. So when I came to myself again I cried him mercy: but he said, I know not how to show mercy; and with that he knocked me down again. He had doubtless made an end of me, but that one came by and bid him forbear.

Christian: Who was that that bid him forbear?

Faithful: I did not know him at first: but as he went by, I perceived the holes in his hands and in his side: Then I concluded that he was our Lord. So I went up the hill.

Christian: That man that overtook you was Moses. He spareth none; neither knoweth he how to shew mercy to those that transgress the law.

Faithful: I know it very well; it was not the first time that he has met with me. Twas he that came to me when I dwelt securely at home, and that told me he would burn my house over my head if I stayed there.

Christian: But did you not see the house that stood there on the top of the hill, on the side of which Moses met you?

Faithful: Yes, and the lions too, before I came at it. But, for the lions, I think they were asleep, for it was about noon; and because I had so much of the day before me, I passed by the Porter, and came down the hill.

Christian: He told me, indeed, that he saw you go by; but I wish you had called at the house, for they would have showed you so many rarities that you would scarce have forgot them to the day of your death. But pray tell me, Did you meet nobody in the Valley of Humility?

Faithful: Yes, I met with one Discontent, who would willingly have persuaded me to go back again with him: his reason was, for that the valley was altogether without honor. He told me, moreover, that to go there was the way to disoblige all my friends, as Pride, Arrogancy, Self-Conceit, Worldly Glory, with others, who he knew, as he said, would be very much offended if I made such a fool of myself as to wade through this valley.

Christian: Well, and how did you answer him?

Faithful: I told him, that although all these that he named, might claim a kindred of me, and that rightly, (for indeed they were my relations according to the flesh,) yet since I became a pilgrim they have disowned me, and I also have rejected them; and therefore they were to me now no more than if they had never been of my lineage. I told him, moreover, that as to this valley, he had quite misrepresented the thing; for before honor is humility, and a haughty spirit before a fall. Therefore, said I, I had rather go through this valley to the honor that was so accounted by the wisest, than choose that which he esteemed most worthy of our affections.

Christian: Met you with nothing else in that valley?

Faithful: Yes, I met with Shame; but of all the men that I met with on my pilgrimage, he, I think, bears the wrong name. The other would be said nay, after a little argumentation, and somewhat else; but this bold-faced Shame would never have done.

Christian: Why, what did he say to you?

Faithful: What? why, he objected against religion itself. He said it was a pitiful, low, sneaking business for a man to mind religion. He said, that a tender conscience was an unmanly thing; and that for a man to watch over his words and ways, so as to tie up himself from that hectoring liberty that the brave spirits of the times accustomed themselves unto, would make him the ridicule of the times. He objected also, that but few of the mighty, rich, or wise, were ever of my opinion; nor any of them neither, before they were persuaded to be fools, and to be of a voluntary fondness to venture the loss of all for nobody knows what. 1 Cor. 1:26; 3:18; Phil. 3:7-9; John 7:48. He, moreover, objected the base and low estate and condition of those that were chiefly the pilgrims of the times in which they lived; also their ignorance and want of understanding in all natural science. Yea, he did hold me to it at that rate also, about a great many more things than here I relate; as, that it was a shame to sit whining and mourning under a sermon, and a shame to come sighing and groaning home; that it was a shame to ask my neighbor forgiveness for petty faults, or to make restitution where I have taken from any. He said also, that religion made a man grow strange to the great, because of a few vices, which he called by finer names, and made him own and respect the base, because of the same religious fraternity: And is not this, said he, a shame?

Christian: And what did you say to him?

Faithful: Say? I could not tell what to say at first. Yea, he put me so to it, that my blood came up in my face; even this Shame fetched it up, and had almost beat me quite off. But at last I began to consider, that that which is highly esteemed among men, is had in abomination with God. Luke 16:15. And I thought again, this Shame tells me what men are; but he tells me nothing what God, or the word of God is. And I thought, moreover, that at the day of doom we shall not be doomed to death or life according to the hectoring spirits of the world, but according to the wisdom and law of the Highest. Therefore, thought I, what God says is best, is indeed best, though all the men in the world are against it. Seeing, then, that God prefers his religion; seeing God prefers a tender Conscience; seeing they that make themselves fools for the kingdom of heaven are wisest, and that the poor man that loveth Christ is richer than the greatest man in the world that hates him; Shame, depart, thou art an enemy to my salvation. Shall I entertain thee against my sovereign Lord? How then shall I look him in the face at his coming? Mark 8:38. Should I now be ashamed of his ways and servants, how can I expect the blessing? But indeed this Shame was a bold villain; I could scarcely shake him out of my company; yea, he would be haunting of me, and continually whispering me in the ear, with some one or other of the infirmities that attend religion. But at last I told him, that it was but in vain to attempt farther in this business; for those things that he disdained, in those did I see most glory: and so at last I got past this importunate one. And when I had shaken him off, then I began to sing,
\begin{verse}
``The trials that those men do meet withal,\\
That are obedient to the heavenly call,\\
Are manifold, and suited to the flesh,\\
And come, and come, and come again afresh;\\
That now, or some time else, we by them may\\
Be taken, overcome, and cast away.\\
O let the pilgrims, let the pilgrims then,\\
Be vigilant, and quit themselves like men."\\
\end{verse}
Christian: I am glad, my brother, that thou didst withstand this villain so bravely; for of all, as thou sayest, I think he has the wrong name; for he is so bold as to follow us in the streets, and to attempt to put us to shame before all men; that is, to make us ashamed of that which is good. But if he was not himself audacious, he would never attempt to do as he does. But let us still resist him; for, notwithstanding all his bravadoes, he promoteth the fool, and none else. ``The wise shall inherit glory," said Solomon; ``but shame shall be the promotion of fools." Prov. 3:35.

Faithful: I think we must cry to Him for help against Shame, that would have us to be valiant for truth upon the earth.

Christian: You say true; but did you meet nobody else in that valley?

Faithful: No, not I; for I had sunshine all the rest of the way through that, and also through the Valley of the Shadow of Death.

Christian: Twas well for you; I am sure it fared far otherwise with me. I had for a long season, as soon almost as I entered into that valley, a dreadful combat with that foul fiend Apollyon; yea, I thought verily he would have killed me, especially when he got me down, and crushed me under him, as if he would have crushed me to pieces; for as he threw me, my sword flew out of my hand: nay, he told me he was sure of me; but I cried to God, and he heard me, and delivered me out of all my troubles. Then I entered into the Valley of the Shadow of Death, and had no light for almost half the way through it. I thought I should have been killed there over and over; but at last day brake, and the sun rose, and I went through that which was behind with far more ease and quiet.

Moreover, I saw in my dream, that as they went on, Faithful, as he chanced to look on one side, saw a man whose name was Talkative, walking at a distance beside them; for in this place there was room enough for them all to walk. He was a tall man, and something more comely at a distance than at hand. To this man Faithful addressed himself in this manner.

Faithful: Friend, whither away? Are you going to the heavenly country?

Talkative: I am going to the same place.

Faithful: That is well; then I hope we shall have your good company?

Talkative: With a very good will, will I be your companion.

Faithful: Come on, then, and let us go together, and let us spend our time in discoursing of things that are profitable.

Talkative: To talk of things that are good, to me is very acceptable, with you or with any other; and I am glad that I have met with those that incline to so good a work; for, to speak the truth, there are but few who care thus to spend their time as they are in their travels, but choose much rather to be speaking of things to no profit; and this hath been a trouble to me.

Faithful: That is, indeed, a thing to be lamented; for what thing so worthy of the use of the tongue and mouth of men on earth, as are the things of the God of heaven?

Talkative: I like you wonderful well, for your saying is full of conviction; and I will add, What thing is so pleasant, and what so profitable, as to talk of the things of God? What things so pleasant? that is, if a man hath any delight in things that are wonderful. For instance, if a man doth delight to talk of the history, or the mystery of things; or if a man doth love to talk of miracles, wonders, or signs, where shall he find things recorded so delightful, and so sweetly penned, as in the holy Scripture?

Faithful: That is true; but to be profited by such things in our talk, should be our chief design.

Talkative: That's it that I said; for to talk of such things is most profitable; for by so doing a man may get knowledge of many things; as of the vanity of earthly things, and the benefit of things above. Thus in general; but more particularly, by this a man may learn the necessity of the new birth, the insufficiency of our works, the need of Christ's righteousness, etc. Besides, by this a man may learn what it is to repent, to believe, to pray, to suffer, or the like: by this, also, a man may learn what are the great promises and consolations of the Gospel, to his own comfort. Farther, by this a man may learn to refute false opinions, to vindicate the truth, and also to instruct the ignorant.

Faithful: All this is true; and glad am I to hear these things from you.

Talkative: Alas! the want of this is the cause that so few understand the need of faith, and the necessity of a work of grace in their soul, in order to eternal life; but ignorantly live in the works of the law, by which a man can by no means obtain the kingdom of heaven.

Faithful: But, by your leave, heavenly knowledge of these is the gift of God; no man attaineth to them by human industry, or only by the talk of them.

Talkative: All this I know very well; for a man can receive nothing, except it be given him from heaven: all is of grace, not of works. I could give you a hundred scriptures for the confirmation of this.

Faithful: Well, then, said Faithful, what is that one thing that we shall at this time found our discourse upon?

Talkative: What you will. I will talk of things heavenly, or things earthly; things moral, or things evangelical; things sacred, or things profane; things past, or things to come; things foreign, or things at home; things more essential, or things circumstantial: provided that all be done to our profit.

Faithful: Now did Faithful begin to wonder; and stepping to Christian, (for he walked all this while by himself,) he said to him, but softly, What a brave companion have we got! Surely, this man will make a very excellent pilgrim.

Christian: At this Christian modestly smiled, and said, This man, with whom you are so taken, will beguile with this tongue of his, twenty of them that know him not.

Faithful: Do you know him, then?

Christian: Know him? Yes, better than he knows himself.

Faithful: Pray what is he?

Christian: His name is Talkative: he dwelleth in our town. I wonder that you should be a stranger to him, only I consider that our town is large.

Faithful: Whose son is he? And whereabout doth he dwell?

Christian: He is the son of one Say-well. He dwelt in Prating-Row; and he is known to all that are acquainted with him by the name of Talkative of Prating-Row; and, notwithstanding his fine tongue, he is but a sorry fellow.

Faithful: Well, he seems to be a very pretty man.

Christian: That is, to them that have not a thorough acquaintance with him, for he is best abroad; near home he is ugly enough. Your saying that he is a pretty man, brings to my mind what I have observed in the work of a painter, whose pictures show best at a distance; but very near, more unpleasing.

Faithful: But I am ready to think you do but jest, because you smiled.

Christian: God forbid that I should jest (though I smiled) in this matter, or that I should accuse any falsely. I will give you a further discovery of him. This man is for any company, and for any talk; as he talketh now with you, so will he talk when he is on the ale-bench; and the more drink he hath in his crown, the more of these things he hath in his mouth. Religion hath no place in his heart, or house, or conversation; all he hath lieth in his tongue, and his religion is to make a noise therewith.

Faithful: Say you so? Then am I in this man greatly deceived.

Christian: Deceived! you may be sure of it. Remember the proverb, ``They say, and do not;" but the kingdom of God is not in word, but in power. Matt. 23:3; 1 Cor. 4:20. He talketh of prayer, of repentance, of faith, and of the new birth; but he knows but only to talk of them. I have been in his family, and have observed him both at home and abroad; and I know what I say of him is the truth. His house is as empty of religion as the white of an egg is of savor. There is there neither prayer, nor sign of repentance for sin; yea, the brute, in his kind, serves God far better than he. He is the very stain, reproach, and shame of religion to all that know him, Rom. 2:24,25; it can hardly have a good word in all that end of the town where he dwells, through him. Thus say the common people that know him, ``A saint abroad, and a devil at home." His poor family finds it so; he is such a churl, such a railer at, and so unreasonable with his servants, that they neither know how to do for or speak to him. Men that have any dealings with him say, It is better to deal with a Turk than with him, for fairer dealings they shall have at their hands. This Talkative (if it be possible) will go beyond them, defraud, beguile, and overreach them. Besides, he brings up his sons to follow his steps; and if he finds in any of them a foolish timorousness, (for so he calls the first appearance of a tender conscience,) he calls them fools and blockheads, and by no means will employ them in much, or speak to their commendation before others. For my part, I am of opinion that he has, by his wicked life, caused many to stumble and fall; and will be, if God prevents not, the ruin of many more.

Faithful: Well, my brother, I am bound to believe you, not only because you say you know him, but also because, like a Christian, you make your reports of men. For I cannot think that you speak these things of ill-will, but because it is even so as you say.

Christian: Had I known him no more than you, I might, perhaps, have thought of him as at the first you did; yea, had I received this report at their hands only that are enemies to religion, I should have thought it had been a slander-a lot that often falls from bad men's mouths upon good men's names and professions. But all these things, yea, and a great many more as bad, of my own knowledge, I can prove him guilty of. Besides, good men are ashamed of him; they can neither call him brother nor friend; the very naming of him among them makes them blush, if they know him.

Faithful: Well, I see that saying and doing are two things, and hereafter I shall better observe this distinction.

Christian: They are two things indeed, and are as diverse as are the soul and the body; for, as the body without the soul is but a dead carcass, so saying, if it be alone, is but a dead carcass also. The soul of religion is the practical part. ``Pure religion and undefiled before God and the Father is this, to visit the fatherless and widows in their affliction, and to keep himself unspotted from the world." James 1:27; see also verses 22-26. This, Talkative is not aware of; he thinks that hearing and saying will make a good Christian; and thus he deceiveth his own soul. Hearing is but as the sowing of the seed; talking is not sufficient to prove that fruit is indeed in the heart and life. And let us assure ourselves, that at the day of doom men shall be judged according to their fruits. Matt. 13:23. It will not be said then, Did you believe? but, Were you doers, or talkers only? and accordingly shall they be judged. The end of the world is compared to our harvest, Matt. 13:30, and you know men at harvest regard nothing but fruit. Not that any thing can be accepted that is not of faith; but I speak this to show you how insignificant the profession of Talkative will be at that day.

Faithful: This brings to my mind that of Moses, by which he describeth the beast that is clean. Lev. 11; Deut. 14. He is such an one that parteth the hoof, and cheweth the cud; not that parteth the hoof only, or that cheweth the cud only. The hare cheweth the cud, but yet is unclean, because he parteth not the hoof. And this truly resembleth Talkative: he cheweth the cud, he seeketh knowledge; he cheweth upon the word, but he divideth not the hoof. He parteth not with the way of sinners; but, as the hare, he retaineth the foot of the dog or bear, and therefore he is unclean.

Christian: You have spoken, for aught I know, the true gospel sense of these texts. And I will add another thing: Paul calleth some men, yea, and those great talkers too, sounding brass, and tinkling cymbals, 1 Cor. 13:1, 3; that is, as he expounds them in another place, things without life giving sound. 1 Cor. 14:7. Things without life; that is, without the true faith and grace of the gospel; and consequently, things that shall never be placed in the kingdom of heaven among those that are the children of life; though their sound, by their talk, be as if it were the tongue or voice of an angel.

Faithful: Well, I was not so fond of his company at first, but I am as sick of it now. What shall we do to be rid of him?

Christian: Take my advice, and do as I bid you, and you shall find that he will soon be sick of your company too, except God shall touch his heart, and turn it.

Faithful: What would you have me to do?

Christian: Why, go to him, and enter into some serious discourse about the power of religion; and ask him plainly, (when he has approved of it, for that he will,) whether this thing be set up in his heart, house, or conversation.

Faithful: Then Faithful stepped forward again, and said to Talkative, Come, what cheer? How is it now?

Talkative: Thank you, well: I thought we should have had a great deal of talk by this time.

Faithful: Well, if you will, we will fall to it now; and since you left it with me to state the question, let it be this: How doth the saving grace of God discover itself when it is in the heart of man?

Talkative: I perceive, then, that our talk must be about the power of things. Well, it is a very good question, and I shall be willing to answer you. And take my answer in brief, thus: First, where the grace of God is in the heart, it causeth there a great outcry against sin. Secondly-

Faithful: Nay, hold; let us consider of one at once. I think you should rather say, it shows itself by inclining the soul to abhor its sin.

Talkative: Why, what difference is there between crying out against, and abhorring of sin?

Faithful: Oh! a great deal. A man may cry out against sin, of policy; but he cannot abhor it but by virtue of a godly antipathy against it. I have heard many cry out against sin in the pulpit, who yet can abide it well enough in the heart, house, and conversation. Gen. 39:15. Joseph's mistress cried out with a loud voice, as if she had been very holy; but she would willingly, notwithstanding that, have committed uncleanness with him. Some cry out against sin, even as the mother cries out against her child in her lap, when she calleth it slut and naughty girl, and then falls to hugging and kissing it.

Talkative: You lie at the catch, I perceive.

Faithful: No, not I; I am only for setting things right. But what is the second thing whereby you would prove a discovery of a work of grace in the heart?

Talkative: Great knowledge of gospel mysteries.

Faithful: This sign should have been first: but, first or last, it is also false; for knowledge, great knowledge, may be obtained in the mysteries of the Gospel, and yet no work of grace in the soul. Yea, if a man have all knowledge, he may yet be nothing, and so, consequently, be no child of God. 1 Cor. 13:2. When Christ said, ``Do you know all these things?" and the disciples answered, Yes, he added, ``Blessed are ye if ye do them." He doth not lay the blessing in the knowing of them, but in the doing of them. For there is a knowledge that is not attended with doing: ``He that knoweth his Master's will, and doeth it not." A man may know like an angel, and yet be no Christian: therefore your sign of it is not true. Indeed, to know is a thing that pleaseth talkers and boasters; but to do is that which pleaseth God. Not that the heart can be good without knowledge, for without that the heart is naught. There are, therefore, two sorts of knowledge, knowledge that resteth in the bare speculation of things, and knowledge that is accompanied with the grace of faith and love, which puts a man upon doing even the will of God from the heart: the first of these will serve the talker; but without the other, the true Christian is not content. ``Give me understanding, and I shall keep thy law; yea, I shall observe it with my whole heart." Psa. 119:34.

Talkative: You lie at the catch again: this is not for edification.

Faithful: Well, if you please, propound another sign how this work of grace discovereth itself where it is.

Talkative: Not I, for I see we shall not agree.

Faithful: Well, if you will not, will you give me leave to do it?

Talkative: You may use your liberty.

Faithful: A work of grace in the soul discovereth itself, either to him that hath it, or to standers-by.

To him that hath it, thus: It gives him conviction of sin, especially the defilement of his nature, and the sin of unbelief, for the sake of which he is sure to be damned, if he findeth not mercy at God's hand, by faith in Jesus Christ. This sight and sense of things worketh in him sorrow and shame for sin. Psa. 38:18; Jer. 31:19; John 16:8; Rom. 7:24; Mark 16:16; Gal. 2:16; Rev. 1:6. He findeth, moreover, revealed in him the Saviour of the world, and the absolute necessity of closing with him for life; at the which he findeth hungerings and thirstings after him; to which hungerings, etc., the promise is made. Now, according to the strength or weakness of his faith in his Saviour, so is his joy and peace, so is his love to holiness, so are his desires to know him more, and also to serve him in this world. But though, I say, it discovereth itself thus unto him, yet it is but seldom that he is able to conclude that this is a work of grace; because his corruptions now, and his abused reason, make his mind to misjudge in this matter: therefore in him that hath this work there is required a very sound judgment, before he can with steadiness conclude that this is a work of grace. John 16:9; Gal. 2:15,16; Acts 4:12; Matt. 5:6; Rev. 21:6.

To others it is thus discovered:

1. By an experimental confession of his faith in Christ. 2. By a life answerable to that confession; to wit, a life of holiness-heart-holiness, family-holiness, (if he hath a family,) and by conversation-holiness in the world; which in the general teacheth him inwardly to abhor his sin, and himself for that, in secret; to suppress it in his family, and to promote holiness in the world: not by talk only, as a hypocrite or talkative person may do, but by a practical subjection in faith and love to the power of the word. Job 42:5,6; Psa. 50:23; Ezek. 20:43; Matt. 5:8; John 14:15; Rom. 10:10; Ezek. 36:25; Phil. 1:27; 3:17-20. And now, sir, as to this brief description of the work of grace, and also the discovery of it, if you have aught to object, object; if not, then give me leave to propound to you a second question.

Talkative: Nay, my part is not now to object, but to hear; let me, therefore, have your second question.

Faithful: It is this: Do you experience this first part of the description of it; and doth your life and conversation testify the same? Or standeth your religion in word or tongue, and not in deed and truth? Pray, if you incline to answer me in this, say no more than you know the God above will say Amen to, and also nothing but what your conscience can justify you in; for not he that commendeth himself is approved, but whom the Lord commendeth. Besides, to say I am thus and thus, when my conversation, and all my neighbors, tell me I lie, is great wickedness.

Then Talkative at first began to blush; but, recovering himself, thus he replied: You come now to experience, to conscience, and to God; and to appeal to him for justification of what is spoken. This kind of discourse I did not expect; nor am I disposed to give an answer to such questions, because I count not myself bound thereto, unless you take upon you to be a catechiser; and though you should so do, yet I may refuse to make you my judge. But I pray, will you tell me why you ask me such questions?

Faithful: Because I saw you forward to talk, and because I knew not that you had aught else but notion. Besides, to tell you all the truth, I have heard of you that you are a man whose religion lies in talk, and that your conversation gives this your mouth-profession the lie. They say you are a spot among Christians, and that religion fareth the worse for your ungodly conversation; that some have already stumbled at your wicked ways, and that more are in danger of being destroyed thereby: your religion, and an ale-house, and covetousness, and uncleanness, and swearing, and lying, and vain company-keeping, etc., will stand together. The proverb is true of you which is said of a harlot, to wit, ``That she is a shame to all women:" so are you a shame to all professors.

Talkative: Since you are so ready to take up reports, and to judge so rashly as you do, I cannot but conclude you are some peevish or melancholy man, not fit to be discoursed with; and so adieu.

Then up came Christian, and said to his brother, I told you how it would happen; your words and his lusts could not agree. He had rather leave your company than reform his life. But he is gone, as I said: let him go; the loss is no man's but his own. He has saved us the trouble of going from him; for he continuing (as I suppose he will do) as he is, would have been but a blot in our company: besides, the apostle says, ``From such withdraw thyself."

Faithful: But I am glad we had this little discourse with him; it may happen that he will think of it again: however, I have dealt plainly with him, and so am clear of his blood if he perisheth.

Christian: You did well to talk so plainly to him as you did. There is but little of this faithful dealing with men now-a-days, and that makes religion to stink so in the nostrils of many as it doth; for they are these talkative fools, whose religion is only in word, and who are debauched and vain in their conversation, that (being so much admitted into the fellowship of the godly) do puzzle the world, blemish Christianity, and grieve the sincere. I wish that all men would deal with such as you have done; then should they either be made more conformable to religion, or the company of saints would be too hot for them. Then did Faithful say,
\begin{verse} 
``How Talkative at first lifts up his plumes!\\
How bravely doth he speak! How he presumes\\
To drive down all before him! But so soon\\
As Faithful talks of heart-work, like the moon\\
That's past the full, into the wane he goes;\\
And so will all but he that heart-work know."\\
\end{verse}
Thus they went on, talking of what they had seen by the way, and so made that way easy, which would otherwise no doubt have been tedious to them, for now they went through a wilderness. 
