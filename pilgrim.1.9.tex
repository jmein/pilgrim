\chapter[THE NINTH STAGE]{}

And I slept, and dreamed again, and saw the same two pilgrims going down the mountains along the highway towards the city. Now, a little below these mountains, on the left hand, lieth the country of Conceit, from which country there comes into the way in which the pilgrims walked, a little crooked lane. Here, therefore, they met with a very brisk lad that came out of that country, and his name was Ignorance. So Christian asked him from what parts he came, and whither he was going.

Ignorance: Sir, I was born in the country that lieth off there, a little on the left hand, and I am going to the Celestial City.

Christian: But how do you think to get in at the gate, for you may find some difficulty there?

Ignorance: As other good people do, said he.

Christian: But what have you to show at that gate, that the gate should be opened to you?

Ignorance: I know my Lord's will, and have been a good liver; I pay every man his own; I pray, fast, pay tithes, and give alms, and have left my country for whither I am going.

Christian: But thou camest not in at the wicket-gate, that is at the head of this way; thou camest in hither through that same crooked lane, and therefore I fear, however thou mayest think of thyself, when the reckoning-day shall come, thou wilt have laid to thy charge, that thou art a thief and a robber, instead of getting admittance into the city.

Ignorance: Gentlemen, ye be utter strangers to me; I know you not: be content to follow the religion of your country, and I will follow the religion of mine. I hope all will be well. And as for the gate that you talk of, all the world knows that is a great way off of our country. I cannot think that any man in all our parts doth so much as know the way to it; nor need they matter whether they do or no, since we have, as you see, a fine, pleasant, green lane, that comes down from our country, the next way into the way.

When Christian saw that the man was wise in his own conceit, he said to Hopeful whisperingly, ``There is more hope of a fool than of him." Prov. 26:12. And said, moreover, ``When he that is a fool walketh by the way, his wisdom faileth him, and he saith to every one that he is a fool. Eccles. 10:3. What, shall we talk farther with him, or outgo him at present, and so leave him to think of what he hath heard already, and then stop again for him afterwards, and see if by degrees we can do any good to him? Then said Hopeful,
\begin{verse}
``Let Ignorance a little while now muse\\
On what is said, and let him not refuse\\
Good counsel to embrace, lest he remain\\
Still ignorant of what's the chiefest gain.\\
God saith, those that no understanding have,\\
(Although he made them,) them he will not save."\\
\end{verse}
Hopeful: He further added, It is not good, I think, to say so to him all at once; let us pass him by, if you will, and talk to him anon, even as he is able to bear it.

So they both went on, and Ignorance he came after. Now, when they had passed him a little way, they entered into a very dark lane, where they met a man whom seven devils had bound with seven strong cords, and were carrying him back to the door that they saw on the side of the hill. Matt. 12:45; Prov. 5:22. Now good Christian began to tremble, and so did Hopeful, his companion; yet, as the devils led away the man, Christian looked to see if he knew him; and he thought it might be one Turn-away, that dwelt in the town of Apostacy. But he did not perfectly see his face, for he did hang his head like a thief that is found; but being gone past, Hopeful looked after him, and espied on his back a paper with this inscription, ``Wanton professor, and damnable apostate."

Then said Christian to his fellow, Now I call to remembrance that which was told me of a thing that happened to a good man hereabout. The name of the man was Little-Faith; but a good man, and he dwelt in the town of Sincere. The thing was this. At the entering in at this passage, there comes down from Broadway-gate, a lane, called Dead-Man's lane; so called because of the murders that are commonly done there; and this Little-Faith going on pilgrimage, as we do now, chanced to sit down there and sleep. Now there happened at that time to come down the lane from Broadway-gate, three sturdy rogues, and their names were Faint-Heart, Mistrust, and Guilt, three brothers; and they, espying Little-Faith where he was, came galloping up with speed. Now the good man was just awaked from his sleep, and was getting up to go on his journey. So they came up all to him, and with threatening language bid him stand. At this, Little-Faith looked as white as a sheet, and had neither power to fight nor fly. Then said Faint-Heart, Deliver thy purse; but he making no haste to do it, (for he was loth to lose his money,) Mistrust ran up to him, and thrusting his hand into his pocket, pulled out thence a bag of silver. Then he cried out, Thieves, thieves! With that, Guilt, with a great club that was in his hand, struck Little-Faith on the head, and with that blow felled him flat to the ground, where he lay bleeding as one that would bleed to death. All this while the thieves stood by. But at last, they hearing that some were upon the road, and fearing lest it should be one Great-Grace, that dwells in the town of Good-Confidence, they betook themselves to their heels, and left this good man to shift for himself. Now, after a while, Little-Faith came to himself, and getting up, made shift to scramble on his way. This was the story.

Hopeful: But did they take from him all that ever he had?

Christian: No; the place where his jewels were they never ransacked; so those he kept still. But, as I was told, the good man was much afflicted for his loss; for the thieves got most of his spending-money. That which they got not, as I said, were jewels; also, he had a little odd money left, but scarce enough to bring him to his journey's end. Nay, (if I was not misinformed,) he was forced to beg as he went, to keep himself alive, for his jewels he might not sell; but beg and do what he could, he went, as we say, with many a hungry belly the most part of the rest of the way. 1 Pet. 4:18.

Hopeful: But is it not a wonder they got not from him his certificate, by which he was to receive his admittance at the Celestial Gate?

Christian: It is a wonder; but they got not that, though they missed it not through any good cunning of his; for he, being dismayed by their coming upon him, had neither power nor skill to hide any thing; so it was more by good providence than by his endeavor that they missed of that good thing. 2 Tim. 1:12-14; 2 Pet. 2:9.

Hopeful: But it must needs be a comfort to him they got not this jewel from him.

Christian: It might have been great comfort to him, had he used it as he should; but they that told me the story said that he made but little use of it all the rest of the way, and that because of the dismay that he had in their taking away his money. Indeed, he forgot it a great part of the rest of his journey; and besides, when at any time it came into his mind, and he began to be comforted therewith, then would fresh thoughts of his loss come again upon him, and these thoughts would swallow up all.

Hopeful: Alas, poor man, this could not but be a great grief to him.

Christian: Grief? Aye, a grief indeed! Would it not have been so to any of us, had we been used as he, to be robbed and wounded too, and that in a strange place, as he was? It is a wonder he did not die with grief, poor heart. I was told that he scattered almost all the rest of the way with nothing but doleful and bitter complaints; telling, also, to all that overtook him, or that he overtook in the way as he went, where he was robbed, and how; who they were that did it, and what he had lost; how he was wounded, and that he hardly escaped with life.

Hopeful: But it is a wonder that his necessity did not put him upon selling or pawning some of his jewels, that he might have wherewith to relieve himself in his journey.

Christian: Thou talkest like one upon whose head is the shell to this very day. For what should he pawn them? or to whom should he sell them? In all that country where he was robbed, his jewels were not accounted of; nor did he want that relief which could from thence be administered to him. Besides, had his jewels been missing at the gate of the Celestial City, he had (and that he knew well enough) been excluded from an inheritance there, and that would have been worse to him than the appearance and villany of ten thousand thieves.

Hopeful: Why art thou so tart, my brother? Esau sold his birthright, and that for a mess of pottage, Heb. 12:16; and that birthright was his greatest jewel: and if he, why might not Little-Faith do so too?

Christian: Esau did sell his birthright indeed, and so do many besides, and by so doing exclude themselves from the chief blessing, as also that caitiff did; but you must put a difference betwixt Esau and Little-Faith, and also betwixt their estates. Esau's birthright was typical; but Little-Faith's jewels were not so. Esau's belly was his god; but Little-Faith's belly was not so. Esau's want lay in his fleshy appetite; Little-Faith's did not so. Besides, Esau could see no further than to the fulfilling of his lusts: For I am at the point to die, said he: and what good will this birthright do me? Gen. 25:32. But Little-Faith, though it was his lot to have but a little faith, was by his little faith kept from such extravagances, and made to see and prize his jewels more than to sell them, as Esau did his birthright. You read not any where that Esau had faith, no, not so much as a little; therefore no marvel, where the flesh only bears sway, (as it will in that man where no faith is to resist,) if he sells his birthright and his soul and all, and that to the devil of hell; for it is with such as it is with the ass, who in her occasion cannot be turned away, Jer. 2:24: when their minds are set upon their lusts, they will have them, whatever they cost. But Little-Faith was of another temper; his mind was on things divine; his livelihood was upon things that were spiritual, and from above: therefore, to what end should he that is of such a temper sell his jewels (had there been any that would have bought them) to fill his mind with empty things? Will a man give a penny to fill his belly with hay? or can you persuade the turtle-dove to live upon carrion, like the crow? Though faithless ones can, for carnal lusts, pawn, or mortgage, or sell what they have, and themselves outright to boot; yet they that have faith, saving faith, though but a little of it, cannot do so. Here, therefore, my brother, is thy mistake.

Hopeful: I acknowledge it; but yet your severe reflection had almost made me angry.

Christian: Why, I did but compare thee to some of the birds that are of the brisker sort, who will run to and fro in untrodden paths with the shell upon their heads: but pass by that, and consider the matter under debate, and all shall be well betwixt thee and me.

Hopeful: But, Christian, these three fellows, I am persuaded in my heart, are but a company of cowards: would they have run else, think you, as they did, at the noise of one that was coming on the road? Why did not Little-Faith pluck up a greater heart? He might, methinks, have stood one brush with them, and have yielded when there had been no remedy.

Christian: That they are cowards, many have said, but few have found it so in the time of trial. As for a great heart, Little-Faith had none; and I perceive by thee, my brother, hadst thou been the man concerned, thou art but for a brush, and then to yield. And verily, since this is the height of thy stomach now they are at a distance from us, should they appear to thee as they did to him, they might put thee to second thoughts.

But consider again, that they are but journeymen thieves; They serve under the king of the bottomless pit, who, if need be, will come to their aid himself, and his voice is as the roaring of a lion. 1 Pet. 5:8. I myself have been engaged as this Little-Faith was, and I found it a terrible thing. These three villains set upon me, and I beginning like a Christian to resist, they gave but a call, and in came their master. I would, as the saying is, have given my life for a penny, but that, as God would have it, I was clothed with armor of proof. Aye, and yet, though I was so harnessed, I found it hard work to quit myself like a man: no man can tell what in that combat attends us, but he that hath been in the battle himself.

Hopeful: Well, but they ran, you see, when they did but suppose that one Great-Grace was in the way.

Christian: True, they have often fled, both they and their master, when Great-Grace hath but appeared; and no marvel, for he is the King's champion. But I trow you will put some difference between Little-Faith and the King's champion. All the King's subjects are not his champions; nor can they, when tried, do such feats of war as he. Is it meet to think that a little child should handle Goliath as David did? or that there should be the strength of an ox in a wren? Some are strong, some are weak; some have great faith, some have little: this man was one of the weak, and therefore he went to the wall.

Hopeful: I would it had been Great-Grace, for their sakes.

Christian: If it had been he, he might have had his hands full: for I must tell you, that though Great-Grace is excellent good at his weapons, and has, and can, so long as he keeps them at sword's point, do well enough with them; yet if they get within him, even Faint-Heart, Mistrust, or the other, it shall go hard but they will throw up his heels. And when a man is down, you know, what can he do?

Whoso looks well upon Great-Grace's face, will see those scars and cuts there that shall easily give demonstration of what I say. Yea, once I heard that he should say, (and that when he was in the combat,) We despaired even of life. How did these sturdy rogues and their fellows make David groan, mourn, and roar! Yea, Heman, Psa. 88, and Hezekiah too, though champions in their days, were forced to bestir them when by these assaulted; and yet, notwithstanding, they had their coats soundly brushed by them. Peter, upon a time, would go try what he could do; but though some do say of him that he is the prince of the apostles, they handled him so that they made him at last afraid of a sorry girl.

Besides, their king is at their whistle; he is never out of hearing; and if at any time they be put to the worst, he, if possible, comes in to help them; and of him it is said, ``The sword of him that layeth at him cannot hold; the spear, the dart, nor the habergeon. He esteemeth iron as straw, and brass as rotten wood. The arrow cannot make him fly; sling-stones are turned with him into stubble. Darts are counted as stubble; he laugheth at the shaking of a spear." Job 41:26-29. What can a man do in this case? It is true, if a man could at every turn have Job's horse, and had skill and courage to ride him, he might do notable things. ``For his neck is clothed with thunder. He will not be afraid as a grasshopper: the glory of his nostrils is terrible. He paweth in the valley, and rejoiceth in his strength; he goeth on to meet the armed men. He mocketh at fear, and is not affrighted; neither turneth he back from the sword. The quiver rattleth against him, the glittering spear and the shield. He swalloweth the ground with fierceness and rage; neither believeth he that it is the sound of the trumpet. He saith among the trumpets, Ha, ha! and he smelleth the battle afar off, the thunder of the captains, and the shoutings." Job 39:19-25.

But for such footmen as thee and I are, let us never desire to meet with an enemy, nor vaunt as if we could do better, when we hear of others that have been foiled, nor be tickled at the thoughts of our own manhood; for such commonly come by the worst when tried. Witness Peter, of whom I made mention before: he would swagger, aye, he would; he would, as his vain mind prompted him to say, do better and stand more for his Master than all men: but who so foiled and run down by those villains as he?

When, therefore, we hear that such robberies are done on the King's highway, two things become us to do.

1. To go out harnessed, and be sure to take a shield with us: for it was for want of that, that he who laid so lustily at Leviathan could not make him yield; for, indeed, if that be wanting, he fears us not at all. Therefore, he that had skill hath said, ``Above all, take the shield of faith, wherewith ye shall be able to quench all the fiery darts of the wicked." Eph. 6:16.

2. It is good, also, that we desire of the King a convoy, yea, that he will go with us himself. This made David rejoice when in the Valley of the Shadow of Death; and Moses was rather for dying where he stood, than to go one step without his God. Exod. 33:15.

O, my brother, if he will but go along with us, what need we be afraid of ten thousands that shall set themselves against us? Psa. 3:5-8; 27:1-3. But without him, the proud helpers fall under the slain. Isa. 10:4.

I, for my part, have been in the fray before now; and though (through the goodness of Him that is best) I am, as you see, alive, yet I cannot boast of any manhood. Glad shall I be if I meet with no more such brunts; though I fear we are not got beyond all danger. However, since the lion and the bear have not as yet devoured me, I hope God will also deliver us from the next uncircumcised Philistine. Then sang Christian,
\begin{verse}
``Poor Little-Faith! hast been among the thieves?\\
Wast robb'd? Remember this, whoso believes,\\
And get more faith; then shall you victors be\\
Over ten thousand-else scarce over three."\\
\end{verse}
So they went on, and Ignorance followed. They went then till they came at a place where they saw a way put itself into their way, and seemed withal to lie as strait as the way which they should go; and here they knew not which of the two to take, for both seemed strait before them: therefore here they stood still to consider. And as they were thinking about the way, behold a man black of flesh, but covered with a very light robe, come to them, and asked them why they stood there. They answered, they were going to the Celestial City, but knew not which of these ways to take. ``Follow me," said the man, ``it is thither that I am going." So they followed him in the way that but now came into the road, which by degrees turned, and turned them so far from the city that they desired to go to, that in a little time their faces were turned away from it; yet they follow him. But by and by, before they were aware, he led them both within the compass of a net, in which they were both so entangled that they knew not what to do; and with that the white robe fell off the black man's back. Then they saw where they were. Wherefore there they lay crying some time, for they could not get themselves out.

Christian: Then said Christian to his fellow, Now do I see myself in an error. Did not the shepherds bid us beware of the Flatterer? As is the saying of the wise man, so we have found it this day: ``A man that flattereth his neighbor, spreadeth a net for his feet." Prov. 29:5.

Hopeful: They also gave us a note of directions about the way, for our more sure finding thereof; but therein we have also forgotten to read, and have not kept ourselves from the paths of the destroyer. Here David was wiser than we; for saith he, ``Concerning the works of men, by the word of thy lips I have kept me from the paths of the Destroyer." Psa. 17:4. Thus they lay bewailing themselves in the net. At last they espied a Shining One coming towards them with a whip of small cords in his hand. When he was come to the place where they were, he asked them whence they came, and what they did there. They told him that they were poor pilgrims going to Zion, but were led out of their way by a black man clothed in white, who bid us, said they, follow him, for he was going thither too. Then said he with the whip, It is Flatterer, a false apostle, that hath transformed himself into an angel of light. Dan. 11:32; 2 Cor. 11:13,14. So he rent the net, and let the men out. Then said he to them, Follow me, that I may set you in your way again. So he led them back to the way which they had left to follow the Flatterer. Then he asked them, saying, Where did you lie the last night? They said, With the shepherds upon the Delectable Mountains. He asked them then if they had not of the shepherds a note of direction for the way. They answered, Yes. But did you not, said he, when you were at a stand, pluck out and read your note? They answered, No. He asked them, Why? They said they forgot. He asked, moreover, if the shepherds did not bid them beware of the Flatterer. They answered, Yes; but we did not imagine, said they, that this fine-spoken man had been he. Rom. 16:17,18.

Then I saw in my dream, that he commanded them to lie down; which when they did, he chastised them sore, to teach them the good way wherein they should walk, Deut. 25:2; 2 Chron. 6:27; and as he chastised them, he said, ``As many as I love, I rebuke and chasten; be zealous, therefore, and repent." Rev. 3:19. This done, he bids them to go on their way, and take good heed to the other directions of the shepherds. So they thanked him for all his kindness, and went softly along the right way, singing,
\begin{verse}
``Come hither, you that walk along the way,\\
See how the pilgrims fare that go astray:\\
They catched are in an entangling net,\\
Cause they good counsel lightly did forget:\\
'Tis true, they rescued were; but yet, you see,\\
They're scouged to boot; let this your caution be."\\
\end{verse}
Now, after awhile, they perceived afar off, one coming softly, and alone, all along the highway, to meet them. Then said Christian to his fellow, Yonder is a man with his back towards Zion, and he is coming to meet us.

Hopeful: I see him; let us take heed to ourselves now, lest he should prove a Flatterer also. So he drew nearer and nearer, and at last came up to them. His name was Atheist, and he asked them whither they were going.

Christian: We are going to Mount Zion.

Then Atheist fell into a very great laughter.

Christian: What's the meaning of your laughter?

Atheist: I laugh to see what ignorant persons you are, to take upon you so tedious a journey, and yet are like to have nothing but your travel for your pains.

Christian: Why, man, do you think we shall not be received?

Atheist: Received! There is not such a place as you dream of in all this world.

Christian: But there is in the world to come.

Atheist: When I was at home in mine own country I heard as you now affirm, and from that hearing went out to see, and have been seeking this city these twenty years, but find no more of it than I did the first day I set out. Eccles. 10:15; Jer. 17:15.

Christian: We have both heard, and believe, that there is such a place to be found.

Atheist: Had not I, when at home, believed, I had not come thus far to seek; but finding none, (and yet I should, had there been such a place to be found, for I have gone to seek it farther than you,) I am going back again, and will seek to refresh myself with the things that I then cast away for hopes of that which I now see is not.

Christian: Then said Christian to Hopeful his companion, Is it true which this man hath said?

Hopeful: Take heed, he is one of the Flatterers. Remember what it cost us once already for our hearkening to such kind of fellows. What! no Mount Zion? Did we not see from the Delectable Mountains the gate of the city? Also, are we not now to walk by faith? 2 Cor. 5:7.

Let us go on, lest the man with the whip overtake us again. You should have taught me that lesson, which I will sound you in the ears withal: ``Cease, my son, to hear the instruction that causeth to err from the words of knowledge." Prov. 19:27. I say, my brother, cease to hear him, and let us believe to the saving of the soul.

Christian: My brother, I did not put the question to thee, for that I doubted of the truth of our belief myself, but to prove thee, and to fetch from thee a fruit of the honesty of thy heart. As for this man, I know that he is blinded by the God of this world. Let thee and me go on, knowing that we have belief of the truth; and no lie is of the truth. 1 John, 5:21.

Hopeful: Now do I rejoice in hope of the glory of God. So they turned away from the man; and he, laughing at them, went his way.

I then saw in my dream, that they went on until they came into a certain country whose air naturally tended to make one drowsy, if he came a stranger into it. And here Hopeful began to be very dull, and heavy to sleep: wherefore he said unto Christian, I do now begin to grow so drowsy that I can scarcely hold open mine eyes; let us lie down here, and take one nap.

Christian: By no means, said the other; lest, sleeping, we never awake more.

Hopeful: Why, my brother? sleep is sweet to the laboring man; we may be refreshed, if we take a nap.

Christian: Do you not remember that one of the shepherds bid us beware of the Enchanted Ground? He meant by that, that we should beware of sleeping; wherefore ``let us not sleep, as do others; but let us watch and be sober." 1 Thess. 5:6.

Hopeful: I acknowledge myself in a fault; and had I been here alone, I had by sleeping run the danger of death. I see it is true that the wise man saith, ``Two are better than one." Eccl. 4:9. Hitherto hath thy company been my mercy; and thou shalt have a good reward for thy labor.

Christian: Now, then, said Christian, to prevent drowsiness in this place, let us fall into good discourse.

Hopeful: With all my heart, said the other.

Christian: Where shall we begin?

Hopeful: Where God began with us. But do you begin, if you please.

Christian: I will sing you first this song:
\begin{verse}
``When saints do sleepy grow, let them come hither,\\
And hear how these two pilgrims talk together;\\
Yea, let them learn of them in any wise,\\
Thus to keep ope their drowsy, slumb'ring eyes.\\
Saints' fellowship, if it be managed well,\\
Keeps them awake, and that in spite of hell."\\
\end{verse}
Then Christian began, and said, I will ask you a question. How came you to think at first of doing what you do now?

Hopeful: Do you mean, how came I at first to look after the good of my soul?

Christian: Yes, that is my meaning.

Hopeful: I continued a great while in the delight of those things which were seen and sold at our fair; things which I believe now would have, had I continued in them still, drowned me in perdition and destruction.

Christian: What things were they?

Hopeful: All the treasures and riches of the world. Also I delighted much in rioting, reveling, drinking, swearing, lying, uncleanness, Sabbath-breaking, and what not, that tended to destroy the soul. But I found at last, by hearing and considering of things that are divine, which, indeed, I heard of you, as also of beloved Faithful, that was put to death for his faith and good living in Vanity Fair, that the end of these things is death, Rom. 6:21-23; and that for these things' sake, the wrath of God cometh upon the children of disobedience. Eph. 5:6.

Christian: And did you presently fall under the power of this conviction?

Hopeful: No, I was not willing presently to know the evil of sin, nor the damnation that follows upon the commission of it; but endeavored, when my mind at first began to be shaken with the word, to shut mine eyes against the light thereof.

Christian: But what was the cause of your carrying of it thus to the first workings of God's blessed Spirit upon you?

Hopeful: The causes were, 1. I was ignorant that this was the work of God upon me. I never thought that by awakenings for sin, God at first begins the conversion of a sinner. 2. Sin was yet very sweet to my flesh, and I was loth to leave it. 3. I could not tell how to part with mine old companions, their presence and actions were so desirable unto me. 4. The hours in which convictions were upon me, were such troublesome and such heart-affrighting hours, that I could not bear, no not so much as the remembrance of them upon my heart.

Christian: Then, as it seems, sometimes you got rid of your trouble?

Hopeful: Yes, verily, but it would come into my mind again; and then I should be as bad, nay, worse than I was before.

Christian: Why, what was it that brought your sins to mind again?

Hopeful: Many things; as,

1. If I did but meet a good man in the streets; or,

2. If I have heard any read in the Bible; or,

3. If mine head did begin to ache; or,

4. If I were told that some of my neighbors were sick; or,

5. If I heard the bell toll for some that were dead; or,

6. If I thought of dying myself; or,

7. If I heard that sudden death happened to others.

8. But especially when I thought of myself, that I must quickly come to judgment.

Christian: And could you at any time, with ease, get off the guilt of sin, when by any of these ways it came upon you?

Hopeful: No, not I; for then they got faster hold of my conscience; and then, if I did but think of going back to sin, (though my mind was turned against it,) it would be double torment to me.

Christian: And how did you do then?

Hopeful: I thought I must endeavor to mend my life; for else, thought I, I am sure to be damned.

Christian: And did you endeavor to mend?

Hopeful: Yes, and fled from, not only my sins, but sinful company too, and betook me to religious duties, as praying, reading, weeping for sin, speaking truth to my neighbors, etc. These things did I, with many others, too much here to relate.

Christian: And did you think yourself well then?

Hopeful: Yes, for a while; but at the last my trouble came tumbling upon me again, and that over the neck of all my reformations.

Christian: How came that about, since you were now reformed?

Hopeful: There were several things brought it upon me, especially such sayings as these: ``All our righteousnesses are as filthy rags." Isa. 64:6. ``By the works of the law shall no flesh be justified." Gal. 2:16. ``When ye have done all these things, say, We are unprofitable," Luke 17:10; with many more such like. From whence I began to reason with myself thus: If all my righteousnesses are as filthy rags; if by the deeds of the law no man can be justified; and if, when we have done all, we are yet unprofitable, then is it but a folly to think of heaven by the law. I farther thought thus: If a man runs a hundred pounds into the shopkeeper's debt, and after that shall pay for all that he shall fetch; yet if his old debt stands still in the book uncrossed, the shopkeeper may sue him for it, and cast him into prison, till he shall pay the debt.

Christian: Well, and how did you apply this to yourself?

Hopeful: Why, I thought thus with myself: I have by my sins run a great way into God's book, and my now reforming will not pay off that score; therefore I should think still, under all my present amendments, But how shall I be freed from that damnation that I brought myself in danger of by my former transgressions?

Christian: A very good application: but pray go on.

Hopeful: Another thing that hath troubled me ever since my late amendments, is, that if I look narrowly into the best of what I do now, I still see sin, new sin, mixing itself with the best of that I do; so that now I am forced to conclude, that notwithstanding my former fond conceits of myself and duties, I have committed sin enough in one day to send me to hell, though my former life had been faultless.

Christian: And what did you do then?

Hopeful: Do! I could not tell what to do, until I broke my mind to Faithful; for he and I were well acquainted. And he told me, that unless I could obtain the righteousness of a man that never had sinned, neither mine own, nor all the righteousness of the world, could save me.

Christian: And did you think he spake true?

Hopeful: Had he told me so when I was pleased and satisfied with my own amendments, I had called him fool for his pains; but now, since I see my own infirmity, and the sin which cleaves to my best performance, I have been forced to be of his opinion.

Christian: But did you think, when at first he suggested it to you, that there was such a man to be found, of whom it might justly be said, that he never committed sin?

Hopeful: I must confess the words at first sounded strangely; but after a little more talk and company with him, I had full conviction about it.

Christian: And did you ask him what man this was, and how you must be justified by him?

Hopeful: Yes, and he told me it was the Lord Jesus, that dwelleth on the right hand of the Most High. Heb. 10:12-21. And thus, said he, you must be justified by him, even by trusting to what he hath done by himself in the days of his flesh, and suffered when he did hang on the tree. Rom. 4:5; Col. 1:14; 1 Pet. 1:19. I asked him further, how that man's righteousness could be of that efficacy, to justify another before God. And he told me he was the mighty God, and did what he did, and died the death also, not for himself, but for me; to whom his doings, and the worthiness of them, should be imputed, if I believed on him.

Christian: And what did you do then?

Hopeful: I made my objections against my believing, for that I thought he was not willing to save me.

Christian: And what said Faithful to you then?

Hopeful: He bid me go to him and see. Then I said it was presumption. He said, No; for I was invited to come. Matt. 11:28. Then he gave me a book of Jesus' inditing, to encourage me the more freely to come; and he said concerning that book, that every jot and tittle thereof stood firmer than heaven and earth. Matt. 24:35. Then I asked him what I must do when I came; and he told me I must entreat upon my knees, Psa. 95:6; Dan. 6:10, with all my heart and soul, Jer. 29:12,13, the Father to reveal him to me. Then I asked him further, how I must make my supplications to him; and he said, Go, and thou shalt find him upon a mercy-seat, where he sits all the year long to give pardon and forgiveness to them that come. Exod. 25:22; Lev. 16:2; Num. 7:89; Heb. 4:16. I told him, that I knew not what to say when I came; and he bid say to this effect: God be merciful to me a sinner, and make me to know and believe in Jesus Christ; for I see, that if his righteousness had not been, or I have not faith in that righteousness, I am utterly cast away. Lord, I have heard that thou art a merciful God, and hast ordained that thy Son Jesus Christ should be the Saviour of the world; and moreover, that thou art willing to bestow him upon such a poor sinner as I am-and I am a sinner indeed. Lord, take therefore this opportunity, and magnify thy grace in the salvation of my soul, through thy Son Jesus Christ. Amen.

Christian: And did you do as you were bidden?

Hopeful: Yes, over, and over, and over.

Christian: And did the Father reveal the Son to you?

Hopeful: Not at the first, nor second, nor third, nor fourth, nor fifth, no, nor at the sixth time neither.

Christian: What did you do then?

Hopeful: What? why I could not tell what to do.

Christian: Had you not thoughts of leaving off praying?

Hopeful: Yes; an hundred times twice told.

Christian: And what was the reason you did not?

Hopeful: I believed that it was true which hath been told me, to wit, that without the righteousness of this Christ, all the world could not save me; and therefore, thought I with myself, if I leave off, I die, and I can but die at the throne of grace. And withal this came into my mind, ``If it tarry, wait for it; because it will surely come, and will not tarry." Hab. 2:3. So I continued praying until the Father showed me his Son.

Christian: And how was he revealed unto you?

Hopeful: I did not see him with my bodily eyes, but with the eyes of my understanding, Eph. 1:18,19; and thus it was. One day I was very sad, I think sadder than at any one time in my life; and this sadness was through a fresh sight of the greatness and vileness of my sins. And as I was then looking for nothing but hell, and the everlasting damnation of my soul, suddenly, as I thought, I saw the Lord Jesus looking down from heaven upon me, and saying, ``Believe on the Lord Jesus Christ, and thou shalt be saved." Acts 16:31.

But I replied, Lord, I am a great, a very great sinner: and he answered, ``My grace is sufficient for thee." 2 Cor. 12:9. Then I said, But, Lord, what is believing? And then I saw from that saying, ``He that cometh to me shall never hunger, and he that believeth on me shall never thirst," John 6:35, that believing and coming was all one; and that he that came, that is, that ran out in his heart and affections after salvation by Christ, he indeed believed in Christ. Then the water stood in mine eyes, and I asked further, But, Lord, may such a great sinner as I am be indeed accepted of thee, and be saved by thee? And I heard him say, ``And him that cometh to me, I will in no wise cast out." John 6:37. Then I said, But how, Lord, must I consider of thee in my coming to thee, that my faith may be placed aright upon thee? Then he said, ``Christ Jesus came into the world to save sinners." 1 Tim. 1:15. He is the end of the law for righteousness to every one that believes. Rom.10:4, and chap. 4. He died for our sins, and rose again for our justification. Rom. 4:25. He loved us, and washed us from our sins in his own blood. Rev. 1:5. He is the Mediator between God and us. 1 Tim. 2:5. He ever liveth to make intercession for us. Heb. 7:25. From all which I gathered, that I must look for righteousness in his person, and for satisfaction for my sins by his blood: that what he did in obedience to his Father's law, and in submitting to the penalty thereof, was not for himself, but for him that will accept it for his salvation, and be thankful. And now was my heart full of joy, mine eyes full of tears, and mine affections running over with love to the name, people, and ways of Jesus Christ.

Christian: This was a revelation of Christ to your soul indeed. But tell me particularly what effect this had upon your spirit.

Hopeful: It made me see that all the world, notwithstanding all the righteousness thereof, is in a state of condemnation. It made me see that God the Father, though he be just, can justly justify the coming sinner. It made me greatly ashamed of the vileness of my former life, and confounded me with the sense of mine own ignorance; for there never came a thought into my heart before now that showed me so the beauty of Jesus Christ. It made me love a holy life, and long to do something for the honor and glory of the name of the Lord Jesus. Yea, I thought that had I now a thousand gallons of blood in my body, I could spill it all for the sake of the Lord Jesus.

I saw then in my dream, that Hopeful looked back, and saw Ignorance, whom they had left behind, coming after. Look, said he to Christian, how far yonder youngster loitereth behind.

Christian: Aye, aye, I see him: he careth not for our company.

Hopeful: But I trow it would not have hurt him, had he kept pace with us hitherto.

Christian: That is true; but I warrant you he thinketh otherwise.

Hopeful: That I think he doth; but, however, let us tarry for him. (So they did.)

Then Christian said to him, Come away, man; why do you stay so behind?

Ignorance: I take my pleasure in walking alone, even more a great deal than in company, unless I like it the better.

Then said Christian to Hopeful, (but softly,) Did I not tell you he cared not for our company? But, however, said he, come up, and let us talk away the time in this solitary place. Then, directing his speech to Ignorance, he said, Come, how do you do? How stands it between God and your soul now?

Ignorance: I hope, well; for I am always full of good motions, that come into my mind to comfort me as I walk.

Christian: What good motions? Pray tell us.

Ignorance: Why, I think of God and heaven.

Christian: So do the devils and damned souls.

Ignorance: But I think of them, and desire them.

Christian: So do many that are never like to come there. ``The soul of the sluggard desireth, and hath nothing." Prov. 13:4.

Ignorance: But I think of them, and leave all for them.

Christian: That I doubt: for to leave all is a very hard matter; yea, a harder matter than many are aware of. But why, or by what, art thou persuaded that thou hast left all for God and heaven?

Ignorance: My heart tells me so.

Christian: The wise man says, ``He that trusteth in his own heart is a fool." Prov. 28:26.

Ignorance: That is spoken of an evil heart; but mine is a good one.

Christian: But how dost thou prove that?

Ignorance: It comforts me in hopes of heaven.

Christian: That may be through its deceitfulness; for a man's heart may minister comfort to him in the hopes of that thing for which he has yet no ground to hope.

Ignorance: But my heart and life agree together; and therefore my hope is well-grounded.

Christian: Who told thee that thy heart and life agree together?

Ignorance: My heart tells me so.

Christian: ``Ask my fellow if I be a thief." Thy heart tells thee so! Except the word of God beareth witness in this matter, other testimony is of no value.

Ignorance: But is it not a good heart that hath good thoughts? and is not that a good life that is according to God's commandments?

Christian: Yes, that is a good heart that hath good thoughts, and that is a good life that is according to God's commandments; but it is one thing indeed to have these, and another thing only to think so.

Ignorance: Pray, what count you good thoughts, and a life according to God's commandments?

Christian: There are good thoughts of divers kinds; some respecting ourselves, some God, some Christ, and some other things.

Ignorance: What be good thoughts respecting ourselves?

Christian: Such as agree with the word of God.

Ignorance: When do our thoughts of ourselves agree with the word of God?

Christian: When we pass the same judgment upon ourselves which the word passes. To explain myself: the word of God saith of persons in a natural condition, ``There is none righteous, there is none that doeth good." It saith also, that, ``every imagination of the heart of man is only evil, and that continually." Gen. 6:5; Rom. 3. And again, ``The imagination of man's heart is evil from his youth." Gen. 8:21. Now, then, when we think thus of ourselves, having sense thereof, then are our thoughts good ones, because according to the word of God.

Ignorance: I will never believe that my heart is thus bad.

Christian: Therefore thou never hadst one good thought concerning thyself in thy life. But let me go on. As the word passeth a judgment upon our hearts, so it passeth a judgment upon our ways; and when the thoughts of our hearts and ways agree with the judgment which the word giveth of both, then are both good, because agreeing thereto.

Ignorance: Make out your meaning.

Christian: Why, the word of God saith, that man's ways are crooked ways, not good but perverse; it saith, they are naturally out of the good way, that they have not known it. Psa. 125:5; Prov. 2:15; Rom. 3:12. Now, when a man thus thinketh of his ways, I say, when he doth sensibly, and with heart-humiliation, thus think, then hath he good thoughts of his own ways, because his thoughts now agree with the judgment of the word of God.

Ignorance: What are good thoughts concerning God?

Christian: Even, as I have said concerning ourselves, when our thoughts of God do agree with what the word saith of him; and that is, when we think of his being and attributes as the word hath taught, of which I cannot now discourse at large. But to speak of him with reference to us: then have we right thoughts of God when we think that he knows us better than we know ourselves, and can see sin in us when and where we can see none in ourselves; when we think he knows our inmost thoughts, and that our heart, with all its depths, is always open unto his eyes; also when we think that all our righteousness stinks in his nostrils, and that therefore he cannot abide to see us stand before him in any confidence, even in all our best performances.

Ignorance: Do you think that I am such a fool as to think that God can see no further than I; or that I would come to God in the best of my performances?

Christian: Why, how dost thou think in this matter?

Ignorance: Why, to be short, I think I must believe in Christ for justification.

Christian: How! think thou must believe in Christ, when thou seest not thy need of him! Thou neither seest thy original nor actual infirmities; but hast such an opinion of thyself, and of what thou doest, as plainly renders thee to be one that did never see the necessity of Christ's personal righteousness to justify thee before God. How, then, dost thou say, I believe in Christ?

Ignorance: I believe well enough, for all that.

Christian: How dost thou believe?

Ignorance: I believe that Christ died for sinners; and that I shall be justified before God from the curse, through his gracious acceptance of my obedience to his laws. Or thus, Christ makes my duties, that are religious, acceptable to his Father by virtue of his merits, and so shall I be justified.

Christian: Let me give an answer to this confession of thy faith.

1. Thou believest with a fantastical faith; for this faith is nowhere described in the word.

2. Thou believest with a false faith; because it taketh justification from the personal righteousness of Christ, and applies it to thy own.

3. This faith maketh not Christ a justifier of thy person, but of thy actions; and of thy person for thy action's sake, which is false.

4. Therefore this faith is deceitful, even such as will leave thee under wrath in the day of God Almighty: for true justifying faith puts the soul, as sensible of its lost condition by the law, upon flying for refuge unto Christ's righteousness; (which righteousness of his is not an act of grace by which he maketh, for justification, thy obedience accepted with God, but his personal obedience to the law, in doing and suffering for us what that required at our hands;) this righteousness, I say, true faith accepteth; under the skirt of which the soul being shrouded, and by it presented as spotless before God, it is accepted, and acquitted from condemnation.

Ignorance: What! would you have us trust to what Christ in his own person has done without us? This conceit would loosen the reins of our lust, and tolerate us to live as we list: for what matter how we live, if we may be justified by Christ's personal righteousness from all, when we believe it?

Christian: Ignorance is thy name, and as thy name is, so art thou: even this thy answer demonstrateth what I say. Ignorant thou art of what justifying righteousness is, and as ignorant how to secure thy soul, through the faith of it, from the heavy wrath of God. Yea, thou also art ignorant of the true effects of saving faith in this righteousness of Christ, which is to bow and win over the heart to God in Christ, to love his name, his word, ways, and people, and not as thou ignorantly imaginest.

Hopeful: Ask him if ever he had Christ revealed to him from heaven.

Ignorance: What! you are a man for revelations! I do believe, that what both you and all the rest of you say about that matter, is but the fruit of distracted brains.

Hopeful: Why, man, Christ is so hid in God from the natural apprehensions of the flesh, that he cannot by any man be savingly known, unless God the Father reveals him to him.

Ignorance: That is your faith, but not mine, yet mine, I doubt not, is as good as yours, though I have not in my head so many whimsies as you.

Christian: Give me leave to put in a word. You ought not so slightly to speak of this matter: for this I will boldly affirm, even as my good companion hath done, that no man can know Jesus Christ but by the revelation of the Father: yea, and faith too, by which the soul layeth hold upon Christ, (if it be right,) must be wrought by the exceeding greatness of his mighty power, Matt. 11:27; 1 Cor. 12:3; Eph. 1:17-19; the working of which faith, I perceive, poor Ignorance, thou art ignorant of. Be awakened, then, see thine own wretchedness, and fly to the Lord Jesus; and by his righteousness, which is the righteousness of God, (for he himself is God,) thou shalt be delivered from condemnation.

Ignorance: You go so fast I cannot keep pace with you; do you go on before: I must stay a while behind.

Then they said,
\begin{verse}
``Well, Ignorance, wilt thou yet foolish be,\\
To slight good counsel, ten times given thee?\\
And if thou yet refuse it, thou shalt know,\\
Ere long, the evil of thy doing so.\\
Remember, man, in time: stoop, do not fear:\\
Good counsel, taken well, saves; therefore hear.\\
But if thou yet shalt slight it, thou wilt be\\
The loser, Ignorance, I'll warrant thee."\\
\end{verse}
