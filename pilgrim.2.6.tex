\chapter[THE SIXTH STAGE]{}

Now I saw that they went on to the ascent that was a little way off, cast up to be a prospect for pilgrims. That was the place from whence Christian had the first sight of Faithful his brother. Wherefore, here they sat down and rested. They also here did eat and drink, and make merry, for that they had gotten deliverance from this so dangerous an enemy. As they sat thus and did eat, Christiana asked the guide, if he had caught no hurt in the battle? Then said Mr. Great-Heart, No, save a little on my flesh; yet that also shall be so far from being to my detriment, that it is at present a proof of my love to my master and you, and shall be a means, by grace, to increase my reward at last.

Christiana: But were you not afraid, good sir, when you saw him come with his club?

Mr. Great-Heart: It is my duty, said he, to mistrust my own ability, that I may have reliance on Him who is stronger than all.

Christiana: But what did you think when he fetched you down to the ground at the first blow?

Mr. Great-Heart: Why, I thought, quoth he, that so my Master himself was served, and yet he it was that conquered at last. 2 Cor. 4:10,11; Rom. 8:37.

Matthew: When you all have thought what you please, I think God has been wonderfully good unto us, both in bringing us out of this valley, and in delivering us out of the hand of this enemy. For my part, I see no reason why we should distrust our God any more, since he has now, and in such a place as this, given us such testimony of his love. Then they got up, and went forward.

Now a little before them stood an oak; and under it, when they came to it, they found an old pilgrim fast asleep. They knew that he was a pilgrim by his clothes, and his staff, and his girdle.

So the guide, Mr. Great-Heart, awaked him; and the old gentleman, as he lifted up his eyes, cried out, What's the matter? Who are you; and what is your business here?

Mr. Great-Heart: Come, man, be not so hot; here are none but friends. Yet the old man gets up, and stands upon his guard, and will know of them what they are. Then said the guide, My name is Great-Heart: I am the guide of these pilgrims that are going to the Celestial country.

Mr. Honest: Then said Mr. Honest, I cry you mercy: I feared that you had been of the company of those that some time ago did rob Little-Faith of his money; but, now I look better about me, I perceive you are honester people.

Mr. Great-Heart: Why, what would or could you have done to have helped yourself, if indeed we had been of that company?

Mr. Honest: Done! Why, I would have fought as long as breath had been in me: and had I so done, I am sure you could never have given me the worst on't; for a Christian can never be overcome, unless he shall yield of himself.

Mr. Great-Heart: Well said, father Honest, quoth the guide; for by this I know thou art a cock of the right kind, for thou hast said the truth.

Mr. Honest: And by this also I know that thou knowest what true pilgrimage is; for all others do think that we are the soonest overcome of any.

Mr. Great-Heart: Well, now we are so happily met, pray let me crave your name, and the name of the place you came from.

Mr. Honest: My name I cannot tell you, but I came from the town of Stupidity: it lieth about four degrees beyond the city of Destruction.

Mr. Great-Heart: Oh, Are you that countryman? Then I deem I have half a guess of you: your name is Old Honesty, is it not?

Mr. Honest: So the old gentleman blushed, and said, Not honesty in the abstract, but Honest is my name; and I wish that my nature may agree to what I am called. But, sir, said the old gentleman, how could you guess that I am such a man, since I came from such a place?

Mr. Great-Heart: I had heard of you before, by my Master; for he knows all things that are done on the earth. But I have often wondered that any should come from your place; for your town is worse than is the city of Destruction itself.

Mr. Honest: Yes, we lie more off from the sun, and so are more cold and senseless. But were a man in a mountain of ice, yet if the Sun of righteousness will arise upon him, his frozen heart shall feel a thaw; and thus it has been with me.

Mr. Great-Heart: I believe it, father Honest, I believe it; for I know the thing is true.

Then the old gentleman saluted all the pilgrims with a holy kiss of charity, and asked them their names, and how they had fared since they set out on their pilgrimage.

Christiana: Then said Christiana, My name I suppose you have heard of; good Christian was my husband, and these four are his children. But can you think how the old gentleman was taken, when she told him who she was? He skipped, he smiled, he blessed them with a thousand good wishes, saying,

Mr. Honest: I have heard much of your husband, and of his travels and wars which he underwent in his days. Be it spoken to your comfort, the name of your husband rings all over these parts of the world: his faith, his courage, his enduring, and his sincerity under all, had made his name famous. Then he turned him to the boys, and asked them of their names, which they told him. Then said he unto them, Matthew, be thou like Matthew the publican, not in vice, but in virtue. Matt. 10:3. Samuel, said he, be thou like Samuel the prophet, a man of faith and prayer. Psa. 99:6. Joseph, said he, be thou like Joseph in Potiphar's house, chaste, and one that flees from temptation. Gen. 39. And James, be thou like James the just, and like James the brother of our Lord. Acts 1:13. Then they told him of Mercy, and how she had left her town and her kindred to come along with Christiana and with her sons. At that the old honest man said, Mercy is thy name: by mercy shalt thou be sustained and carried through all those difficulties that shall assault thee in thy way, till thou shalt come thither where thou shalt look the Fountain of mercy in the face with comfort. All this while the guide, Mr. Great-Heart, was very well pleased, and smiled upon his companions.

Now, as they walked along together, the guide asked the old gentleman if he did not know one Mr. Fearing, that came on pilgrimage out of his parts.

Mr. Honest: Yes, very well, said he. He was a man that had the root of the matter in him; but he was one of the most troublesome pilgrims that ever I met with in all my days.

Mr. Great-Heart: I perceive you knew him, for you have given a very right character of him.

Mr. Honest: Knew him! I was a great companion of his; I was with him most an end; when he first began to think upon what would come upon us hereafter, I was with him.

Mr. Great-Heart: I was his guide from my Master's house to the gates of the Celestial City.

Mr. Honest: Then you knew him to be a troublesome one.

Mr. Great-Heart: I did so; but I could very well bear it; for men of my calling are oftentimes intrusted with the conduct of such as he was.

Mr. Honest: Well then, pray let us hear a little of him, and how he managed himself under your conduct.

Mr. Great-Heart: Why, he was always afraid that he should come short of whither he had a desire to go. Every thing frightened him that he heard any body speak of, if it had but the least appearance of opposition in it. I heard that he lay roaring at the Slough of Despond for above a month together; nor durst he, for all he saw several go over before him, venture, though they many of them offered to lend him their hands. He would not go back again, neither. The Celestial City-he said he should die if he came not to it; and yet he was dejected at every difficulty, and stumbled at every straw that any body cast in his way. Well, after he had lain at the Slough of Despond a great while, as I have told you, one sunshiny morning, I do not know how, he ventured, and so got over; but when he was over, he would scarce believe it. He had, I think, a Slough of Despond in his mind, a slough that he carried every where with him, or else he could never have been as he was. So he came up to the gate, you know what I mean, that stands at the head of this way, and there also he stood a good while before he would venture to knock. When the gate was opened, he would give back, and give place to others, and say that he was not worthy. For, all he got before some to the gate, yet many of them went in before him. There the poor man would stand shaking and shrinking; I dare say it would have pitied one's heart to have seen him. Nor would he go back again. At last he took the hammer that hanged on the gate, in his hand, and gave a small rap or two; then one opened to him, but he shrunk back as before. He that opened stepped out after him, and said, Thou trembling one, what wantest thou? With that he fell down to the ground. He that spoke to him wondered to see him so faint, so he said to him, Peace be to thee; up, for I have set open the door to thee; come in, for thou art blessed. With that he got up, and went in trembling; and when he was in, he was ashamed to show his face. Well, after he had been entertained there a while, as you know how the manner is, he was bid go on his way, and also told the way he should take. So he went on till he came out to our house; but as he behaved himself at the gate, so he did at my Master the Interpreter's door. He lay there about in the cold a good while, before he would adventure to call; yet he would not go back: and the nights were long and cold then. Nay, he had a note of necessity in his bosom to my master to receive him, and grant him the comfort of his house, and also to allow him a stout and valiant conductor, because he was himself so chicken-hearted a man; and yet for all that he was afraid to call at the door. So he lay up and down thereabouts, till, poor man, he was almost starved; yea, so great was his dejection, that though he saw several others for knocking get in, yet he was afraid to venture. At last, I think I looked out of the window, and perceiving a man to be up and down about the door, I went out to him, and asked what he was: but, poor man, the water stood in his eyes; so I perceived what he wanted. I went therefore in, and told it in the house, and we showed the thing to our Lord: so he sent me out again, to entreat him to come in; but I dare say, I had hard work to do it. At last he came in; and I will say that for my Lord, he carried it wonderful lovingly to him. There were but a few good bits at the table, but some of it was laid upon his trencher. Then he presented the note; and my Lord looked thereon, and said his desire should be granted. So when he had been there a good while, he seemed to get some heart, and to be a little more comfortable. For my Master, you must know, is one of very tender bowels, especially to them that are afraid; wherefore he carried it so towards him as might tend most to his encouragement. Well, when he had had a sight of the things of the place, and was ready to take his journey to go to the city, my Lord, as he did to Christian before, gave him a bottle of spirits, and some comfortable things to eat. Thus we set forward, and I went before him; but the man was but of few words, only he would sigh aloud.

When we were come to where the three fellows were hanged, he said that he doubted that that would be his end also. Only he seemed glad when he saw the cross and the sepulchre. There I confess he desired to stay a little to look; and he seemed for a while after to be a little cheery. When he came to the Hill Difficulty, he made no stick at that, nor did he much fear the lions: for you must know, that his troubles were not about such things as these; his fear was about his acceptance at last.

I got him in at the house Beautiful, I think, before he was willing. Also, when he was in, I brought him acquainted with the damsels of the place; but he was ashamed to make himself much in company. He desired much to be alone; yet he always loved good talk, and often would get behind the screen to hear it. He also loved much to see ancient things, and to be pondering them in his mind. He told me afterward, that he loved to be in those two houses from which he came last, to wit, at the gate, and that of the Interpreter, but that he durst not be so bold as to ask.

When we went also from the house Beautiful, down the hill, into the Valley of Humiliation, he went down as well as ever I saw a man in my life; for he cared not how mean he was, so he might be happy at last. Yea, I think there was a kind of sympathy betwixt that Valley and him; for I never saw him better in all his pilgrimage than he was in that Valley.

Here he would lie down, embrace the ground, and kiss the very flowers that grew in this valley. Lam. 3:27-29. He would now be up every morning by break of day, tracing and walking to and fro in the valley.

But when he was come to the entrance of the Valley of the Shadow of Death, I thought I should have lost my man: not for that he had any inclination to go back; that he always abhorred; but he was ready to die for fear. Oh, the hobgoblins will have me! the hobgoblins will have me! cried he; and I could not beat him out of it. He made such a noise, and such an outcry here, that had they but heard him, it was enough to encourage them to come and fall upon us.

But this I took very great notice of, that this valley was as quiet when we went through it, as ever I knew it before or since. I suppose those enemies here had now a special check from our Lord, and a command not to meddle until Mr. Fearing had passed over it.

It would be too tedious to tell you of all; we will therefore only mention a passage or two more. When he was come to Vanity Fair, I thought he would have fought with all the men in the fair. I feared there we should have been both knocked on the head, so hot was he against their fooleries. Upon the Enchanted Ground he was very wakeful. But when he was come at the river where was no bridge, there again he was in a heavy case. Now, now, he said, he should be drowned forever, and so never see that face with comfort that he had come so many miles to behold.

And here also I took notice of what was very remarkable: the water of that river was lower at this time than ever I saw it in all my life; so he went over at last, not much above wetshod. When he was going up to the gate, I began to take leave of him, and to wish him a good reception above. So he said, I shall, I shall. Then parted we asunder, and I saw him no more.

Mr. Honest: Then it seems he was well at last?

Mr. Great-Heart: Yes, yes, I never had doubt about him. He was a man of a choice spirit, only he was always kept very low, and that made his life so burdensome to himself, and so troublesome to others. Psa. 88. He was, above many, tender of sin: he was so afraid of doing injuries to others, that he often would deny himself of that which was lawful, because he would not offend. Rom. 14:21; 1 Cor. 8:13.

Mr. Honest: But what should be the reason that such a good man should be all his days so much in the dark?

Mr. Great-Heart: There are two sorts of reasons for it. One is, the wise God will have it so: some must pipe, and some must weep. Matt. 11:16. Now Mr. Fearing was one that played upon the bass. He and his fellows sound the sackbut, whose notes are more doleful than the notes of other music are: though indeed, some say, the bass is the ground of music. And for my part, I care not at all for that profession which begins not in heaviness of mind. The first string that the musician usually touches is the bass, when he intends to put all in tune. God also plays upon this string first, when he sets the soul in tune for himself. Only there was the imperfection of Mr. Fearing; he could play upon no other music but this till towards his latter end.

[I make bold to talk thus metaphorically for the ripening of the wits of young readers, and because, in the book of Revelation, the saved are compared to a company of musicians, that play upon their trumpets and harps, and sing their songs before the throne.Rev. 5:8; 14:2,3.]

Mr. Honest: He was a very zealous man, as one may see by the relation you have given of him. Difficulties, lions, or Vanity Fair, he feared not at all; it was only sin, death, and hell, that were to him a terror, because he had some doubts about his interest in that celestial country.

Mr. Great-Heart: You say right; those were the things that were his troublers; and they, as you have well observed, arose from the weakness of his mind thereabout, not from weakness of spirit as to the practical part of a pilgrim's life. I dare believe that, as the proverb is, he could have bit a firebrand, had it stood in his way; but the things with which he was oppressed, no man ever yet could shake off with ease.

Christiana: Then said Christiana, This relation of Mr. Fearing has done me good; I thought nobody had been like me. But I see there was some semblance betwixt this good man and me: only we differed in two things. His troubles were so great that they broke out; but mine I kept within. His also lay so hard upon him, they made him that he could not knock at the houses provided for entertainment; but my trouble was always such as made me knock the louder.

Mercy: If I might also speak my heart, I must say that something of him has also dwelt in me. For I have ever been more afraid of the lake, and the loss of a place in paradise, than I have been of the loss other things. O, thought I, may I have the happiness to have a habitation there! Tis enough, though I part with all the world to win it.

Matthew: Then said Matthew, Fear was one thing that made me think that I was far from having that within me which accompanies salvation. But if it was so with such a good man as he, why may it not also go well with me?

James: No fears no grace, said James. Though there is not always grace where there is the fear of hell, yet, to be sure, there is no grace where there is no fear of God.

Mr. Great-Heart: Well said, James; thou hast hit the mark. For the fear of God is the beginning of wisdom; and to be sure, they that want the beginning have neither middle nor end. But we will here conclude our discourse of Mr. Fearing, after we have sent after him this farewell.
\begin{verse}
``Well, Master Fearing, thou didst fear\\
Thy God, and wast afraid\\
Of doing any thing, while here,\\
That would have thee betrayed.\\
And didst thou fear the lake and pit?\\
Would others do so too!\\
For, as for them that want thy wit,\\
They do themselves undo."\\
\end{verse}

Now I saw that they still went on in their talk. For after Mr. Great-Heart had made an end with Mr. Fearing, Mr. Honest began to tell them of another, but his name was Mr. Self-will. He pretended himself to be a pilgrim, said Mr. Honest; but I persuade myself he never came in at the gate that stands at the head of the way.

Mr. Great-Heart: Had you ever any talk with him about it?

Mr. Honest: Yes, more than once or twice; but he would always be like himself, self-willed. He neither cared for man, nor argument, nor yet example; what his mind prompted him to, that he would do, and nothing else could he be got to do.

Mr. Great-Heart: Pray, what principles did he hold? for I suppose you can tell.

Mr. Honest: He held that a man might follow the vices as well as the virtues of pilgrims; and that if he did both, he should be certainly saved.

Mr. Great-Heart: How? If he had said, it is possible for the best to be guilty of the vices, as well as to partake of the virtues of pilgrims, he could not much have been blamed; for indeed we are exempted from no vice absolutely, but on condition that we watch and strive. But this, I perceive, is not the thing; but if I understand you right, your meaning is, that he was of opinion that it was allowable so to be.

Mr. Honest: Aye, aye, so I mean, and so he believed and practised.

Mr. Great-Heart: But what grounds had he for his so saying?

Mr. Honest: Why, he said he had the Scripture for his warrant.

Mr. Great-Heart: Prithee, Mr. Honest, present us with a few particulars.

Mr. Honest: So I will. He said, to have to do with other men's wives had been practised by David, God's beloved; and therefore he could do it. He said, to have more women than one was a thing that Solomon practised, and therefore he could do it. He said, that Sarah and the godly midwives of Egypt lied, and so did save Rahab, and therefore he could do it. He said, that the disciples went at the bidding of their Master, and took away the owner's ass, and therefore he could do so too. He said, that Jacob got the inheritance of his father in a way of guile and dissimulation, and therefore he could do so too.

Mr. Great-Heart: High base indeed! And are you sure he was of this opinion?

Mr. Honest: I heard him plead for it, bring Scripture for it, bring arguments for it, etc.

Mr. Great-Heart: An opinion that is not fit to be with any allowance in the world!

Mr. Honest: You must understand me rightly: he did not say that any man might do this; but that they who had the virtues of those that did such things, might also do the same.

Mr. Great-Heart: But what more false than such a conclusion? For this is as much as to say, that because good men heretofore have sinned of infirmity, therefore he had allowance to do it of a presumptuous mind; or that if, because a child, by the blast of the wind, or for that it stumbled at a stone, fell down and defiled itself in the mire, therefore he might wilfully lie down and wallow like a boar therein. Who could have thought that any one could so far have been blinded by the power of lust? But what is written must be true: they ``stumble at the word, being disobedient; whereunto also they were appointed." 1 Peter, 2:8. His supposing that such may have the godly men's virtues, who addict themselves to their vices, is also a delusion as strong as the other. To eat up the sin of God's people, Hos. 4:8, as a dog licks up filth, is no sign that one is possessed with their virtues. Nor can I believe that one who is of this opinion, can at present have faith or love in him. But I know you have made strong objections against him; prithee what can he say for himself?

Mr. Honest: Why, he says, to do this by way of opinion, seems abundantly more honest than to do it, and yet hold contrary to it in opinion.

Mr. Great-Heart: A very wicked answer. For though to let loose the bridle to lusts, while our opinions are against such things, is bad; yet, to sin, and plead a toleration so to do, is worse: the one stumbles beholders accidentally, the other leads them into the snare.

Mr. Honest: There are many of this man's mind, that have not this man's mouth; and that makes going on pilgrimage of so little esteem as it is.

Mr. Great-Heart: You have said the truth, and it is to be lamented: but he that feareth the King of paradise, shall come out of them all.

Christiana: There are strange opinions in the world. I know one that said, it was time enough to repent when we come to die.

Mr. Great-Heart: Such are not overwise; that man would have been loth, might he have had a week to run twenty miles in his life, to defer his journey to the last hour of that week.

Mr. Honest: You say right; and yet the generality of them who count themselves pilgrims, do indeed do thus. I am, as you see, an old man, and have been a traveller in this road many a day; and I have taken notice of many things.

I have seen some that have set out as if they would drive all the world before them, who yet have, in a few days, died as they in the wilderness, and so never got sight of the promised land. I have seen some that have promised nothing at first setting out to be pilgrims, and who one would have thought could not have lived a day, that have yet proved very good pilgrims. I have seen some who have run hastily forward, that again have, after a little time, run just as fast back again. I have seen some who have spoken very well of a pilgrim's life at first, that after a while have spoken as much against it. I have heard some, when they first set out for paradise, say positively, there is such a place, who, when they have been almost there, have come back again, and said there is none. I have heard some vaunt what they would do in case they should be opposed, that have, even at a false alarm, fled faith, the pilgrim's way, and all.

Now, as they were thus on their way, there came one running to meet them, and said, Gentlemen, and you of the weaker sort, if you love life, shift for yourselves, for the robbers are before you.

Mr. Great-Heart: Then said Mr. Great-Heart, They be the three that set upon Little-Faith heretofore. Well, said he, we are ready for them: so they went on their way. Now they looked at every turning when they should have met with the villains; but whether they heard of Mr. Great-Heart, or whether they had some other game, they came not up to the pilgrims.

Christiana then wished for an inn to refresh herself and her children, because they were weary. Then said Mr. Honest, There is one a little before us, where a very honorable disciple, one Gaius, dwells. Rom. 16:23. So they all concluded to turn in thither; and the rather, because the old gentleman gave him so good a report. When they came to the door they went in, not knocking, for folks use not to knock at the door of an inn. Then they called for the master of the house, and he came to them. So they asked if they might lie there that night.

Gaius: Yes, gentlemen, if you be true men; for my house is for none but pilgrims. Then were Christiana, Mercy, and the boys the more glad, for that the innkeeper was a lover of pilgrims. So they called for rooms, and he showed them one for Christiana and her children and Mercy, and another for Mr. Great-Heart and the old gentleman.

Mr. Great-Heart: Then said Mr. Great-Heart, good Gaius, what hast thou for supper? for these pilgrims have come far to-day, and are weary.

Gaius: It is late, said Gaius, so we cannot conveniently go out to seek food; but such as we have you shall be welcome to, if that will content.

Mr. Great-Heart: We will be content with what thou hast in the house; for as much as I have proved thee, thou art never destitute of that which is convenient.

Then he went down and spake to the cook, whose name was, Taste-that-which-is-good, to get ready supper for so many pilgrims. This done, he comes up again, saying, Come, my good friends, you are welcome to me, and I am glad that I have a house to entertain you in; and while supper is making ready, if you please, let us entertain one another with some good discourse: so they all said, Content.

Gaius: Then said Gaius, Whose wife is this aged matron? and whose daughter is this young damsel?

Mr. Great-Heart: This woman is the wife of one Christian, a pilgrim of former times; and these are his four children. The maid is one of her acquaintance, one that she hath persuaded to come with her on pilgrimage. The boys take all after their father, and covet to tread in his steps; yea, if they do but see any place where the old pilgrim hath lain, or any print of his foot, it ministereth joy to their hearts, and they covet to lie or tread in the same.

Gaius: Then said Gaius, Is this Christian's wife, and are these Christian's children? I knew your husband's father, yea, also his father's father. Many have been good of this stock; their ancestors dwelt first at Antioch. Acts 11:26. Christian's progenitors (I suppose you have heard your husband talk of them) were very worthy men. They have, above any that I know, showed themselves men of great virtue and courage for the Lord of the pilgrims, his ways, and them that loved him. I have heard of many of your husband's relations that have stood all trials for the sake of the truth. Stephen, that was one of the first of the family from whence your husband sprang, was knocked on the head with stones. Acts 7:59, 60. James, another of this generation, was slain with the edge of the sword. Acts 12:2. To say nothing of Paul and Peter, men anciently of the family from whence your husband came, there was Ignatius, who was cast to the lions; Romanus, whose flesh was cut by pieces from his bones; and Polycarp, that played the man in the fire. There was he that was hanged up in a basket in the sun for the wasps to eat; and he whom they put into a sack, and cast him into the sea to be drowned. It would be impossible utterly to count up all of that family who have suffered injuries and death for the love of a pilgrim's life. Nor can I but be glad to see that thy husband has left behind him four such boys as these. I hope they will bear up their father's name, and tread in their father's steps, and come to their father's end.

Mr. Great-Heart: Indeed, sir, they are likely lads: they seem to choose heartily their father's ways.

Gaius: That is it that I said. Wherefore Christian's family is like still to spread abroad upon the face of the ground, and yet to be numerous upon the face of the earth; let Christiana look out some damsels for her sons, to whom they may be betrothed, etc., that the name of their father, and the house of his progenitors, may never be forgotten in the world.

Mr. Honest: Tis pity his family should fall and be extinct.

Gaius: Fall it cannot, but be diminished it may; but let Christiana take my advice, and that is the way to uphold it. And, Christiana, said this innkeeper, I am glad to see thee and thy friend Mercy together here, a lovely couple. And if I may advise, take Mercy into a nearer relation to thee: if she will, let her be given to Matthew thy eldest son. It is the way to preserve a posterity in the earth. So this match was concluded, and in process of time they were married: but more of that hereafter.

Gaius also proceeded, and said, I will now speak on the behalf of women, to take away their reproach. For as death and the curse came into the world by a woman, Gen. 3, so also did life and health: God sent forth his Son, made of a woman. Gal. 4:4. Yea, to show how much they that came after did abhor the act of the mother, this sex in the Old Testament coveted children, if happily this or that woman might be the mother of the Saviour of the world. I will say again, that when the Saviour was come, women rejoiced in him, before either man or angel. Luke 1:42-46. I read not that ever any man did give unto Christ so much as one groat; but the women followed him, and ministered to him of their substance. Luke 8:2,3. Twas a woman that washed his feet with tears, Luke 7:37-50, and a woman that anointed his body at the burial. John 11:2; 12:3. They were women who wept when he was going to the cross, Luke 23:27, and women that followed him from the cross, Matt. 27:55,56; Luke 23:55, and sat over against his sepulchre when he was buried. Matt. 27:61. They were women that were first with him at his resurrection-morn, Luke 24:1, and women that brought tidings first to his disciples that he was risen from the dead. Luke 24:22,23. Women therefore are highly favored, and show by these things that they are sharers with us in the grace of life.

Now the cook sent up to signify that supper was almost ready, and sent one to lay the cloth, and the trenchers, and to set the salt and bread in order.

Then said Matthew, The sight of this cloth, and of this forerunner of the supper, begetteth in me a greater appetite for my food than I had before.

Gaius: So let all ministering doctrines to thee in this life beget in thee a greater desire to sit at the supper of the great King in his kingdom; for all preaching, books, and ordinances here, are but as the laying of the trenchers, and the setting of salt upon the board, when compared with the feast which our Lord will make for us when we come to his house.

So supper came up. And first a heave-shoulder and a wave-breast were set on the table before them; to show that they must begin their meal with prayer and praise to God. The heave-shoulder David lifted up his heart to God with; and with the wave-breast, where his heart lay, he used to lean upon his harp when he played. Lev. 7: 32-34; 10:14,15; Psalm 25:1; Heb. 13:15. These two dishes were very fresh and good, and they all ate heartily thereof.

The next they brought up was a bottle of wine, as red as blood. Deut. 32:14; Judges 9:13; John 15:5. So Gaius said to them, Drink freely; this is the true juice of the vine, that makes glad the heart of God and man. So they drank and were merry.

The next was a dish of milk well crumbed; Gaius said, Let the boys have that, that they may grow thereby. 1 Pet. 2:1,2.

Then they brought up in course a dish of butter and honey. Then said Gaius, Eat freely of this, for this is good to cheer up and strengthen your judgments and understandings. This was our Lord's dish when he was a child: ``Butter and honey shall he eat, that he may know to refuse the evil, and choose the good." Isa. 7:15.

Then they brought them up a dish of apples, and they were very good-tasted fruit. Then said Matthew, May we eat apples, since it was such by and with which the serpent beguiled our first mother?

Then said Gaius,
\begin{verse}
``Apples were they with which we were beguil'd,\\
Yet sin, not apples, hath our souls defil'd:\\
Apples forbid, if ate, corrupt the blood;\\
To eat such, when commanded, does us good:\\
Drink of his flagons then, thou church, his dove,\\
And eat his apples, who art sick of love."\\
\end{verse}

Then said Matthew, I made the scruple, because I a while since was sick with the eating of fruit.

Gaius: Forbidden fruit will make you sick; but not what our Lord has tolerated.

While they were thus talking, they were presented with another dish, and it was a dish of nuts. Song 6:11. Then said some at the table, Nuts spoil tender teeth, especially the teeth of children: which when Gaius heard, he said,
\begin{verse}
``Hard texts are nuts, (I will not call them cheaters,)\\
Whose shells do keep the kernel from the eaters:\\
Open the shells, and you shall have the meat;\\
They here are brought for you to crack and eat."\\
\end{verse}

Then were they very merry, and sat at the table a long time, talking of many things. Then said the old gentleman, My good landlord, while we are cracking your nuts, if you please, do you open this riddle:
\begin{verse}
``A man there was, though some did count him mad,\\
The more he cast away, the more he had."\\
\end{verse}

Then they all gave good heed, wondering what good Gaius would say; so he sat still a while, and then thus replied:
\begin{verse}
``He who bestows his goods upon the poor,\\
Shall have as much again, and ten times more."\\
\end{verse}

Then said Joseph, I dare say, sir, I did not think you could have found it out.

Oh, said Gaius, I have been trained up in this way a great while: nothing teaches like experience. I have learned of my Lord to be kind, and have found by experience that I have gained thereby. There is that scattereth, and yet increaseth; and there is that withholdeth more than is meet, but it tendeth to poverty: There is that maketh himself rich, yet hath nothing; there is that maketh himself poor, yet hath great riches. Prov. 11:24; 13:7.

Then Samuel whispered to Christiana, his mother, and said, Mother, this is a very good man's house: let us stay here a good while, and let my brother Matthew be married here to Mercy, before we go any further. The which Gaius the host overhearing, said, With a very good will, my child.

So they stayed there more than a month, and Mercy was given to Matthew to wife.

While they stayed here, Mercy, as her custom was, would be making coats and garments to give to the poor, by which she brought a very good report upon the pilgrims.

But to return again to our story: After supper the lads desired a bed, for they were weary with travelling: Then Gaius called to show them their chamber; but said Mercy, I will have them to bed. So she had them to bed, and they slept well: but the rest sat up all night; for Gaius and they were such suitable company, that they could not tell how to part. After much talk of their Lord, themselves, and their journey, old Mr. Honest, he that put forth the riddle to Gaius, began to nod. Then said Great-Heart, What, sir, you begin to be drowsy; come, rub up, now here is a riddle for you. Then said Mr. Honest, Let us hear it. Then replied Mr. Great-heart,
\begin{verse}
``He that would kill, must first be overcome:\\
Who live abroad would, first must die at home."\\
\end{verse}

Ha, said Mr. Honest, it is a hard one; hard to expound, and harder to practise. But come, landlord, said he, I will, if you please, leave my part to you: do you expound it, and I will hear what you say.

No, said Gaius, it was put to you, and it is expected you should answer it. Then said the old gentleman,
\begin{verse}
``He first by grace must conquered be,\\
That sin would mortify;\\
Who that he lives would convince me,\\
Unto himself must die."\\
\end{verse}

It is right, said Gaius; good doctrine and experience teach this. For, first, until grace displays itself, and overcomes the soul with its glory, it is altogether without heart to oppose sin. Besides, if sin is Satan's cords, by which the soul lies bound, how should it make resistance before it is loosed from that infirmity? Secondly, Nor will any one that knows either reason or grace, believe that such a man can be a living monument of grace that is a slave to his own corruptions. And now it comes into my mind, I will tell you a story worth the hearing. There were two men that went on pilgrimage; the one began when he was young, the other when he was old. The young man had strong corruptions to grapple with; the old man's were weak with the decays of nature. The young man trod his steps as even as did the old one, and was every way as light as he. Who now, or which of them, had their graces shining clearest, since both seemed to be alike?

Mr. Honest: The young man's, doubtless. For that which makes head against the greatest opposition, gives best demonstration that it is strongest; especially when it also holdeth pace with that which meets not with half so much, as to be sure old age does not. Besides, I have observed that old men have blessed themselves with this mistake; namely, taking the decays of nature for a gracious conquest over corruptions, and so have been apt to beguile themselves. Indeed, old men that are gracious are best able to give advice to them that are young, because they have seen most of the emptiness of things: but yet, for an old and a young man to set out both together, the young one has the advantage of the fairest discovery of a work of grace within him, though the old man's corruptions are naturally the weakest. Thus they sat talking till break of day.

Now, when the family were up, Christiana bid her son James that he should read a chapter; so he read 53d of Isaiah. When he had done, Mr. Honest asked why it was said that the Saviour was to come ``out of a dry ground;" and also, that ``he had no form nor comeliness in him."

Mr. Great-Heart: Then said Mr. Great-Heart, To the first I answer, because the church of the Jews, of which Christ came, had then lost almost all the sap and spirit of religion. To the second I say, the words are spoken in the person of unbelievers, who, because they want the eye that can see into our Prince's heart, therefore they judge of him by the meanness of his outside; just like those who, not knowing that precious stones are covered over with a homely crust, when they have found one, because they know not what they have found, cast it away again, as men do a common stone.

Well, said Gaius, now you are here, and since, as I know, Mr. Great-Heart is good at his weapons, if you please, after we have refreshed ourselves, we will walk into the fields, to see if we can do any good. About a mile from hence there is one Slay-good, a giant, that doth much annoy the King's highway in these parts; and I know whereabout his haunt is. He is master of a number of thieves: t would be well if we could clear these parts of him. So they consented and went: Mr. Great-Heart with his sword, helmet, and shield; and the rest with spears and staves.

When they came to the place where he was, they found him with one Feeble-mind in his hand, whom his servants had brought unto him, having taken him in the way. Now the giant was rifling him, with a purpose after that to pick his bones; for he was of the nature of flesheaters.

Well, so soon as he saw Mr. Great-Heart and his friends at the mouth of his cave, with their weapons, he demanded what they wanted.

Mr. Great-Heart: We want thee; for we are come to revenge the quarrels of the many that thou hast slain of the pilgrims, when thou hast dragged them out of the King's highway: wherefore come out of thy cave. So he armed himself and came out, and to battle they went, and fought for above an hour, and then stood still to take wind.

Slay-Good: Then said the giant, Why are you here on my ground?

Mr. Great-Heart: To revenge the blood of pilgrims, as I told thee before. So they went to it again, and the giant made Mr. Great-Heart give back; but he came up again, and in the greatness of his mind he let fly with such stoutness at the giant's head and sides, that he made him let his weapon fall out of his hand. So he smote him, and slew him, and cut off his head, and brought it away to the inn. He also took Feeble-mind the pilgrim, and brought him with him to his lodgings. When they were come home, they showed his head to the family, and set it up, as they had done others before, for a terror to those that should attempt to do as he hereafter.

Then they asked Mr. Feeble-Mind how he fell into his hands.

Mr. Feeble-Mind: Then said the poor man, I am a sickly man, as you see: and because death did usually once a day knock at my door, I thought I should never be well at home; so I betook myself to a pilgrim's life, and have traveled hither from the town of Uncertain, where I and my father were born. I am a man of no strength at all of body, nor yet of mind, but would, if I could, though I can but crawl, spend my life in the pilgrim's way. When I came at the gate that is at the head of the way, the Lord of that place did entertain me freely; neither objected he against my weakly looks, nor against my feeble mind; but gave me such things as were necessary for my journey, and bid me hope to the end. When I came to the house of the Interpreter, I received much kindness there: and because the hill of Difficulty was judged too hard for me, I was carried up that by one of his servants. Indeed, I have found much relief from pilgrims, though none were willing to go so softly as I am forced to do: yet still as they came on, they bid me be of good cheer, and said, that it was the will of their Lord that comfort should be given to the feeble-minded, 1 Thess. 5:14; and so went on their own pace. When I was come to Assault-lane, then this giant met with me, and bid me prepare for an encounter. But, alas, feeble one that I was, I had more need of a cordial; so he came up and took me. I conceited he would not kill me. Also when he had got me into his den, since I went not with him willingly, I believed I should come out alive again; for I have heard, that not any pilgrim that is taken captive by violent hands, if he keeps heart whole towards his Master, is, by the laws of providence, to die by the hand of the enemy. Robbed I looked to be, and robbed to be sure I am; but I have, as you see, escaped with life, for the which I thank my King as the author, and you as the means. Other brunts I also look for; but this I have resolved on, to wit, to run when I can, to go when I cannot run, and to creep when I cannot go. As to the main, I thank him that loved me, I am fixed; my way is before me, my mind is beyond the river that has no bridge, though I am, as you see, but of a feeble mind.

Mr. Honest: Then said old Mr. Honest, Have not you, sometime ago, been acquainted with one Mr. Fearing, a pilgrim?

Mr. Feeble-Mind: Acquainted with him! Yes, he came from the town of Stupidity, which lieth four degrees to the northward of the city of Destruction, and as many off of where I was born: yet we were well acquainted, for indeed he was my uncle, my father's brother. He and I have been much of a temper: he was a little shorter than I, but yet we were much of a complexion.

Mr. Honest: I perceive you knew him, and I am apt to believe also that you were related one to another; for you have his whitely look, a cast like his with your eye, and your speech is much alike.

Mr. Feeble-Mind: Most have said so that have known us both: and, besides, what I have read in him I have for the most part found in myself.

Gaius: Come, sir, said good Gaius, be of good cheer; you are welcome to me, and to my house. What thou hast a mind to, call for freely; and what thou wouldst have my servants do for thee, they will do it with a ready mind.

Then said Mr. Feeble-mind, This is an unexpected favor, and as the sun shining out of a very dark cloud. Did giant Slay-good intend me this favor when he stopped me, and resolved to let me go no further? Did he intend, that after he had rifled my pockets I should go to Gaius mine host? Yet so it is.

Now, just as Mr. Feeble-mind and Gaius were thus in talk, there came one running, and called at the door, and said, that about a mile and a half off there was one Mr. Not-right, a pilgrim, struck dead upon the place where he was, with a thunderbolt.

Mr. Feeble-Mind: Alas! said Mr. Feeble-mind, is he slain? He overtook me some days before I came so far as hither, and would be my company-keeper. He was also with me when Slay-good the giant took me, but he was nimble of his heels, and escaped; but it seems he escaped to die, and I was taken to live.
\begin{verse}
``What one would think doth seek to slay outright,\\
Ofttimes delivers from the saddest plight.\\
That very Providence whose face is death,\\
Doth ofttimes to the lowly life bequeath.\\
I taken was, he did escape and flee;\\
Hands cross'd gave death to him and life to me."\\
\end{verse}

Now, about this time Matthew and Mercy were married; also Gaius gave his daughter Phebe to James, Matthew's brother, to wife; after which time they yet stayed about ten days at Gaius' house, spending their time and the seasons like as pilgrims use to do.

When they were to depart, Gaius made them a feast, and they did eat and drink, and were merry. Now the hour was come that they must be gone; wherefore Mr. Great-heart called for a reckoning. But Gaius told him, that at his house it was not the custom for pilgrims to pay for their entertainment. He boarded them by the year, but looked for his pay from the good Samaritan, who had promised him, at his return, whatsoever charge he was at with them, faithfully to repay him. Luke 10:34,35. Then said Mr. Great-heart to him,

Mr. Great-Heart: Beloved, thou doest faithfully whatsoever thou doest to the brethren, and to strangers, who have borne witness of thy charity before the church, whom if thou yet bring forward on their journey, after a godly sort, thou shalt do well. 3 John 5,6. Then Gaius took his leave of them all, and his children, and particularly of Mr. Feeble-mind. He also gave him something to drink by the way.

Now Mr. Feeble-mind, when they were going out of the door, made as if he intended to linger. The which, when Mr. Great-Heart espied, he said, Come, Mr. Feeble-mind, pray do you go along with us: I will be your conductor, and you shall fare as the rest.

Mr. Feeble-Mind: Alas! I want a suitable companion. You are all lusty and strong, but I, as you see, am weak; I choose, therefore, rather to come behind, lest, by reason of my many infirmities, I should be both a burden to myself and to you. I am, as I said, a man of a weak and feeble mind, and shall be offended and made weak at that which others can bear. I shall like no laughing; I shall like no gay attire; I shall like no unprofitable questions. Nay, I am so weak a man as to be offended with that which others have a liberty to do. I do not yet know all the truth: I am a very ignorant Christian man. Sometimes, if I hear some rejoice in the Lord, it troubles me because I cannot do so too. It is with me as it is with a weak man among the strong, or as with a sick man among the healthy, or as a lamp despised; so that I know not what to do. ``He that is ready to slip with his feet is as a lamp despised in the thought of him that is at ease." Job 12:5.

Mr. Great-Heart: But, brother, said Mr. Great-Heart, I have it in commission to comfort the feeble-minded, and to support the weak. You must needs go along with us; we will wait for you; we will lend you our help; we will deny ourselves of some things, both opinionative and practical, for your sake: we will not enter into doubtful disputations before you; we will be made all things to you, rather than you shall be left behind. 1 Thess. 5:14; Rom. 14; 1 Cor. 8:9-13; 9:22.

Now, all this while they were at Gaius' door; and behold, as they were thus in the heat of their discourse, Mr. Ready-to-halt came by, with his crutches in his hand, and he also was going on pilgrimage.

Mr. Feeble-Mind: Then said Mr. Feeble-mind to him, Man, how camest thou hither? I was but now complaining that I had not a suitable companion, but thou art according to my wish. Welcome, welcome, good Mr. Ready-to-halt; I hope thou and I may be some help.

Mr. Ready-to-Halt: I shall be glad of thy company, said the other; and, good Mr. Feeble-mind, rather than we will part, since we are thus happily met, I will lend thee one of my crutches.

Mr. Feeble-Mind: Nay, said he, though I thank thee for thy good-will, I am not inclined to halt before I am lame. Howbeit, I think when occasion is, it may help me against a dog.

Mr. Ready-to-Halt: If either myself or my crutches can do thee a pleasure, we are both at thy command, good Mr. Feeble-mind.

Thus, therefore, they went on. Mr. Great-Heart and Mr. Honest went before, Christiana and her children went next, and Mr. Feeble-mind came behind, and Mr. Ready-to-halt with his crutches. Then said Mr. Honest,

Mr. Honest: Pray, sir, now we are upon the road, tell us some profitable things of some that have gone on pilgrimage before us.

Mr. Great-Heart: With a good will. I suppose you have heard how Christian of old did meet with Apollyon in the Valley of Humiliation, and also what hard work he had to go through the Valley of the Shadow of Death. Also I think you cannot but have heard how Faithful was put to it by Madam Wanton, with Adam the First, with one Discontent, and Shame; four as deceitful villains as a man can meet with upon the road.

Mr. Honest: Yes, I have heard of all this; but indeed good Faithful was hardest put to it with Shame: he was an unwearied one.

Mr. Great-Heart: Aye; for, as the pilgrim well said, he of all men had the wrong name.

Mr. Honest: But pray, sir, where was it that Christian and Faithful met Talkative? That same was also a notable one.

Mr. Great-Heart: He was a confident fool; yet many follow his ways.

Mr. Honest: He had like to have beguiled Faithful.

Mr. Great-Heart: Aye, but Christian put him into a way quickly to find him out.

Thus they went on till they came to the place where Evangelist met with Christian and Faithful, and prophesied to them what should befall them at Vanity Fair. Then said their guide, Hereabouts did Christian and Faithful meet with Evangelist, who prophesied to them of what troubles they should meet with at Vanity Fair.

Mr. Honest: Say you so? I dare say it was a hard chapter that then he did read unto them.

Mr. Great-Heart: It was so, but he gave them encouragement withal. But what do we talk of them? They were a couple of lion-like men; they had set their faces like a flint. Do not you remember how undaunted they were when they stood before the judge?

Mr. Honest: Well: Faithful bravely suffered.

Mr. Great-Heart: So he did, and as brave things came on't; for Hopeful, and some others, as the story relates it, were converted by his death.

Mr. Honest: Well, but pray go on; for you are well acquainted with things.

Mr. Great-Heart: Above all that Christian met with after he had passed through Vanity Fair, one By-ends was the arch one.

Mr. Honest: By-ends! what was he?

Mr. Great-Heart: A very arch fellow, a downright hypocrite; one that would be religious, whichever way the world went; but so cunning, that he would be sure never to lose or suffer for it. He had his mode of religion for every fresh occasion, and his wife was as good at it as he. He would turn from opinion to opinion; yea, and plead for so doing, too. But, so far as I could learn, he came to an ill end with his by-ends; nor did I ever hear that any of his children were ever of any esteem with any that truly feared God.

Now by this time they were come within sight of the town of Vanity, where Vanity Fair is kept. So, when they saw that they were so near the town, they consulted with one another how they should pass through the town; and some said one thing, and some another. At last Mr. Great-Heart said, I have, as you may understand, often been a conductor of pilgrims through this town. Now, I am acquainted with one Mr. Mnason, Acts 21:16, a Cyprusian by nation, an old disciple, at whose house we may lodge. If you think good, we will turn in there.

Content, said old Honest; Content, said Christiana; Content, said Mr. Feeble-mind; and so they said all. Now you must think it was eventide by that they got to the outside of the town; but Mr. Great-Heart knew the way to the old man's house. So thither they came; and he called at the door, and the old man within knew his tongue as soon as ever he heard it; so he opened the door, and they all came in. Then said Mnason, their host, How far have ye come to-day? So they said, from the house of Gaius our friend. I promise you, said he, you have gone a good stitch. You may well be weary; sit down. So they sat down.

Mr. Great-Heart: Then said their guide, Come, what cheer, good sirs? I dare say you are welcome to my friend.

Mr. Mnason: I also, said Mr. Mnason, do bid you welcome; and whatever you want, do but say, and we will do what we can to get it for you.

Mr. Honest: Our great want, a while since, was harbor and good company, and now I hope we have both.

Mr. Mnason: For harbor, you see what it is; but for good company, that will appear in the trial.

Mr. Great-Heart: Well, said Mr. Great-Heart, will you have the pilgrims up into their lodging?

Mr. Mnason: I will, said Mr. Mnason So he had them to their respective places; and also showed them a very fair dining-room, where they might be, and sup together until the time should come to go to rest.

Now, when they were seated in their places, and were a little cheery after their journey, Mr. Honest asked his landlord if there was any store of good people in the town.

Mr. Mnason: We have a few: for indeed they are but a few when compared with them on the other side.

Mr. Honest: But how shall we do to see some of them? for the sight of good men to them that are going on pilgrimage, is like the appearing of the moon and stars to them that are sailing upon the seas.

Mr. Mnason: Then Mr. Mnason stamped with his foot, and his daughter Grace came up. So he said unto her, Grace, go you, tell my friends, Mr. Contrite, Mr. Holy-man, Mr. Love-saints, Mr. Dare-not-lie, and Mr. Penitent, that I have a friend or two at my house who have a mind this evening to see them. So Grace went to call them, and they came; and after salutation made, they sat down together at the table.

Then said Mr. Mnason their landlord, My neighbors, I have, as you see, a company of strangers come to my house; they are pilgrims: they come from afar, and are going to Mount Zion. But who, quoth he, do you think this is? pointing his finger to Christiana. It is Christiana, the wife of Christian, the famous pilgrim, who, with Faithful his brother, was so shamefully handled in our town. At that they stood amazed, saying, We little thought to see Christiana when Grace came to call us; wherefore this is a very comfortable surprise. They then asked her of her welfare, and if these young men were her husband's sons. And when she had told them they were, they said, The King whom you love and serve make you as your father, and bring you where he is in peace.

Mr. Honest: Then Mr. Honest (when they were all sat down) asked Mr. Contrite and the rest, in what posture their town was at present.

Mr. Contrite: You may be sure we are full of hurry in fair-time. T is hard keeping our hearts and spirits in good order when we are in a cumbered condition. He that lives in such a place as this is, and has to do with such as we have, has need of an item to caution him to take heed every moment of the day.

Mr. Honest: But how are your neighbors now for quietness?

Mr. Contrite: They are much more moderate now than formerly. You know how Christian and Faithful were used at our town; but of late, I say, they have been far more moderate. I think the blood of Faithful lieth as a load upon them till now; for since they burned him, they have been ashamed to burn any more. In those days we were afraid to walk the street; but now we can show our heads. Then the name of a professor was odious; now, especially in some parts of our town, (for you know our town is large,) religion is counted honorable. Then said Mr. Contrite to them, Pray how fareth it with you in your pilgrimage? how stands the country affected towards you?

Mr. Honest: It happens to us as it happeneth to wayfaring men: sometimes our way is clean, sometimes foul; sometimes up hill, sometimes down hill; we are seldom at a certainty. The wind is not always on our backs, nor is every one a friend that we meet with in the way. We have met with some notable rubs already, and what are yet behind we know not; but for the most part, we find it true that has been talked of old, A good man must suffer trouble.

Mr. Contrite: You talk of rubs; what rubs have you met withal?

Mr. Honest: Nay, ask Mr. Great-Heart, our guide; for he can give the best account of that.

Mr. Great-Heart: We have been beset three or four times already. First, Christiana and her children were beset by two ruffians, who they feared would take away their lives. We were beset by Giant Bloody-man, Giant Maul, and Giant Slay-good. Indeed, we did rather beset the last than were beset by him. And thus it was: after we had been some time at the house of Gaius mine host, and of the whole church, we were minded upon a time to take our weapons with us, and go see if we could light upon any of those that are enemies to pilgrims; for we heard that there was a notable one thereabouts. Now Gaius knew his haunt better than I, because he dwelt thereabout. So we looked, and looked, till at last we discerned the mouth of his cave: then we were glad, and plucked up our spirits. So we approached up to his den; and lo, when we came there, he had dragged, by mere force, into his net, this poor man, Mr. Feeble-mind, and was about to bring him to his end. But when he saw us, supposing, as we thought, he had another prey, he left the poor man in his hole, and came out. So we fell to it full sore, and he lustily laid about him; but, in conclusion, he was brought down to the ground, and his head cut off, and set up by the way-side for a terror to such as should after practise such ungodliness. That I tell you the truth, here is the man himself to affirm it, who was as a lamb taken out of the mouth of the lion.

Mr. Feeble-Mind: Then said Mr. Feeble-mind, I found this true, to my cost and comfort: to my cost, when he threatened to pick my bones every moment; and to my comfort, when I saw Mr. Great-Heart and his friends, with their weapons, approach so near for my deliverance.

Mr. Holy-Man: Then said Mr. Holy-man, There are two things that they have need to possess who go on pilgrimage; courage, and an unspotted life. If they have not courage, they can never hold on their way; and if their lives be loose, they will make the very name of a pilgrim stink.

Mr. Love-Saints: Then said Mr. Love-saints, I hope this caution is not needful among you: but truly there are many that go upon the road, who rather declare themselves strangers to pilgrimage, than strangers and pilgrims on the earth.

Mr. Dare-Not-Lie: Then said Mr. Dare-not-lie, Tis true. They have neither the pilgrim's weed, nor the pilgrim's courage; they go not uprightly, but all awry with their feet; one shoe goeth inward, another outward; and their hosen are out behind: here a rag, and there a rent, to the disparagement of their Lord.

Mr. Penitent: These things, said Mr. Penitent, they ought to be troubled for; nor are the pilgrims like to have that grace put upon them and their Pilgrim's Progress as they desire, until the way is cleared of such spots and blemishes. Thus they sat talking and spending the time until supper was set upon the table, unto which they went, and refreshed their weary bodies: so they went to rest.

Now they staid in the fair a great while, at the house of Mr. Mnason, who in process of time gave his daughter Grace unto Samuel, Christian's son, to wife, and his daughter Martha to Joseph.

The time, as I said, that they staid here, was long, for it was not now as in former times. Wherefore the pilgrims grew acquainted with many of the good people of the town, and did them what service they could. Mercy, as she was wont, labored much for the poor: wherefore their bellies and backs blessed her, and she was there an ornament to her profession. And, to say the truth for Grace, Phebe, and Martha, they were all of a very good nature, and did much good in their places. They were also all of them very fruitful; so that Christian's name, as was said before, was like to live in the world.

While they lay here, there came a monster out of the woods, and slew many of the people of the town. It would also carry away their children, and teach them to suck its whelps. Now, no man in the town durst so much as face this monster; but all fled when they heard the noise of his coming.

The monster was like unto no one beast on the earth. Its body was like a dragon, and it had seven heads and ten horns. It made great havoc of children, and yet it was governed by a woman. Rev. 17:3. This monster propounded conditions to men; and such men as loved their lives more than their souls, accepted of those conditions. So they came under.

Now Mr. Great-Heart, together with those who came to visit the pilgrims at Mr. Mnason's house, entered into a covenant to go and engage this beast, if perhaps they might deliver the people of this town from the paws and mouth of this so devouring a serpent.

Then did Mr. Great-Heart, Mr. Contrite, Mr. Holy-man, Mr. Dare-not-lie, and Mr. Penitent, with their weapons, go forth to meet him. Now the monster at first was very rampant, and looked upon these enemies with great disdain; but they so belabored him, being sturdy men at arms, that they made him make a retreat: so they came home to Mr. Mnason's house again.

The monster, you must know, had his certain seasons to come out in, and to make his attempts upon the children of the people of the town. At these seasons did these valiant worthies watch him, and did still continually assault him; insomuch that in process of time he became not only wounded, but lame. Also he has not made that havoc of the townsmen's children as formerly he had done; and it is verily believed by some that this beast will die of his wounds.

This, therefore, made Mr. Great-Heart and his fellows of great fame in this town; so that many of the people that wanted their taste of things, yet had a reverent esteem and respect for them. Upon this account, therefore, it was, that these pilgrims got not much hurt here. True, there were some of the baser sort, that could see no more than a mole, nor understand any more than a beast; these had no reverence for these men, and took no notice of their valor and adventures. 
